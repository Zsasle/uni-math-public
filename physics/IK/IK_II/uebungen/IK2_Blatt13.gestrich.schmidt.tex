\documentclass[sectionformat=aufgabe]{gadsescript}

\setsemester{Summer Semester 2024}%
\setuniversity{University of Konstanz}%
\setfaculty{Faculty of Science\\(Physics)}%
\settitle{Übungsblatt 10}
\setsubtitle{Davina Schmidt, Elias Gestrich}

\sisetup{exponent-mode = input}

\begin{document}
\maketitle

\section{Ebene elektromagnetische Welle}

\section{Impedanz}
$ U_{ind} = L \frac{ \dd I }{ \dd t } = L \dot I $, $ U_C = \frac{ Q }{ C }  $, $ U_R = R \cdot I $

Kirchhoff: $ U_1 = U_{ind} + U_{R} + U_{C} $,
\[
	\frac{ \dd U_e }{ \dd t } = L \frac{ \dd ^2 I }{ \dd t^2 } + R \cdot \frac{ \dd I }{ \dd t }  + \frac{ 1 }{ C } \cdot I
\]

komplexer Lösungsansatz nach dem Skript:
\[
	U_1 = U_0 \exp i \omega t 
\]
$ \implies  $ sinus/cosinus Kurve
\[
	I = I_0 \exp i ( \omega t + \varphi )
\]
$ \implies  $ selbe Peridendauer, aber möglicher weiße Phasenverschoben?

oben Einsetzen:
\[
	i \omega U_1 = - \omega ^2 L I + i \omega R I + \frac{ 1 }{ C } I
\]
also
\[
	Z_1 = \frac{U_1}{ I }  = \left( i \omega L + R - i\frac{ 1 }{ \omega C } \right)
\]

$ Z_2 = \frac{ U_2 }{ I } = R  $


$ U_2 = R \cdot I = R \cdot \frac{ U_1 }{ R + i\left( \omega L - \frac{ 1 }{ \omega C }  \right)  }  $
\[
 	U_2 = \frac{ 1 }{ 1 + i \left( \omega \frac{ L }{ R } - \frac{ 1 }{ \omega CR }  \right)  } U_1
	= \frac{ 1 - i \left( \omega \frac{ L }{ R } - \frac{ 1 }{ \omega CR }  \right) }{ 1 + \left( \omega \frac{ L }{ R } - \frac{ 1 }{ \omega CR }  \right) ^2 } U_1
\]
\[
	\left| U_2 \right| = \sqrt{\frac{ 1 + \left( \omega \frac{ L }{ R } - \frac{ 1 }{ \omega CR }  \right) ^2 }{ \left( 1 + \left( \omega \frac{ L }{ R } - \frac{ 1 }{ \omega CR }  \right) ^2 \right) ^2 }} \left| U_1 \right| = \sqrt{\frac{ 1 }{ 1 + \left( \omega \frac{ L }{ R } - \frac{ 1 }{ \omega C R }  \right) ^2 } } \left| U_1 \right| 
\]

Also maximal wenn
\[
	\omega \frac{ L }{ R } - \frac{ 1 }{ \omega C R } = 0 \iff LC = \frac{ 1 }{ \omega^2 } \iff \omega = \frac{ 1 }{ \sqrt{LC}  } 
\]



\end{document}
