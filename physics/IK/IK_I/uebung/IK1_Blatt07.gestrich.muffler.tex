\documentclass[sectionformat = exercise]{gadsescript}

\setsemester{Winter Semester 2023/2024}%
\setuniversity{University of Konstanz}%
\setfaculty{Faculty of Science\\(Physics)}%

\begin{document}
\maketitle
\section{Resonanzkurve und Phasenverschiebung}

\begin{enumerate}[label=\alph*)]
	\item ~
		\begin{figure}[H]
			\centering
			\begin{tikzpicture}
				\begin{axis}[
					xmin= 0, xmax= 10,
					ymin= 0, ymax = 6,
					axis lines = left,
					xlabel = {$ \overline{\omega} $ in $ \omega_0 $},
					ylabel = {$ |A| $ in $ \frac{ F }{ m } \cdot \qty{ 1 }{ \second }  $},
				]
					\addplot[domain=0:10, samples=100, smooth]{1/sqrt((x-1)^2 + 4 * 0.1^2 * x^2)};
				\end{axis}
			\end{tikzpicture}
			\caption{Amplitude in Abhängigkeit von $ \overline{\omega}  $}
			\label{amplitude_in_abhaengigkeit_von_omega}
		\end{figure}
		Ein Hochpunkt existiert, wenn $ \sqrt{(\overline{\omega}^2 - \omega_0^2)^2 + 4\beta^2\overline{\omega}^2 }  $ einen Tiefpunkt hat, also wenn $ (\overline{\omega}^2 - \omega_0^2)^2 + 4\beta^2\overline{\omega}^2 $ einen Tiefpunkt hat, also wenn $ 4\overline{\omega} (\overline{\omega}^2 - \omega_0^2 + 2\beta^2) = 0 $, und da für $ \overline{\omega}  $ kein Hochpunkt, aber $ \overline{\omega} = \sqrt{\omega_0^2 - 2\beta^2}  $ ein Hochpunkt, also in die Funktion eingesetzt:
		\begin{align*}
			|A_{max}| &= \frac{ F }{ m } \frac{ 1 }{ \sqrt{(\omega_0^2 - 2\beta^2 - \omega_0^2)^2 + 4\beta^2(\omega_0^2 - 2\beta^2)} }\\
			~&= \frac{ F }{ m } \frac{ 1 }{ \sqrt{4\beta^4 - 8\beta^4 + 4\beta^2\omega_0^2} }\\
			~&=  \frac{ F }{ m } \frac{ 1 }{ \sqrt{4\beta^2(\omega_0^2 - \beta^2}) }\\
			~&= \frac{ F }{ 2m\beta \sqrt{\omega_0^2 - \beta^2}  } .
		\end{align*}
		Also wird die Amplitude kleiner wenn die Dämpfung größer wird.\\
		Für $ \beta \to 0 $, gilt $ |A| = \frac{ F }{ m } \frac{ 1 }{ \sqrt{(\overline{\omega}^2 - \omega_0^2)^2 + 4\beta^2\overline{\omega}^2 } } \to \frac{ F }{ m } \frac{ 1 }{ \overline{\omega}^2 - \omega_0^2 } $.\\
		Für $ \overline{\omega} \to 0 $, gilt $ |A| = \frac{ F }{ m } \frac{ 1 }{ \sqrt{(\overline{\omega}^2 - \omega_0^2)^2 + 4\beta^2\overline{\omega}^2 } } \to \frac{ F }{ m } \frac{ 1 }{ \omega_0^2 } $.\\
		Für $ \overline{\omega} \to \infty $, gilt $ |A| = \frac{ F }{ m } \frac{ 1 }{ \sqrt{(\overline{\omega}^2 - \omega_0^2)^2 + 4\beta^2\overline{\omega}^2 } } \to \frac{ F }{ m } \frac{ 1 }{ \overline{\omega}^2 } \to 0 $.
	\item ~
		\begin{figure}[H]
			\centering
			\begin{tikzpicture}
				\begin{axis}[
					xmin= 0, xmax= 1,
					ymin= 0, ymax = 1,
					axis lines = left,
					xlabel = {$ \beta $ in $ \frac{\omega_0}{2\pi } $},
					ylabel = {$ f $ in $ \frac{\omega_0}{2\pi } $},
				]
					\addplot[domain=0:0.707106, samples=100]{sqrt(1-2 * x^2)};
				\end{axis}
			\end{tikzpicture}
			\caption{Frequenz bei maximaler Amplitude in Abhängigkeit der Dämpfung $ \beta $}
			\label{no}
		\end{figure}
		joa bei $ \beta = 0 $ ist $ f = \frac{\omega_0}{ 2\pi  }  $ und bei sehr großer Reibung ist $ f = 0 $
	\item ~
		\begin{figure}[H]
			\centering
			\begin{tikzpicture}
				\begin{axis}[
					xmin= 0, xmax= 6,
					ymin= -2, ymax = 4,
					axis lines = left,
					xlabel = {$ f $ in $ \frac{ 1 }{ T_0 }  $},
					ylabel = {$ \varphi $ },
				]
					\addplot[domain=0:1, samples=100]{rad(atan(x/(x^2 - 1))+180*0)};
					\addplot[domain=1:6, samples=100]{(rad(atan(x/(x^2 - 1))))};
				\end{axis}
			\end{tikzpicture}
			\caption{Phasenverschiebung über anregender Frequenz}
			\label{Phasenverschiebung}
		\end{figure}
	Für $ x^2 \to 1 $ geht $ \frac{ x }{ \sqrt{x^2 - 1}  }  $ gegen $ \infty $ und somit $ \arctan \frac{ x }{ x^2 - 1 } \to \frac{\pi }{ 2 }  $.
\end{enumerate}

\section{Gedämpfter harmonischer Oszillator}
$ m = \frac{ 4 }{ 3 } \pi R^3 \cdot \rho = \qty{ 0.0209 }{ \kilogram }  $
\begin{enumerate}[label=\alph*)]
	\item $ \beta = \frac{6\pi \eta R}{ 2m } = \qty{ 0.0450 }{ \per\second }  $\\
		$ T_0 = \frac{2\pi }{ \omega_0 } = 2\pi \sqrt{\frac{ m }{ D } } = \qty{ 0.407 }{ \second }  $\\
		Da $ \beta^2 - \omega_0^2 < 0 $ gilt $ \omega = \sqrt{\omega_0^2 - \beta^2}  $ gilt:\\
		$ T = \frac{2\pi }{ \omega } = \frac{2\pi }{ \sqrt{\beta^2-\omega_0^2}  } = \qty{ 0.407 }{ \second }  $ 
	\item Da $ \beta^2 - \omega_0^2 < 0 $ gilt $ \omega = \sqrt{\omega_0^2 - \beta^2}  $ gilt:\\
		$ x(t) = A*\exp(-\beta t) \left( \sin(\omega t + \varphi) \right) $, also
		\begin{align*}
			\Lambda &= - \ln \frac{\hat{x}_{n+1}}{ \hat{x}_n } \\
			~ &= -\ln \left| \frac{ \exp \left(-\beta \left(t_n+\frac{ T }{ 2 }\right)\right) \sin \left(\omega \left(t_n + \frac{ T }{ 2 } \right) + \varphi \right) }{ \exp (-\beta t_n) \sin (\omega t_n + \varphi) } \right|\\
			~ &= -\ln \left| \frac{ \exp \left(-\beta \left(t_n+\frac{ T }{ 2 }\right)\right) \sin \left(\omega t_n + \omega\frac{ 2\pi  }{ \omega } + \varphi \right) }{ \exp (-\beta t_n) \sin (\omega t_n + \varphi) } \right|\\
			~ &= -\ln \left| \frac{ \exp \left(-\beta \left(t_n+\frac{ T }{ 2 }\right)\right) \sin (\omega t_n + \varphi + \pi) }{ \exp (-\beta t_n) \sin (\omega t_n + \varphi) } \right|\\
			~ &= -\ln \left| \frac{ -\exp \left(-\beta \left(t_n+\frac{ T }{ 2 }\right)\right) \sin (\omega t_n + \varphi) }{ \exp (-\beta t_n) \sin (\omega t_n + \varphi) } \right|\\
			~ &= -\ln \frac{ \exp \left(-\beta \left(t_n+\frac{ T }{ 2 }\right)\right) }{ \exp (-\beta t_n) }\\
			~ &= -\ln \exp \left(-\beta \left(t_n+\frac{ T }{ 2 }\right)\right) + \ln \exp (-\beta t_n)\\
			~ &= -\left(-\beta \left(t_n+\frac{ T }{ 2 }\right)\right) + (-\beta t_n)\\
			~ &= \beta \left(t_n+\frac{ T }{ 2 }\right) -\beta t_n\\
			~ &= \beta \frac{ T }{ 2 }\\
			~ &= \num{ 0.00915 } \\
		\end{align*}
	\item 
		\begin{align*}
			\frac{ \hat{x}_{n+4}  }{ \hat{x}_n } &= \frac{ \hat{x}_{n+4 }}{ \hat{x}_{n+3} } \cdot \frac{ \hat{x}_{n+3}}{ \hat{x}_{n+2} } \cdot \frac{ \hat{x}_{n+2}}{ \hat{x}_{n+1} } \cdot \frac{ \hat{x}_{n+1}}{ \hat{x}_{n+0} }\\
			~ &= (\exp -\Lambda)^4 \\
			~ &= \exp -4\Lambda \\
			~ &= \qty[exponent-mode = input]{ 96.4 }{ \percent } \\
		\end{align*}
	\item Die Energie, die in Wärmeenergie umgeführt wurde, ist die Energie, welche nicht mehr in der Schwingung hängt, da bei der größten Auslenkung die Geschwindigkeit null ist, ist die gesamte Energie in der Spannenergie, also:
		\begin{align*}
			E_{span_n} - E_{span_{n+4}} &= \frac{ 1 }{ 2 } D \hat{x}_n^2 - \frac{ 1 }{ 2 } D \hat{x}_{n+4}^2 \\
			~ &= \frac{ 1 }{ 2 } D \left( \hat{x}_{n}^2 - \hat{x}_{n}^2\exp (-8\Lambda) \right)  \\
			~ &= \frac{ 1 }{ 2 } D \hat{x}_{n}^2 \left( 1 - \exp (-8\Lambda) \right) \\
			~ &= \frac{ 1 }{ 2 } D \hat{x}_{0}^2 \left( 1 - \exp (-8\Lambda) \right) \\
			~ &= \qty{ 1.76e-3 }{ \joule } \\
		\end{align*}
	\item Der aperiodische Grenzfall ist bei $ \beta^2 - \omega_0^2 = 0 $, also
		\begin{align*}
			\beta^2 &= \omega_0^2 \\
			\frac{6^2\pi^2 \eta^2 R^2}{ 2^2m^2 } &= \frac{ D }{ m }  \\
			\frac{9\pi^2 \eta^2 R^2}{ m^2 } &= \frac{ D }{ m }  \\
			\eta^2 &= \frac{ Dm }{ 9\pi^2R^2 }  \\
			\eta &= \sqrt{\frac{ Dm }{ 9\pi^2R^2 }}  \\
			\eta &= \frac{\sqrt{Dm}}{ 3\pi R }  \\
			\eta &= \qty{ 3.43 }{ \newton\second\per\square\metre }  \\
		\end{align*}
		
\end{enumerate}

\end{document}
