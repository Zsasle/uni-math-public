\documentclass[sectionformat = aufgabe]{gadsescript}

\setsemester{Winter Semester 2023/2024}%
\setuniversity{University of Konstanz}%
\setfaculty{Faculty of Science\\(Physics)}%
\settitle{Übungsblatt 12}

\begin{document}
\maketitle
\section{Elastischer Stoß in zwei Dimensionen}
Die Billardkuglen stoßen in einem Winkel von $ \frac{ \pi }{ 2 }  $ aufeinander, da wegen Impulserhaltung
\[
	m_0 \vec v_0  + m_1 \vec v_1 = m_0 \vec v_0^{\prime} + m_1 v_1^{\prime} 
\]
gilt und da $ m_0 = m_1 $ und $ v_1 = 0 $ gilt:
\[
	\vec v_0 = \vec v_0^{\prime} + \vec v_1^{\prime} 
\]
und somit Bilden die Vektoren ein Dreieck mit den Seiten $ v_0, v_0^\prime, v_1^\prime $.\\
Außerdem gilt Energieerhaltung, also gilt:
\[
	\frac{ 1 }{ 2 } m_0 v_0^2 + \frac{ 1 }{ 2 } m_1 v_1^2 = \frac{ 1 }{ 2 } m_0 v_0^{\prime 2} + \frac{ 1 }{ 2 } m_1 v_1^{\prime 2} 
\]
Also gilt nach oben genannten Vorraussetzungen
\[
	v_0^2 = v_0^{\prime 2} + v_1^{\prime 2} 
\]
Das bedeuted aber, dass das ober erwähnte Dreieck ein rechtwinkliges Dreieck ist mit der Hypothenuse $ v_0 $, also ist zwischen $ v_0^{\prime} $ und $ v_1^\prime $ ein rechter Winkel.

\section{Stoßexperimente in einer Dimension}
\begin{enumerate}[label=\alph*)]
	\item Es gilt
		\begin{align*}
			m_1 v_1 + m_2 v_2 &= m_1 v_1^{\prime} + m_2 v_2^{\prime} \\
			\qty{ 10 }{ \kilogram } \cdot v_2 &= \qty{ 80 }{ \kilogram } \cdot v_1^{\prime}  \\
			v_1^{\prime} &= \frac{ v_2 }{ 8 } \\
		\end{align*}
	\item Es gilt
		\begin{align*}
			m_2 v_2 &= m_1 v_1^{\prime} + m_2 v_2^{\prime}  \\
			m_2 v_2 - m_2 v_2^{\prime} &= m_1 v_1^{\prime}  \\
			m_2 \left( v_2 - v_2^{\prime} \right) &= m_1 v_1^{\prime}  \\
			v_1^{\prime} &= \frac{ m_2 }{ m_1 } \left( v_2 - v_2^{\prime} \right) \\
		\end{align*}
		Außerdem gilt, da es ein elastischer Stoß ist:
		\begin{align*}
			\frac{ 1 }{ 2 } m_2 v_2^2 &= \frac{ 1 }{ 2 } m_1 v_1^{\prime 2} + \frac{ 1 }{ 2 } m_2 v_2^{\prime 2}  \\
			m_2 v_2^2 &= \frac{ m_2^2 }{ m_1 } \left( v_2 - v_2^{\prime} \right)^2 + m_2 v_2^{\prime 2} \\
			0 &= v_2^{\prime 2} \left( 1 + \frac{ m_2 }{ m_1 } \right) - 2 v_2^{\prime} v_2 \cdot \frac{ m_2 }{ m_1 } + \left( \frac{ m_2 }{ m_1 } - 1 \right) v_2^2  \\
			0 &= \frac{ 8 }{ 7 } v_2^{\prime 2} - \frac{ 2 }{ 7 } v_2^{\prime} v_2 - \frac{ 6 }{ 7 } v_2^2  \\
			0 &= 8 v_2^{\prime 2} - 2 v_2^{\prime} v_2 - 6 v_2^2  \\
			v_{2_{1,2} }^{\prime} &= \frac{ 2 v_2 \pm \sqrt{4 v_2^2 + 4 \cdot 6 \cdot 8 v_2^2}  }{ 16 } \\
			v_2^{\prime} &= \frac{ 2 v_2 - \sqrt{196 v_2^2} }{ 16 } \\
			v_2^{\prime} &= \frac{ 2 v_2 - \sqrt{196} }{ 16 } \\
			v_2^{\prime} &= -\frac{ 3 }{ 4 } v_2 \\
		\end{align*}
		Also gilt:
		\begin{align*}
			v_1^{\prime} &= \frac{ m_2 }{ m_1 } \left( v_2 - v_2^{\prime} \right) \\
			v_1^{\prime} &= \frac{ 1 }{ 7 }  \left( v_2 + \frac{ 3 }{ 4 } v_2 \right) \\
			v_1^{\prime} &= \frac{ 1 }{ 4 }  v_2 \\
		\end{align*}
	\item $ v_1^{\prime} = \frac{ 1 }{ 2 } v_2 $.
		\begin{align*}
			m_2 v_2 &= \frac{ 1 }{ 2 } m_1 v_2 + m_2 v_2^{\prime} \\
			v_2 - \frac{ m_1 }{ 2m_2 } v_2 &= v_2^{\prime} \\
			-\frac{ 5 }{ 2 } v_2 &= v_2^{\prime} \\
		\end{align*}
		Und
		\begin{align*}
			\frac{ 1 }{ 2 } m_2 v_2^2 + E &= \frac{ 1 }{ 8 } m_1 v_2^2 + \frac{ 25 }{ 8 } m_2 v_2^2 \\
			E &= \frac{ m_1 + 21 m_2 }{ 8 } v_2^2 \\
			E &= \qty{ 35 }{ \kilogram } v_2^2 \\
		\end{align*}
	\item 
		\begin{enumerate}[label=\alph*)]
			\item inelastisch, da die kinetische Energie nach dem Stoß nach dem Stoß geringer ist, als vor dem Stoß
			\item elastisch, da die kinetische Energie vor und nach dem Stoß gleich sind
			\item superelastisch, da die kinetische Energie nach dem Stoß größer ist, als vor dem Stoß.
		\end{enumerate}
		
\end{enumerate}

\end{document}
