\documentclass[11pt]{scrartcl}
\usepackage[utf8]{inputenc}
\usepackage[ngerman]{babel}
\usepackage{amsmath}
\usepackage{amssymb}
\usepackage{mathtools}
\usepackage{enumitem}
\usepackage{tabularray}
\usepackage[dvipsnames]{xcolor}

\makeatletter
\newcommand{\qed}{\@ifstar\@qedwhite\@qedblack}
\DeclareRobustCommand{\@qedblack}{%
        \ifmmode \tag*{$\blacksquare$}%
        \else \leavevmode\unskip\penalty9999 \hbox{}\nobreak\hfill\quad\hbox{$\blacksquare$}%
        \fi%
}
\DeclareRobustCommand{\qedwhite}{%
        \ifmmode \tag*{$\Box$}%
        \else \leavevmode\unskip\penalty9999 \hbox{}\nobreak\hfill\quad\hbox{$\Box$}%
        \fi%
}
\makeatother

\newcommand{\N}{\mathbb{N}}
\newcommand{\Q}{\mathbb{Q}}
\newcommand{\R}{\mathbb{R}}
\newcommand{\C}{\mathbb{C}}

%====================================================================
% SETZEN SIE HIER DIE NUMMER DES ÜBUNGSBLATTES UND IHRE(N) NAMEN EIN
% FILL IN THE SHEET NUMBER AND YOUR NAME(S)
%====================================================================
\newcommand{\sheetNum}{9} % Nummer des Übungsblatt / Sheet number
\newcommand{\studentOne}{Alexandros Paraskevaidis} % Name 1
\newcommand{\studentTwo}{Elias Gestrich} % Name 2
%====================================================================

\setkomafont{sectioning}{\normalfont\normalcolor\bfseries}
\usepackage[left=2.5cm,right=2.5cm,top=2.0cm,bottom=2.5cm]{geometry}

% Hyperlinks
\usepackage{hyperref}

% Pseudocode
\usepackage[german,vlined,longend,ruled,linesnumbered]{algorithm2e}
\SetKw{KwDownTo}{downto}
\SetKw{KwAnd}{and}
\SetKw{KwOr}{or}
\DontPrintSemicolon

% Grafiken
\usepackage{graphicx}
\graphicspath{{img/}}
\usepackage{tikz}
\usetikzlibrary {arrows.meta}

% Makros für Aufgaben und Teilaufgaben
\usepackage{marginnote}
\reversemarginpar
\setlength{\parindent}{0cm}
\newcommand{\task}[1]{\subsubsection*{#1}}
\newcommand{\subtask}[1]{\marginnote{#1)}}


\begin{document}

%====================================================================
\begin{small}
\begin{minipage}{0.5 \linewidth}
  Algorithmen und Datenstrukturen\\
  Sommersemester 2024
\end{minipage}
\begin{minipage}{0.5\linewidth}
  \begin{flushright}
    \studentOne\\
    \studentTwo
  \end{flushright}
\end{minipage}
\end{small}
\begin{center}
\begin{sffamily}\Large\bfseries \sheetNum. Übungsblatt\end{sffamily}
\end{center}
%====================================================================


\task{Aufgabe 1 - Raumzeit}
$ [1,3], [3.5, 5], [5.5, 11.5], [17, 19], [21, 22], [25, 27], [30, 31] $\\
$ I_1 = \left\{ [1, 5], [5.5, 11.5], [14, 24], [25, 27], [30, 31] \right\}  $,
$ I_2 = \left\{ [1, 3], [3.5, 5], [7, 15], [16, 26] \right\}  $,
$ I_3 = \left\{ [1, 18], [20, 24] \right\}  $,
$ I_4 = \left\{ [2, 4], [4, 6], [8, 12], [17, 19], [21, 22], [23, 32] \right\}  $,
$ I_5 = \left\{ [10, 20], [21, 27] \right\}  $.\\
$ D(26) = 4, D(I) = 5 $

\task{Aufgabe 2 - Profitabel}
Sortiere Jobs $ i $ nach Profit $ p_i $.
Gehe nun vom Job mit höchsten Profit zum niedrigsten und führe folgenden Algorithmus aus:
Teile den Job $ i $ der Zeiteinheit vor $ t_i $ ein, sodass er genau dann fertig wird, wenn die Deadline ist.
Ist zu dieser Zeit schon ein Job geplant und der Zeitslot vorher ist nach Zeitpunkt 0, dann teile den Job zu dieser Zeit an.
Sonst lasse den Job und er wird nicht ausgeführt.\\
Korrektheit: Da alle Jobs eine Zeiteinheit dauern, müsste wenn der Algorithmus nicht Korrekt ist, ein Job durch einen anderen ersetzt werden.
Da der Algorithmus aber immer den Profitablesten nimmt, heißt dies, dass ein profitablerer Job, durch einen ersetzt wird, der weniger profitabel ist.

\task{Aufgabe 3 - Manchmal optimal}
\begin{description}
	\item[$ \left| S \right| \leq 1 $] trivial, man kann sowieso nichts aufteilen.
	\item[$ \left| S \right| = 2 $] Sei $ S = \left\{ s_1, s_2 \right\}  $ mit $ s_1 \leq  s_2 $, dann ist $ 0 \leq 2 s_1 \iff s_2 - s_1 \leq s_2 + s_1 $, also ist es schlauer $ s_1 $ und $ s_2 $ in unterschiedliche Haufen zu packen.
	\item[$ \left| S \right| = 3 $] Sei $ S = \left| s_1, s_2, s_3 \right|  $ mit $ s_1 \leq s_2 \leq s_3 $.
		Dann sortiert der Greedy Algorithmus $ S_1 = \left\{ s_3 \right\} , S_2 = \left\{ s_1, s_2 \right\}  $, also zu zeigen
		$ \left| s_3 + s_2 - s_1 \right| \geq   \left| s_3 - s_2 - s_1 \right| \leq \left| s_3 + s_1 - s_2 \right| $,
		hierfür, für $ s_1 + s_2 \leq  s_3 $:
		$ \left| s_3 + s_2 - s_1 \right| \geq s_3 \geq s_3 - s_2 - s_1 = \left| s_3 - s_2 - s_1 \right|   $,
		$ \left| s_3 + s_1 - s_2 \right| = s_3 + s_1 - s_2 \geq s_3 - s_1 - s_2 = \left| s_3 - s_1 - s_2 \right|  $
		für $ s_1 + s_2 > s_3 $:
		$ | s_3 + \underbrace{s_2 - s_1}_{\geq 0} | \geq s_3 = s_3 + s_3 - s_3 \geq s_2 + s_1 - s_3 = \left| s_3 - s_2 - s_1 \right|   $,
		$ | \underbrace{s_3 - s_2}_{\geq 0} + s_1 | = s_3 + s_1 - s_2 \geq s_3 + s_1 - s_3 \geq s_2 + s_1 - s_3 = \left| s_3 - s_2 - s_1 \right| $, was zu zeigen war.
	\item[$ \left| S \right| = 4 $]
		Sei $ S = \left\{ s_1, s_2, s_3, s_4 \right\}  $ mit $ s_1 \leq s_2 \leq s_3 \leq s_4 $.
		Nach vorherigem Schritt sortiert der Greedy Algorithmus die ersten drei Elemente wie folt ein:
		$ S_1 = \left\{ s_4 \right\} , S_2 = \left\{ s_2, s_3 \right\}  $, für $ s_2 + s_3 \geq s_4 $ wird dann $ s_1 in S_1 $ gepackt, sonst in $ S_2 $.
		\begin{description}
			\item[$ s_2 + s_3 \geq s_4 $:] 
				Zu zeigen: $ \left| s_4 + s_1 - s_2 - s_3 \right| \leq | s_4 + s_1 + s_2 - s_3 | $:
				Für $ s_4 + s_1 \geq s_2 + s_3 $:
				$ \left| s_4 + s_1 + s_2 - s_3 \right| = s_4 + s_1 + s_2 - s_3 \geq s_4 + s_1 - s_2 - s_3 = \left| s_4 + s_1 - s_2 - s_3 \right|  $\\
				Für $ s_4 + s_1 < s_2 + s_3 $:
				$ \left| s_4 + s_1 + s_2 - s_3 \right| = s_4 + s_1 + s_2 - s_3 \geq s_3 + s_1 + s_2 - s_4 \geq s_3 + s_2 - s_4 - s_1 = \left| s_4 + s_1 - s_2 - s_3 \right|  $, analog für $ \left| s_4 + s_1 - s_2 - s_3 \right| \leq \left| s_4 + s_1 + s_3 - s_2 \right| $.\\
				Zu zeigen $ \left| s_4 - s_1 - s_2 - s_3 \right| \geq \left| s_4 + s_1 - s_2 - s_3 \right|  $:
				$ \left| s_4 - s_1 - s_2 - s_3 \right| = s_1 + s_2 + s_3 - s_4 \geq  s_2 + s_3 - s_1 - s_4 = \left| s_4 + s_1 - s_2 - s_3 \right|  $ 
			\item[$ s_2 + s_3 < s_4 $]
				Zu zeigen $ \left| s_4 - s_3 - s_2 - s_1 \right| \leq \left| s_4 + s_1 - s_2 - s_3 \right|  $:
				Für $ s_4 \geq s_1 + s_2 + s_3 $:
				$ \left| s_4 + s_1 - s_2 - s_3 \right| = s_4 + s_1 - s_2 - s_3 \geq s_4 - s_1 - s_2 - s_3 = \left| s_4 - s_3 - s_2 - s_1 \right|  $
				sonst:
				$ \left| s_4 + s_1 - s_2 - s_3 \right| = s_4 + s_1 - s_2 - s_3 \geq s_2 + s_3 + s_1 - s_4 = \left| s_4 - s_3 - s_2 - s_1 \right|  $, rest ist analog.
		\end{description}
	\item[$ \left| S \right| = 5 $]
		Wähle $ S = \left\{ 10, 8, 6, 5, 4 \right\}  $, dann sortiert der Greedy-Algorithmus: $ S_1 = \left\{ 10, 5 \right\}, S_2 = \left\{ 8, 6, 4 \right\}   $ so, dass $ \left| 10 + 5 - 8 - 6 - 4 \right| = 3 $, aber für $ S_1 = \left\{ 10, 6 \right\} , S_2 = \left\{ 8, 5, 4 \right\}  $, wäre $ \left| 10 + 6 - 8 - 5 - 4 \right| = 1 $
\end{description}

\task{Aufgabe 4 - Mit Sack und Pack}
\begin{enumerate}[label=(\alph*)]
	\item 
		Wenn alle Gegenstände das gleiche Gewicht haben, dann ist die Anzahl der Gegenstände die in den Rucksack passen fest.\\
		\textbf{Beh.:} Wenn man die teuersten Gegenstände mitnimmt, ist der Rucksack optimal gepackt.\\
		\textbf{Bew.:} Zum Widerspruch: Wir haben ein Gegenstand mitgenommen, der nicht zu den teuersten gehört, dafür haben wir einen teureren nicht mitgenommen.
		Tausche diese beiden Gegenstände, dann ist der Rucksack besser bepackt, also war der Rucksack nicht optimal gepackt.

		Wie können wir die teuersten Gegenstände mitnehmen? Sortiere in polynomieller Zeit die Gegenstände nach Wert, nicht-aufsteigend, dann stopfe solange die teuersten Gegenstände in den Rucksack, bis dieser voll ist.
	\item 
		Bins mit größe 6, Volumen von Gegenständen: 5, 2, 6, 1
		First Fit: $ \left\{ 5, 1 \right\} , \left\{ 2 \right\} , \left\{ 6 \right\}  $,
		Next Fit : $ \left\{ 5 \right\} , \left\{ 2 \right\} , \left\{ 6 \right\}, \left\{ 1 \right\}   $

		2-Approximation:
		\textbf{Beh.:} der Durchschnitt der Inhalte $ B - 1 $ Bins ist mind. die Hälfte des gesamten Fassungsvermögen.
		Betrachte: Die Fülle eines Bins ist mindestens so groß wie das Volumen, das dem letzem Bin fehlt, sonst würde der Inhalt dieses Bins in den letzen passen.
		Packe die Bins also in Zweier-Paare, sodass maximal ein Bin ohne Paar übrigbleibt, sodass Bin 1 und Bin 2 ein Paar, Bin 3 und Bin 4 ein anderes und so weiter, alle einzelnen Paare haben zusammen mindestens den Inhalt von zwei Bins, also durchschnittlich ist mindestens voll.
		Wenn man nun den Durchschnitt dieser Paare nimmt, sind wiederum alle durchschnittlich mind. zur Hälfte gefüllt.
		Der übrig bleibende Bin ist konstant maximal 1 Bin.

\end{enumerate}


\end{document}
