\documentclass[11pt]{scrartcl}
\usepackage[utf8]{inputenc}
\usepackage[ngerman]{babel}
\usepackage{amsmath}
\usepackage{amssymb}
\usepackage{mathtools}
\usepackage{enumitem}

\makeatletter
\newcommand{\qed}{\@ifstar\@qedwhite\@qedblack}
\DeclareRobustCommand{\@qedblack}{%
        \ifmmode \tag*{$\blacksquare$}%
        \else \leavevmode\unskip\penalty9999 \hbox{}\nobreak\hfill\quad\hbox{$\blacksquare$}%
        \fi%
}
\DeclareRobustCommand{\qedwhite}{%
        \ifmmode \tag*{$\Box$}%
        \else \leavevmode\unskip\penalty9999 \hbox{}\nobreak\hfill\quad\hbox{$\Box$}%
        \fi%
}
\makeatother

\newcommand{\N}{\mathbb{N}}
\newcommand{\Q}{\mathbb{Q}}
\newcommand{\R}{\mathbb{R}}
\newcommand{\C}{\mathbb{C}}

%====================================================================
% SETZEN SIE HIER DIE NUMMER DES ÜBUNGSBLATTES UND IHRE(N) NAMEN EIN
% FILL IN THE SHEET NUMBER AND YOUR NAME(S)
%====================================================================
\newcommand{\sheetNum}{1} % Nummer des Übungsblatt / Sheet number
\newcommand{\studentOne}{Alexandros Paraskevaidis} % Name 1
\newcommand{\studentTwo}{Elias Gestrich} % Name 2
%====================================================================

\setkomafont{sectioning}{\normalfont\normalcolor\bfseries}
\usepackage[left=2.5cm,right=2.5cm,top=2.0cm,bottom=2.5cm]{geometry}

% Hyperlinks
\usepackage{hyperref}

% Pseudocode
\usepackage[german,vlined,longend,ruled,linesnumbered]{algorithm2e}
\SetKw{KwDownTo}{downto}
\SetKw{KwAnd}{and}
\SetKw{KwOr}{or}
\DontPrintSemicolon

% Grafiken
\usepackage{graphicx}
\graphicspath{{img/}}
\usepackage{tikz}

% Makros für Aufgaben und Teilaufgaben
\usepackage{marginnote}
\reversemarginpar
\setlength{\parindent}{0cm}
\newcommand{\task}[1]{\subsubsection*{#1}}
\newcommand{\subtask}[1]{\marginnote{#1)}}


\begin{document}

%====================================================================
\begin{small}
\begin{minipage}{0.5 \linewidth}
  Algorithmen und Datenstrukturen\\
  Sommersemester 2024
\end{minipage}
\begin{minipage}{0.5\linewidth}
  \begin{flushright}
    \studentOne\\
    \studentTwo
  \end{flushright}
\end{minipage}
\end{small}
\begin{center}
\begin{sffamily}\Large\bfseries \sheetNum. Übungsblatt\end{sffamily}
\end{center}
%====================================================================


\task{Aufgabe 1 - Asymptotisches Wachstum}
Die Reihenfolge ist: $ b(n) \in o(d(n)), d(n) \in o(e(n)), e(n) \in o(a(n)), a(n) \in o(c(n)), c(n) \in o(f(n)) $, da:
\begin{align*}
	\lim_{n \to \infty} \frac{ b(n) }{ d(n) } &= \lim_{n \to \infty} \frac{ 12 \cdot \frac{ n }{ \sqrt{n}  } }{ \sqrt{n} \cdot \log (42 \cdot n) }  \\
	~ &= \lim_{n \to \infty} \frac{ 12 n }{ n \cdot \log (42 \cdot n ) }  \\
	~ &= \lim_{n \to \infty} \frac{ 12 }{ \log ( 42 \cdot n ) }  \\
	~ &= \lim_{n \to \infty} 0,
\end{align*}
\begin{align*}
	\lim_{n \to \infty} \frac{ d(n) }{ e(n) } &= \lim_{n \to \infty} \frac{ \sqrt{n} \cdot \log (42 \cdot n) }{ n - 213465 \cdot \log ( n^5 ) }  \\
	~ &\overset{\text{l.H.} }{=} \lim_{n \to \infty} \frac{ \frac{ \log (42 \cdot n) }{ 2 \sqrt{n}  } + \frac{ 42 \sqrt{n} }{ 42 n } }{ 1 - \frac{ 213465 \cdot 5n^4 }{ n^5 }  }  \\
	~ &= \lim_{n \to \infty} \frac{ \log ( 42 \cdot n ) + 2 }{ 2 \sqrt{n} - \frac{ 2134650 }{ \sqrt{n}  }  }  \\
	~ &= \lim_{n \to \infty} \frac{ \sqrt{n} \cdot  \log ( 42 \cdot n ) + 2 \sqrt{n}  }{ 2 \cdot  n - 2134650 }  \\
	~ &\overset{\text{l.H.} }{=} \lim_{n \to \infty} \frac{ \frac{ \log (42 \cdot n) }{ 2 \sqrt{n}  } + \frac{ 42 \sqrt{n} }{ 42 n } + \frac{ 1 }{ \sqrt{n}  }  }{ 2 }  \\
	~ &= \lim_{n \to \infty} \frac{ \log (42 \cdot n) + 4 }{ 4 \sqrt{n} } \\
	~ &\overset{\text{l.H.} }{=} \lim_{n \to \infty} \frac{ \frac{ 42 }{ 42 \cdot  n } }{ \frac{ 2 }{ \sqrt{n} } } \\
	~ &= \lim_{n \to \infty} \frac{ 1 }{ 2 \sqrt{n} } \\
	~ &= 0,
\end{align*}
\begin{align*}
	\lim_{n \to \infty} \frac{ e(n) }{ a(n) } &= \lim_{n \to \infty} \frac{ n - 213465 \cdot \log ( n^5 ) }{ 16^{\log _4 n} }  \\
	~ &= \lim_{n \to \infty} \frac{n - 213465 \cdot 5 \cdot \log n}{ 4^{2\log _4n}  } \\
	~ &= \lim_{n \to \infty} \frac{n - 213465 \cdot 5 \cdot \log n}{ n^2 } \\
	~ &\overset{\text{l.H.} }{=} \lim_{n \to \infty}  \frac{ 1 - \frac{ 213465 \cdot 5 }{ n } }{ 2n } \\
	~ &= \lim_{n \to \infty} \frac{ n - 213465 \cdot 5 }{ 2n^2 } \\
	~ &\overset{\text{l.H.} }{=} \lim_{n \to \infty} \frac{ 1 }{ 4n } \\
	~ &= 0,
\end{align*}
\begin{align*}
	\lim_{n \to \infty} \frac{ a(n) }{ c(n) } &= \lim_{n \to \infty} \frac{ 16^{\log _4 n} }{n^5 + 111}  \\
	~ &\overset{\text{l.H.} }{=} \lim_{n \to \infty} \frac{ 2n }{ 5n^4 } \\
	~ &= \lim_{n \to \infty} \frac{ 2 }{ 5n^3 } \\
	~ &= 0,
\end{align*}
Für $ c(n) \in o(f(n)) $:
\[
	2^{\sqrt{n} } = \exp \left(\sqrt{n} \ln 2\right) = \sum_{k=0}^{\infty} \frac{(\ln 2)^k \sqrt{n} ^k}{ k! } \geq  \frac{(\ln 2)^{12} \sqrt{n} ^{12} }{ 12! } = \frac{ (\ln 2)^{12} n^6 }{ 12! } 
\]
Also wenn $ c(n) \in o\left(\frac{ ( \ln 2 )^{12} n^6 }{ 12! }\right) $, dann gilt auch $ c(n) \in o(f(n)) $.
\begin{align*}
	\lim_{n \to \infty} \frac{ n^5 - 111 }{ \frac{ (\ln 2)^{12} }{ 12! } n^6 } &\overset{\text{l.H.} }{=} \lim_{n \to \infty} \frac{ 5n^4 }{ \frac{6 (\ln 2)^{12} }{ 12! } n^5 }  \\
	~ &= \lim_{n \to \infty} \frac{ 5 \cdot 12! }{ 6 (\ln 2)^{12} n }  \\
	~ &= 0 \qed
\end{align*}

\task{Aufgabe 2 - Laufzeit}
Algorithmus 1: $ f_1(n) \in \mathcal{O} \left(\sqrt{n} \right) $,\\
Algorithmus 2: $ f_2(n) \in \mathcal{=} \left( \frac{ n }{ 2 } - 1 \right) = \mathcal{O} \left( \frac{ n }{ 2 }  \right) = \mathcal{O} (n) $,\\
Algorithmus 3: $ f_3(n) \in \mathcal{O} (n \cdot (n - 1) = \mathcal{O} (n^2 - n) = \mathcal{O} (n^2) $\\
Algorithmus 4: $ f_4(n) \in \mathcal{O} (1) $ (für $ \mathbb{N} = \left\{ 0, 1, 2, \dotsc \right\} : f_4(n) \in \mathcal{O} (2) = \mathcal{O} (1) $
Algorithmus 5: $ f_5 \in \mathcal{O} (1) $, da für $ n_1 \coloneqq  1000001 $ gilt, dass für $ n \in \mathbb{N} : n \geq n_1 : f_5(n) = f_4(n) \in \mathcal{O} (1) $

\task{Aufgabe 3 - Rekursion}
\begin{enumerate}[label=\alph*)]
	\item 
		\begin{itemize}
			\item \textbf{Vor.}: $ T(0) = 0 $ und $ T(n) = 3 \cdot T(n - 1) + 5 $ für alle $ n \geq 1 $\\
				\textbf{Beh.}: Für $ n \geq 1 $ gilt $ T(n) = 2.5 \cdot (3^n - 1) $ \\
				\textbf{Bew.}:
				\begin{description}
					\item[I.A.] $ n = 1 : T(n) = T(1) = 3 \cdot T(0) + 5 = 3 \cdot 0 + 5 = 5 = 2.5 \cdot 2 = 2.5 \cdot (3 - 1) = 2.5 \cdot (3^n - 1) $
					\item[I.S.] $ n \curvearrowright n + 1 $
						\begin{description}
							\item[I.V.] $ T(n) = 2.5 \cdot (3^n - 1) $
						\end{description}
						Zu zeigen $ T(n+1) = 2.5 \cdot (3^{n+1} -1) $:
						\begin{align*}
							T(n + 1) &= 3 \cdot T(n) + 5 && | \quad \text{I.V.}  \\
								 &= 3 \cdot ( 2.5 \cdot (3^n - 1) ) + 5 \\
								 &= 2.5 \cdot ( 3 \cdot 3^n - 3 ) + 2.5 \cdot 2 \\
								 &= 2.5 \cdot ( 3^{n + 1} - 1 ) \qed
						\end{align*}
				\end{description}
			\item \textbf{Vor.}: $ T(n) = 1 $ für $ n \leq 3 $ und $ T(n) = T(n - 1) + \frac{ n }{ 3 }  $ sonst\\
				\textbf{Beh.}: Für $ n \geq 1 $ gilt $ T(n) \leq n^2 $ \\
				\textbf{Bew.}:
				\begin{description}
					\item[I.A.] $ n = 1 : T(n) = T(1) = 1 \leq 1^2 = n^2 $
					\item[I.S.] $ n \curvearrowright n + 1 $
						\begin{description}
							\item[I.V.] $ T(n) \leq n^2 $
						\end{description}
						Zu zeigen $ T(n+1) \leq ( n + 1 )^2 $:
						\begin{align*}
							T(n + 1) &= T(n) + \frac{ n }{ 3 } && | \quad \text{I.V.} \\
								 &\leq n^2 + \underbrace{\frac{ n }{ 3 }}_{\leq 2n + 1} \\
								 &\leq n^2 + 2n + 1 \\
								 &\leq ( n + 1 )^2 \qed
						\end{align*}
				\end{description}
			\item \textbf{Vor.}: $ T(1) = 1 $ für $ n = 1 $ und $ T(n) = T(\frac{ n }{ 2 }) + 2 $ \\
				\textbf{Beh.}: Für $ n \in \left\{ 2^i : i \in \N_0 \right\} $ gilt $ T(n) \leq 2 \log _2(n) + 1 $ \\
				Also für $ i \in \N _0  $ gilt $ T(2^i) \leq  2 \log _2(2^i) + 1 = 2\cdot i \cdot \log _2(2) + 1 = 2i + 1 $ \\
				\textbf{Bew.}:
				\begin{description}
					\item[I.A.] $ i = 0 : T(2^i) = T(2^0) = T(1) = 1 \leq 1 = 2 \cdot 0 + 1 = 2i + 1 $
					\item[I.S.] $ i \curvearrowright i + 1 $
						\begin{description}
							\item[I.V.] $ T(2^i) \leq 2i + 1 $
						\end{description}
						Zu zeigen $ T(2^{i + 1}) \leq 2(i + 1) + 1 $:
						\begin{align*}
							T(2^{i + 1}) &= T(\frac{ 2^{i + 1}  }{ 2 } ) + 2 \\
								     &= T(2^i) + 2 && | \quad \text{I.V.} \\
								     &\leq 2i + 1 + 2 \\
								     &\leq 2(i + 1) + 1 \qed
						\end{align*}
				\end{description}
		\end{itemize}
	\item $ T(n) = 1 $ für $ n \leq 42 $, $ T(n) = T(n - 1) + 1 $ sonst.
\end{enumerate}

\task{Aufgabe 4 - Lahme Schafe}
Man kann eine Art ``Bubble-Sort'' verwenden. Also man sagt das erste Schaf wäre das langsamste und vergleicht dieses dann mit dem nächstem Schaf, schaut welches langsamer ist, sagt dieses wäre das langsamste und vergleicht das langsamste Schaf wieder mit dem nächsten Schaf usw.

\begin{algorithm}
	\caption{$ find\_slowest\_sheep(Sheep[~]~ sheep) $}
	$ slowest\_sheep \leftarrow sheep[0] $\\
	\For{ $ i = 1; i < sheep.count; $ }{
		$ slowest\_sheep \leftarrow slower\_sheep(slowest\_sheep, sheep[i] $ 
	}
	\Return $ slowest\_sheep $
\end{algorithm}

Für $ n $ Schafe ist die Laufzeit also $ 1 + n - 1 = n $, folglich linear.
Das funktioniert, da uns nur das langsamste Schaf interessiert und nicht wie schnell jedes Schaf im Vergleich zu jedem anderem Schaf ist.
Wenn wir so also wissen, dass Schaf $ 3 $ schneller ist als Schaf $ 4 $, dann ist es irrelevant zu wissen ob Schaf $ 5 $ langsamer ist als Schaf $ 3 $, da dies nichts darüber aussagt ob Schaf $ 5 $ langsamer ist als Schaf $ 4 $, somit brauchen wir nur zu wissen wie sich Schaf $ 5 $ zu Schaf $ 4 $ verhält, wenn wir davon ausgehen, dass das Schaf $ 4 $ das langsamste Schaf ist, das wir bisher betrachtet haben.

%Bei zwei Schafen, lasse die beiden Schafe gegeneinander laufen. Für $ n $ Schafe, nehme das langsamste aus den $ n - 1 $ Schafen und vergleiche dieses mit dem $ n $-ten Schaf, das langsamere der beiden ist das langsamste der $ n $ Schafe.

%\begin{algorithm}
	%\caption{$ find\_slowest\_sheep(Sheep[~]~ sheep) $}
	%\If{$ sheep.number == 2 $}{
	%	\Return $ slower\_sheep(sheep[0],~ sheep[1]) $
	%}\Else{
	%	\Return $ slower\_sheep(find\_slowest\_sheep(sheep[1,~ \dotsc,~ n - 1],~sheep[n]) $
	%}
%\end{algorithm}

%Für zwei Schafe ist somit die Laufzeit 1 und für $ n $ ist die Laufzeit um eins größer als für $ n - 1 $, also verläuft die Laufzeit linear.

\end{document}
