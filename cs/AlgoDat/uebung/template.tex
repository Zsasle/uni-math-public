\documentclass[11pt]{scrartcl}
\usepackage[utf8]{inputenc}
\usepackage[ngerman]{babel}

%====================================================================
% SETZEN SIE HIER DIE NUMMER DES ÜBUNGSBLATTES UND IHRE(N) NAMEN EIN
% FILL IN THE SHEET NUMBER AND YOUR NAME(S)
%====================================================================
\newcommand{\sheetNum}{1} % Nummer des Übungsblatt / Sheet number
\newcommand{\studentOne}{Vorname Nachname} % Name 1
\newcommand{\studentTwo}{Vorname Nachname} % Name 2
\newcommand{\studentThree}{Vorname Nachname} % Name 3
%====================================================================

\setkomafont{sectioning}{\normalfont\normalcolor\bfseries}
\usepackage[left=2.5cm,right=2.5cm,top=2.0cm,bottom=2.5cm]{geometry}

% Hyperlinks
\usepackage{hyperref}

% Pseudocode
\usepackage[german,vlined,longend,ruled,linesnumbered]{algorithm2e}
\SetKw{KwDownTo}{downto}
\SetKw{KwAnd}{and}
\SetKw{KwOr}{or}
\DontPrintSemicolon

% Grafiken
\usepackage{graphicx}
\graphicspath{{img/}}
\usepackage{tikz}

% Makros für Aufgaben und Teilaufgaben
\usepackage{marginnote}
\reversemarginpar
\setlength{\parindent}{0cm}
\newcommand{\task}[1]{\subsubsection*{#1}}
\newcommand{\subtask}[1]{\marginnote{#1)}}


\begin{document}

%====================================================================
\begin{small}
\begin{minipage}{0.5 \linewidth}
  Algorithmen und Datenstrukturen\\
  Sommersemester 2024
\end{minipage}
\begin{minipage}{0.5\linewidth}
  \begin{flushright}
    \studentOne\\
    \studentTwo\\
    \studentThree
  \end{flushright}
\end{minipage}
\end{small}
\begin{center}
\begin{sffamily}\Large\bfseries \sheetNum. Übungsblatt\end{sffamily}
\end{center}
%====================================================================


\task{Aufgabe 1 - Pseudocode}
Mit der \texttt{algorithm}-Umgebung aus dem Paket \href{https://www.ctan.org/pkg/algorithm2e}{algorithm2e} lässt sich wunderschöner Pseudocode schreiben.

\begin{algorithm}[H]
\caption{InsertionSort(int[] $A$)}
\For{$j=2$ \KwTo $A.length$}{
  $key = A[j]$\;
  $i = j-1$\;
  \tcp{Verschiebe die größeren Elemente aus $A[1\dots j-1]$ nach rechts}
  \While{$i > 0$ \KwAnd $A[i] > key$}{
    A[i+1] = A[i]\;
    i = i-1
  }
  $A[i+1] = key$\;
}
\end{algorithm}

\task{Aufgabe 2 - Grafiken}

\subtask{a} \href{http://ipe.otfried.org/}{Ipe} ist ein Zeichenprogramm, welches Grafiken im PDF-Format erzeugt und es erlaubt \LaTeX{} einzubinden. Eine Beispieldatei, die sich mit Ipe editieren lässt, befindet sich im Ordner \texttt{img}.
\begin{figure}[h]
\centering
\includegraphics{ipe-example}
\caption{Eine mit Ipe erzeugte Grafik}
\end{figure}

\subtask{b} Mit Hilfe des Pakets \href{https://www.ctan.org/pkg/pgf}{TikZ} lassen sich Grafiken direkt im \LaTeX-Quellcode erzeugen. Hilfreiche Links zum Umgang mit TikZ sind
\begin{itemize}
  \item \url{http://cremeronline.com/LaTeX/minimaltikz.pdf} - Kurze Einführung
  \item \url{http://www.texample.net/tikz/examples/} - Viele Beispiele
  \item \url{http://mirrors.ctan.org/graphics/pgf/base/doc/pgfmanual.pdf} - Ausführliches Handbuch mit vielen Beispielen
\end{itemize}

\begin{figure}[h]
\centering
\tikzset{bul/.style={circle,fill=black,inner sep=1pt,draw=black}}
\begin{tikzpicture}[scale=.4]
 \draw (0,0)node[bul,label=below:{$t_2$}](t2){} (1,0)node[bul,label=right:{$s_1$}](s1){} (4,0)node[bul,label=left:{$s_2$}](s2){} (5,0)node[bul,label=above:{$t_1$}](t1){};
 \draw (1.5,-1.5)node[bul](ll){} (3.5,-1.5)node[bul](lr){} (1.5,1.5)node[bul](ul){} (3.5,1.5)node[bul](ur){};
 \draw [->](s1) to (ul);\draw [->] (s1) to (ll);
 \draw [->](s2) to (ur);\draw [->] (s2) to (lr);
 \draw [->](ul) to (t2);\draw [->] (ll) to (t2);
 \draw [->](ur) to (t1);\draw [->] (lr) to (t1);
 \draw [red,line width=2pt] (ul) -- (ur) (ll) -- (lr);
\end{tikzpicture}
\caption{Eine mit TikZ erzeugte Grafik}
\end{figure}

\end{document}
