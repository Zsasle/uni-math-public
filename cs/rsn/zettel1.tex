\documentclass{gadsescript}

\settitle{Übungsblatt 1}

\begin{document}
\maketitle

\section{Rechnersysteme: Geschichte}
\begin{enumerate}[label=\alph*)]
	\item Wilhelm Schickard war Astronom und Mathematiker und hat eine ``Rechenuhr'' zum Addieren und Subtrahieren erfunden (1623) (hat den Plan an Kepler geschickt)
	\item Die Pascaline ist eine Rechenmaschine, die mithilfe eines Räderwerkes addieren und subtrahieren kann
	\item Wolgang von Kempelen schaffte dies daher, da er keinen Automaten baute, sondern eine von einem von innen von Menschen gesteuerte ``Maschine'', die mithilfe von Magneten die Schachfiguren bewegen konnte
	\item Anscheinend dadurch, dass scherzhaft versucht wurde Fehlfunktionen auf Ungeziefer zurückzuführen, wie in Briefen von Thomas Edison, oder Wiliam Orton, so wirklich populär wurde es für Computer nachdem Grace Hopper erzählte, nach einer Fehlfunktion eines Relais eine Motte entfernt und ins Logbuch geklebt wurde
	\item das Moorsche Gesetz wurde von dem Herrn Moor entdeckt und sagt aus, dass sich alle paar Jahre die Dichte an Tranistoren auf Computerchips verdoppelt. Ich glaube aber, das Moorsche Gesetz gilt seit ein paar Jahren nicht mehr
\end{enumerate}

\section{Rechnersysteme: Hardware}
\begin{enumerate}[label=\alph*)]
	\item Physik -> Chips und (Logik-)Gatter -> Hardware-Plattform -> Maschinen-Sprache -> Assembler-Sprache -> Software 
	\item woha warum?
\end{enumerate}
\end{document}

