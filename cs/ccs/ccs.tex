\documentclass{gadsescript}

\settitle{Konzepte der Informatik}
\setsubtitle{concepts of computer science}

\begin{document}
\maketitle

Folien liegen auf ILIAS\\
Jäger-Honz: für Übungsaufgaben\\

Tutoren:
\begin{itemize}
	\item Maxima Gebhardt
	\item Fabienne Nowak
	\item Sven Geerdes
	\item Marina Haugers
\end{itemize}

\section{Organisatorisches}
\begin{itemize}
	\item Konzeption
		\begin{itemize}
			\item \textit{Konzepte der Informatik} und \textit{Programmierkurs} bilden das Modul ``Informatik 1''\\
				Es gibt Vorlesung von 2017/18 auf ILIAS
			\item Konzepte der Informatik:
				\begin{itemize}
					\item pro Woche 4 SWS Vorlesung
					\item Übungsbetreib
						\begin{itemize}
							\item wöchentliche Aufgaben
							\item Korrektur durch die Tutoren
							\item 5 Tutorien in Präsenz
							\item Anmeldung/Einteilung bis Mittwoch 25.10. in ZEUS
							\item Tutorien Donnerstag/Freitag !! Tutorien sind das Wichtigste !!
						\end{itemize}
				\end{itemize}
		\end{itemize}
\end{itemize}

Leistungsnachweise
\begin{itemize}
	\item Konzepte der Informatik: \textit{benotete} Klausur nach der Vorlesungszeit
		\begin{itemize}
			\item Teilnahmevorraussetzung ist erfolgreiches Bearbeiten der Übungen mit
				\begin{itemize}
					\item mindestens 50\% aller Punkte aus den Aufgaben
				\end{itemize}
			\item Ersttermin: Ende der Vorlesungszeit
			\item Nachtermin: Ende der volesungsfreien Zeit
		\end{itemize}
\end{itemize}

yeah halt keine Anwesenheitspflicht

Übungsbetrieb
\begin{itemize}
	\item Ein Aufgabenblatt pro Woche
	\item Ausgabe der Übungsaufgaben Diesntags
	\item Bearbeitungszeit bis {\color{red} Montag 10:00 Uhr}
	\item Abgabe elektroneisch
		\begin{itemize}
			\item im ILIAS
			\item automatischer und manueller Abschreibertest!
			\item Abschreiben ist Betrug und kann zur Exmatrikulation führen!
		\end{itemize}
\end{itemize}

Übungsbetrieb\\
\indent\quad
\begin{minipage}{0.5\textwidth}
	Denken ist wie googlen nur krasser
\end{minipage}

\begin{itemize}
	\item Kdl-Übungsaufgaben (sind auf Engslisch)
		\begin{itemize}
			\item Lösungen können auf Deutsch oder auf Englisch abgegeben werden.
			\item Abgabe als PDF
				\begin{itemize}
					\item \LaTeX
					\item OpenOffice lol
					\item PDF Creator ... naja ich weißt nicht
				\end{itemize}
			\item nur 1 PDF Datei
		\end{itemize}
\end{itemize}

Morgen kommt der erste Zettel raus c:

\section{Inhalt}
Wir schauen ein bisschen in die Geschichte der Informatik, wo kommen dinge her
\begin{itemize}
	\item Informationscodierung und -speicherung
		\begin{itemize}
			\item Zahlen und Zeichen
			\item Speicherbereiche
			\item Datentypen und Datenstrukturen
			\item Hashing
		\end{itemize}
	\item Algorithmik
		\begin{itemize}
			\item Darstellung und Eigenschaften
			\item Abstrakte Datenstrukturen
			\item Grundlegende algorithmische Konszepte
			\item Berechnungs- und Speicherkomplexität
		\end{itemize}
	\item Theoretische Informatit
		\begin{itemize}
			\item Automaten
			\item Grammatiken undn formale Sorachen
			\item Berechenbarkeit und ...
		\end{itemize}
	\item Parallelisierung
\end{itemize}

\section{Was ist Informatik}
Sehr allgemein wird gebraucht

\begin{itemize}
	\item Kofferwort aus Information und Mathematik oder Elektronik oder Automatik
	\item Erstamal erwähnt im Jahr 1957
	\item Seit 1968 als Bezeichung ...
\end{itemize}
computer science anscheinend im wandel, weil es nicht notwendigerweiße um computer geht

\section{Kurze Geschichte der Informatik}
\begin{itemize}
	\item ca 300 v. Chr. : Euklid von Alexandria entwickelt das Euklid-Verfahrung zur berechnung des ggT
	\item 	um 820: Al-Chwarizi entwickelt Verfahren zu r Lpsung bekannter mathematikscher Probleme
	\item 	1524: Rechengesetze zum Dezimalsystem von Adam Riese
	\item 	1623: Wilhelm Schickard erfindet die erste Rechenmaschiene (Rechenuhr)
	\item 	1642: Blaise Pascal konstruiert eine Rechenmaschine mit sechs Stellen
	\item 	1679: Gottfried Wilhelm Leibniz baut ein Maschine für die vier Grundrechenarten
		\begin{itemize}
			\item es ist unwürdig gebildete Leute mit rechnen verbringen zu lassen, wenn mit Maschienen selbst die einfältigsten Leute auf das richtige Ergebnis kommen
			\item erst die Replika 1960 oder so hat funktioniert, weil voher die Feinmechanik zu schlecht war
		\end{itemize}
	1759: Ph\item ilipp Matthäus Hahn baut die erste alltagstaugliche Rechenmaschine
	\item 	1822: Prinzip der ``Ànalytical Engine'' durch Chales Babbage erstes Computerprogramm von Ada Lovelace
	\item 	1890: Hermann Hollerith erfindet die Lochkarte
	\item 	um 1900: Gottlob Frege entwickelt eine formale Sprache mit Beweismethoden; ``Begriffsschrigt''
	\item 	1930-1940: Arbeiten an der Theorie der Berechenbarkeit, eÉntscheidbarkeit und Vollständikgkeit durch
		\begin{itemize}
			\item adam turing ...
		\end{itemize}
		\begin{itemize}
			\item 2. Weltkrieg -> deutsche Enigma
			\item Alan Turin -> ``Die Bombe''
		\end{itemize}
	\item konrad Zuse
		\begin{itemize}
			\item 1941 erste funktionstüchtiger Computer Z3
			\item 1945: esrte universelle Programmiertsprache ``Plankalkühl''
		\end{itemize}
	\item 1944: Bau der teilweise programmgesteuertern Rehenalnlage...
	\item 1946: Fertigstellung des ersten elektronisches Rechners ENIAC durch John Presper Eckert und John ...
	\item 1949 John von Neumann
	\item 1953: Fortran
	\item 1959: TRADIC Transisorrechner
	\item 1959: funktionale Programmiersprache LISP
	\item ...
	\item 1972 Einführung der Programmiersprache C
	\item 1979: Entwicklung der objektorientierten Programmiersprache C++ durch Bjarne Stroustroup
	\item 1985-1988: allgemeine Vergügbarkeit
	\item 1995: Java
	\item ...
\end{itemize}

Als nächstes Systemcodierung und -speicher\\

\end{document}
