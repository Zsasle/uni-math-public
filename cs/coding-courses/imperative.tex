\documentclass{gadsescript}

\usepackage[ngerman]{babel}
\settitle{Programmierkurs 1 (imperative Sprache)}
\setsubtitle{imperative programming}

\begin{document}
\maketitle

\section{Organisatorisches}

Greifen verschiedene Dinge aus der Vorlesung IdK auf\\
Angewandte vorlesung\\
Programmieren ist fast das wichtigeste das wir lernen in Informatik\\
Wir schauen uns nicht nur die Welt an, sondern setzten auch um, ``schöpferisch''\\

Literautur
\begin{itemize}
	\item exemplarisch in Folien
	\item exemplarisch in Code-Beispilen
	\item https://en.cppreference.com
	\item google und co
	\item Bjarne Stroustrup: Eine Tour
		\begin{itemize}
			\item \textbf{scheint gute Bücher zu haben auch in kdi empfohlen}
		\end{itemize}
\end{itemize}
~\\\par

\textbf{Übungsaufgaben über GitLab}
\begin{itemize}
	\item Jeweils drei aufgaben ( 10P + 5P + 5P )
	\item Kriterien: Korrektheit, Code Style, Lines of Code
	\item Einzelabgabe
\end{itemize}
Zum Bestehen müssen pro Woche $ \geq 8 $ von 20 Punkten ( 2 Strikes ) und insgesamt $ \geq 140 $ von 200 Punkten erzielt werden\\
Ab und an gibt es auch Bonusaufgaben\\

\section{Einführung}

\begin{itemize}
	\item Programmiersprache ist eine \textbf{formale} Sprache zur Spezifikation von Algorithmen und Datenstrukturen
	\item Programmiersprachen \textbf{abstrahieren} den Computer, der die Programme ausführen soll
	\item Ein \textbf{Compiler} übersetzt das Programm in eineRechenvorschrigt, die der Computer versteht
	\item Verschiedene Programmiersprachen sind oft auf verschiedenen \textbf{Abstraktionsebenen} angesiedelt
		\begin{itemize}
			\item Wieviel ``sieht'' der Mensch von der Maschine?
			\item Welche Aufgaben übernimmt der Mensch?
			\item Welche Aufgaben übernimmt der Compiler?
		\end{itemize}
\end{itemize}

\end{document}
