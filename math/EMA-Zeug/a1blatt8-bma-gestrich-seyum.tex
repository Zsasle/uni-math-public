\documentclass{gadsescript}
\defaultmathtitle
\settitle{BMA}

\begin{document}
\maketitle

\textbf{Vor.:} $ a_1, b_1 \in \R $ und $ 0 < a_1 < b_1 $ Seien $ (a_n)_{n \in \N } $ und $ (b_n)_{b \in \N } $ Folgen reeler Zahlen, rekursiv definiert durch 
\[
	a_{n+1} \coloneqq \frac{ 2 a_n b_n }{ a_n + b_n } \text{ und } b_{n+1} \coloneqq \frac{ a_n + b_n }{ 2 } \quad (n \in \N ).
\]

\textbf{Beh.:}
\begin{enumerate}[label=(\alph*)]
	\item
		\begin{enumerate}[label=(\roman*)]
			\item $ 0 < a_n < b_n, $ 
			\item $ a_n \leq a_{n+1} $ und $ b_n \geq b_{n+1} $ 
			\item $ [a_{n+1}, b_{n+1}] \subset [a_n, b_n] $
		\end{enumerate}
	\item $ (a_n)_{n \in \N } und (b_n)_{n \in \N } $ konvergieren gegen den gleichen Limes und $ (a_n - b_n)  $ ist eine Nullfolge
	\item $ \exists ! x \in \bigcap_{n \in \N } [a_n, b_n]  $ und für dieses $ x $ gilt $ x = \sqrt{a_1b_1}  $ 
\end{enumerate}
\begin{proof*}
	\begin{enumerate}[label=(\alph*)]
		\item
			\begin{enumerate}[label=(\roman*)]
				\item 
					\begin{description}
						\item[I.V.] $ \exists n : 0 < a_n < b_n $ 
						\item[I.A.] $ n = 1 $\\
							$ 0 < a_n < b_n $ gegeben.
						\item[I.S.] $ n \curvearrowright n + 1 $\\
							zu zeigen $ 0 < a_{n+1} < b_{n+1} $, also zu zeigen:
							\[
								0 < \frac{ 2 a_n b_n }{ a_n + b_n } < \frac{ a_n + b_n }{ 2 } 
							\]
							\begin{alignat*}{3}
								0 &< \frac{ 2a_nb_n }{ a_n + b_n } &&< \frac{ a_n + b_n }{ 2 } \\
								0 &< 4a_nb_n &&< (a_n + b_n)^2 \\
								0 &< 4a_nb_n &&< a_n^2 + 2a_nb_n + b_n^2 \\
								-4a_nb_n &< 0 &&< a_n^2 - 2a_nb_n + b_n^2 \\
							\underbrace{-4a_nb_n}_{\text{da } a_n > 0 \text{ und } b_n > 0} &< 0 &&< (a_n - b_n)^2 \qed
							\end{alignat*}
							
					\end{description}
				\item 
					\begin{align*}
						a_n &\leq 1 \cdot a_n\\
						a_n &\leq \frac{ 2b_n }{ b_n + b_n } a_n \quad | \quad \text{da } a_n + b_n \leq b_n + b_n \iff \frac{ 1 }{ b_n + b_n } \leq \frac{ 1 }{ a_n + b_n } \\
						a_n &\leq \frac{ 2b_n }{ a_n + b_n } a_n \\
						a_n &\leq a_{n + 1} \\
					\end{align*}
					\begin{align*}
						b_n &\geq 1 \cdot b_n\\
						b_n &\geq \frac{ 2b_n }{ 2 } \quad | \quad \text{da } a_n + b_n \leq b_n + b_n \\
						b_n &\geq \frac{ a_n + b_n }{ 2 } \\
						b_n &\geq b_{n + 1} \qed
					\end{align*}
				\item Also zu zeigen: $ \forall x \subset [a_{n+1}, b_{n+1}]: x \in [a_n, b_n] $.
					Sei $ x \in [a_{n+1}, b_{n+1}] $ gegeben, zu zeigen: $ x \in [a_n, b_n] $.\\
					Also zu zeigen $ a_n \leq x \leq b_n $. Es gilt, da $ x \in [a_{n+1}, b_{n+1}] $:
					$ a_n \leq a_{n+1} \leq x \leq b_{n+1} \leq b_n $\qed
			\end{enumerate}
		\item zu zeigen $ \lim_{n \to \infty} a_n = \lim_{n \to \infty} b_n $.\\
			Da $ (a_n), (b_n) $ monoton und durch sich gegenseitig beschränkt, gilt $ (a_n), (b_n) $ konvergent, also $ \exists a \coloneqq \lim_{n \to \infty} a_n $ und $ \exists b \coloneqq \lim_{n \to \infty} b_n $, wähle ein solches $ a $ und $ b $. Zu zeigen $ a = b $.\\
			Es gilt:
			\begin{align*}
				\lim_{n \to \infty} b_{n+1} &= \lim_{n \to \infty} \frac{ a_n + b_n }{ 2 }\\
				\lim_{n \to \infty} b_n &= \frac{ a + b }{ 2 }  \\
				b &= \frac{ a + b }{ 2 }  \\
				2b &= a + b \\
				b &= a\\
			\end{align*}
			Und zu zeigen $ (b_n - a_n)_{n \in \N } $ eine Nullfolge.
			Also $ 0 = \lim_{n \to \infty} a_n - \lim_{n \to \infty} a_n = \lim_{n \to \infty} b_n - \lim_{n \to \infty} a_n = \lim_{n \to \infty} (b_n - a_n)  $\qed
		\item zu zeigen Existiert $ x \in \bigcap_{n \in \N} [a_n, b_n] $ und Eindeutigkeit dieses $ x $'es
			Setze $ x \coloneqq \sqrt{a_1b_1}  $ zu zeigen $ x \in \bigcap_{n \in \N} [a_n, b_n] $
			\begin{description}
				\item[I.V.] $ (a_n b_n)_{n \in \N } = a_1b_1 $ 
				\item[I.A.] $ n = 1 $\\
					$ a_1 b_1 = a_1b_1 $ gegeben.
				\item[i.S.] $ n \curvearrowright n+1 $
					\begin{align*}
						a_{n+1}b_{n+1} &= \frac{ 2a_nb_n }{ a_n + b_n } \cdot \frac{ a_n + b_n }{ 2 } \\
						a_{n+1}b_{n+1} &= a_n b_n \quad | \quad \text{nach I.V.} \\
						a_{n+1}b_{n+1} &= a_1b_1 \\
					\end{align*}
			\end{description}
			Also $ \lim_{n \to \infty} a_nb_n = a_1b_1 $\\
			Also $ a_1b_1 = \lim_{n \to \infty} a_nb_n = \lim_{n \to \infty} a_n \lim_{n \to \infty} b_n = \lim_{n \to \infty} a_n \cdot \lim_{n \to \infty} a_n = \left( \lim_{n \to \infty} a_n \right)^2 $\\
			Also $ \lim_{n \to \infty} b_n = \lim_{n \to \infty} a_n = \sqrt{a_1b_1} = x$\\
			Also $ x \geq a_n $ und $ x \leq b_n $, also $ x \in \bigcap_{n \in \N } [a_n, b_n] $. Noch zu zeigen Eindeutigkeit von $ x $\\
			Da $ \lim_{n \to \infty} \diam([a_n, b_n]) = \lim_{n \to \infty} (b_n - a_n) = 0 $ gilt das Intervallschachtelungsprinzip, also $ \exists ! guenter \in \bigcap_{n \in \N } [a_n, b_n] $, also folgt für ein soches $ guenter $, $ x = guenter $ und da $ guenter $ eindeutig ist, ist auch $ x $ eindeutig.\qed
	\end{enumerate}
\end{proof*}
\begin{enumerate}[label=(\alph*)]
	\setcounter{enumi}{3}
	\item 
		\begin{align*}
			[a_1,b_1]&=[1,2]\\
			[a_2,b_2]&=\left[\frac{ 4 }{ 3 },\frac{ 3 }{ 2 }  \right]\\
			[a_3,b_3]&=\left[\frac{ 24 }{ 17 },\frac{ 17 }{ 12 }  \right]\\
			[a_4,b_4]&=\left[\frac{ 816 }{ 577 },\frac{ 577 }{ 408 }  \right]\\
		\end{align*}
		\begin{align*}
			x_1&=1\\
			x_2&=\frac{ 3 }{ 2 }\\
			x_3&=\frac{ 17 }{ 12 }\\
			x_4&=\frac{ 577 }{ 408 } \\
		\end{align*}
\end{enumerate}
\end{document}
