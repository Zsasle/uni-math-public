\documentclass[consecutivenumbering]{gadsescript}

\settitle{Plenumsübung Lineare Algebra}

\begin{document}
\maketitle

\section*{Organisatorisches}
\textbf{Sebastian} Krapp\\
sebastian.krapp@uni-konstanz.de\\

Literatur
\begin{itemize}
	\item Merlin Carl: Wie kommt man darauf?
	\item Dierk Schleicher: Eine Einladung in die Mathematik
	\item Regula Krapf: Elementare Grundlagen der Hochschulmathematik
	\item Clara Löh et al.: Quod erat knobelandum
\end{itemize}

{\color{red}\scshape
\noindent Regelmäßig website checken\\
Schick Konstanz Uni -> Lehre -> Lina
}

Resultate:
\begin{itemize}
	\item Theorem: Wichtiger \textbf{Hauptsatz}/Fundamentalsatz
	\item Proposition/Satz: normale Resultate
		\begin{itemize}
			\item (Satz von = Theorem)
		\end{itemize}
	\item Korollar: Folgesatz, fogt schnell aus anderem Resultat
	\item Lemma: Hilfssatz, wird \textbf{nur} für Beweis anderer Resultate benötigt
		\begin{itemize}
			\item (Lemma + Name = Theorem)
		\end{itemize}
\end{itemize}
\textbf{zu lernen: alles was einen Namen hat}

\begin{task}[Schreibe das Resultat ``Divisionsalgorithmus'' auf]
	Seien $ a, b \in \Z $ mit $ b > 0 $\\
	Dann gilt : $ \exists! q, r \in \Z $ mit\\
	$ 0 \leq r < b $ und\\
	$ a = bq + r $
\end{task}

\begin{alignat*}{2}
	&52&&0 : 3 = 173 \text{ Rest } 1\\
	-&3&&~\\
	&22&&~\\
	-&20&&~\\
	&01&&0\\
	&-&&9\\
	~&&&1
\end{alignat*}

\begin{proof}
	$ a = 520, b = 3, ( q = 173 ), ( r = 1 ) $\\
	$ 520 > 0 \rightsquigarrow \text{ Fall 1.} $\\
	$ 0 < 520 < 3 \text{ nein } \rightsquigarrow \text{ Zeile überspringen } $\\
	$ 520 \geq 3 \rightsquigarrow \text{ ja } $\\
	Start der Prozedur\\
	$ S = \{ s \in \N : s \times 3 \leq 520 \} $\\
	$ 1 \in S, \text{ da } 1 \times 3 \leq 520 \implies S \neq \emptyset $.
	$ S $ ist endlich, denn $ 1000 \notin S $, da $ 1000 \times 3 \nleq 520 $ und Zahlen größer $ 1000 $ nicht, da ..., also hat $ S $ maximal $ 999 $ Elemente.\\
	{\color{violet}
	Beweistechnik\\
	Jede nach oben beschänkte Teilmenge der natürlichen Zahlen besitzt ein Maximum
	}\\
	\fbox{$ q \coloneqq \operatorname{max} S $}\\
	\[ 173 \times 3 = 519 \leq 520 \rightsquigarrow 173 \in S \]
	\[ 174 \times 3 = 522 > 520 \rightsquigarrow 174 \notin S \]
	also ist $ 173 $ das Maximum von $ S $\\
	\fbox{ $ r = 520 - 173 \times 3 = 1 $ }\\
	{\color{red}\scshape Ende der Prozedur}
	Fall 2: $ a \leq 0 $\\
	$ \quad \overset{\text{Fall $ A $}}{\rightsquigarrow} a = 0 \rightsquigarrow q = 0, r = a $\\
	$ \quad \overset{\text{Fall $ B $}}{\rightsquigarrow} a < 0 \rightsquigarrow $\\
	\begin{itemize}
		\item Wende FAll 1 auf $ - a $ an
		\item Erhalte $ r\prime, q\prime $ aus Fall 1
		\item Falls $ r\prime = 0 $, setze $ q = -q\prime, r = 0 $
		\item Falls $ r\prime \neq 0 $, setze $ q = -(q\prime + 1), r = b - r\prime $
	\end{itemize}
\end{proof}

\end{document}
