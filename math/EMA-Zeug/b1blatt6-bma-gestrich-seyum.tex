\documentclass{gadsescript}

\settitle{BMA}
\defaultmathtitle
\begin{document}
\maketitle
\textbf{Vor.:} $ (R, +, \cdot ) $ ein kommutativer Ring mit Eins, den wir im Folgenden mit $ R $ bezeichnen. Ein Element $ x \in R $ heißt \textit{Nullteiler von $ R $}, wenn es ein $ y \in R, y \neq 0 $ mit $ x \cdot y = 0 $ gibt. Der Ring $ R $ heißt \textit{nullteilerfrei}, wenn 0 der einzieger Nullteiler von $ R $ ist. Ein Element $  x \in R $ heißt \textit{Einheit von $ R $}, wenn es ein $ y \in R $ mit $ x \cdot y = 1 $ gibt.
\begin{enumerate}[label=(\alph*)]
	\item \textbf{Beh.:} Es sei $ x \in R , x \neq 0 $. Betrachte die Abbildung $ f: R \to R $ mit $ f(r) = x \cdot r $ für $ r \in R $. Zeige: Ist $ x $ kein Nullteiler von $ R $, dann ist die Abbildung f injektiv.\\
		\begin{proof*}
			Zu zeigen, $ x $ kein Nullteiler, dann $ f $ injektiv.\\
			Sei $ x \in R $ gegeben und $ x $ kein Nullteiler, zu zeigen $ f $ injektiv. Also zu zeigen $ \forall r_1, r_2 \in R: f(r_1) = f(r_2) \implies r_1 = r_2 $.\\
			Seien $ r_1,r_2 \in R $ gegeben, zu zeigen $ f(r_1) = f(r_2) \implies r_1 = r_2 $. Sei $ f(r_1) = f(r_2) $ gegeben, zu zeigen $ r_1 = r_2 $.
			\begin{align*}
				f(r_1) &= f(r_2) \\
				x \cdot r_1 &= x \cdot r_2 \\
				x \cdot r_1 + (-(x \cdot r_2)) &= x \cdot r_2 + (-(x \cdot r_2)) \\
				x \cdot r_1 + x \cdot (-r_2) &= 0 \\
				x \cdot (r_1 - r_2) &= 0 \\
			\end{align*}
			Und da $ x $ kein Nullteiler, muss $ r_1 - r_2 $ gleich 0 sein, damit $ x \cdot (r_1 - r_2) = 0 $.\qed
		\end{proof*}
	\item \textbf{Beh.:} Ist $ x \in R $ eine Einheit von $ R $, dann ist $ x $ kein Nullteiler von $ R $.
		\begin{proof*}
			Zu zeigen, wenn $ x \in R $ eine Einheit von $ R $, dann ist $ x $ kein Nullteiler von $ R $.\\
			Sei $ x \in R $ eine Einheit von $ R $, so zeigen, $ x $ kein Nullteiler von $ R $.\\
			Nach Vorraussetzung $ \exists y_1 \in R : x \cdot y_1 = 1 $. Wähle ein solches $ y_1 $.\\
			Wir führen einen Beweis durch Widerspruch und nehmen dazu an, dass $ x $ ein Nullteiler ist, also $ \exists y_0 : x \cdot y_0 = 0 \wedge y_0 \neq 0 $\\
			Da die Assoziativität in Ringen gilt, folgt:
			\begin{align*}
				(y_0 \cdot x)\cdot y_1 &= y_0\cdot (x \cdot y_1) \\
				0 \cdot y_1 &= y_0 \cdot 1 \\
				0 &= y_0 \\
			\end{align*}
			Da aber $ 0 \neq y_0 $, führt dies zu einem Widerspruch, also war unsere Annahme falsch, dass $ x $ ein Nullteiler ist, also kann $ x $ kein Nullteiler sein\qed
		\end{proof*}
	\item \textbf{Beh.:} Ist $ R $ ein Körper, dann ist $ R $ nullteilerfrei.
		\begin{proof*}
			Zu zeigen, wenn $ R $ ein Körper, dann ist $ R $ nullteilerfrei.\\
			Also zu zeigen $ \forall x \in R : x $ ist kein Nullteiler, oder $ x = 0 $\\
			\begin{description}
				\item[Fall 1:] $ x = 0 $, dann gilt trivialer weiße die Aussage
				\item[Fall 2:] $ x \neq 0 $, dann gilt $ x \in R\setminus{0} $, und da $ R $ ein Körper $ \exists x^{-1} mit x \cdot x^{-1} = 1 $, also ist $ x $ eine Einheit von $ R $, also gilt nach der b), dass $ x $ kein Nullteiler\qed
			\end{description}
			
		\end{proof*}
		
\end{enumerate}


\end{document}
