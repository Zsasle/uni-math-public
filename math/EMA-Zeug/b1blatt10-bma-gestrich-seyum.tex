\documentclass{gadsescript}

\setsemester{Winter Semester 2023/2024}%
\setuniversity{University of Konstanz}%
\setfaculty{Faculty of Science\\(Mathematics and Statistics)}%
\settitle{BMA}

\begin{document}
\maketitle
\textbf{Vor.:} Es seien $ m, r \in \N  $, $ V $ ein $ \R  $-Vektorraum der Dimension $ m $ und $ W $ ein beliebiger $ \R  $-Vektorraum.
Weiter seien $ v_1, \dotsc, v_r \in V $ und $ w_1, \dotsc, w_r \in W $.
\textbf{Beh.:} $ v_1, \dotsc, v_r $ linear unabhängig $ \implies \exists f: V \to W $ mit
\[
	f(v_j) = w_j \quad (j = 1, \dotsc, r).
\]
und $ f $ eine lineare Abbildung
\begin{proof*}
	Sei $ v_1, \dotsc, v_r \in V $ linear unabhängig und $ w_1, \dotsc, w_r \in W $$ f: V \to W $ mit
	und da $ v_1, \dotsc, v_r $ linear unabhängig Existiert nach Korollar 13.9. eine Basis $ \mathcal{B} $ mit $ \left\{ v_1, \dotsc, v_r \right\} \subset \mathcal{B}  $, sodass $ \mathcal{B} = \left\{ v_1, \dotsc, v_r, \dotsc, v_m \right\}  $.\\
	Außerdem seien $ w_{r+1} = w_m = 0 $.\\
	Sei
	\[
		f: V \to W, v = \sum_{n=1}^{m} c_n v_n \mapsto \sum_{n=1}^{m} c_n w_n \quad \forall c_n \in \R 
	\]
	sodass $ f(v_j) = w_j $ für $ j = 1, \dotsc, r $.
	zu zeigen $ f $ lineare Abbildung, also zu zeigen $ \forall c \in \R: \forall \alpha, \beta \in V: f(c\alpha + \beta) = cf(\alpha) + f(\beta) $:
	Seien $ c \in R, \alpha, \beta \in V $ gegeben, zu zeigen $  f(c\alpha + \beta) = cf(\alpha) + f(\beta) $:
	Seien $ a_n, b_n \in \R  $ mit $ n = 1, \dotsc, r $ gegeben, sodass $ \alpha = \sum_{n=1}^{m} a_n v_n $ und $ \beta = \sum_{n=1}^{m} b_n v_n $:
	\begin{align*}
		f(c\alpha + \beta) &= f\left(c \sum_{n=1}^{m} a_n v_n + \sum_{n=1}^{m} b_n v_n \right) \\
		~&= f\left( \sum_{n=1}^{m} (ca_n +b_n) v_n \right) \\
		~&= \sum_{n=1}^{m} (ca_n +b_n) w_n \\
		~&= c \sum_{n=1}^{m} a_n w_n + \sum_{n=1}^{m} b_n w_n\\
		~&= cf\left( \sum_{n=1}^{m} a_n v_n \right) + f\left( \sum_{n=1}^{m} b_n v_n \right) \\
		~&= cf(\alpha) + f(\beta) \qed
	\end{align*}
	

	
\end{proof*}

\end{document}
