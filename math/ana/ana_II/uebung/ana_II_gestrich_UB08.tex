\documentclass[sectionformat=aufgabe]{gadsescript}

\setsemester{Summer Semester 2024}%
\setuniversity{University of Konstanz}%
\settitlelh{\today\\\semester\\Gruppe 3, Jonathan Weihing}
\setfaculty{Faculty of Science\\(Mathematics and Statistics)}%
\settitle{Übungsblatt 8}
\setsubtitle{Elias Gestrich}
\usepackage{subfig}

\begin{document}
\maketitle
\section{Taylor-Entwicklung I}
\[
	\partial_x f = \frac{ 1(x + y) - (x - y) \cdot 1 }{ ( x + y)^2 } = \frac{ 2y }{ (x + y)^2 }
\]
\[
	\partial_y f = \frac{ (-1)\cdot (x + y) - (x - y)\cdot 1 }{ ( x + y)^2 } = - \frac{ 2x }{ (x + y)^2 }
\]
\[
	\partial_x^2 f = - \frac{4y}{ (x + y)^3 } 
\]
\[
	\partial_y \partial_x f = \frac{2(x + y)^2 - 2y \cdot 2 \cdot (x + y)}{ (x + y)^4 } = \frac{ 2(x + y) - 4y }{ (x + y )^3 } = \frac{ 2(x - y) }{ (x + y)^3 } 
\]
\[
	\partial_x \partial_y f = - \frac{2(x + y)^2 - 2x \cdot 2 \cdot (x + y)}{ (x + y)^4 } = -\frac{ 2(x + y) - 4x }{ (x + y )^3 } = -\frac{ 2(-x + y) }{ (x + y)^3 } = \frac{ 2(x - y) }{ ( x + y )^3 } 
\]
\[
	\partial_y^2 f = \frac{4x}{ (x + y)^3 } 
\]


0. Glied:
\[
	\sum_{\left| \alpha \right| = 0}^{} \frac{ \partial^{\alpha} f(1, 1) }{ \alpha! } \xi^{\alpha} = f(1, 1) = 0
\]

1. Glied:
\[
	\sum_{\left| \alpha \right| = 1}^{} \frac{ \partial^{\alpha} f(1, 1) }{ \alpha! } \xi^{\alpha}
	= \frac{ \partial_x f(1, 1) }{ 1! } \xi_x
	+ \frac{ \partial_y f(1, 1) }{ 1! } \xi_y
	= \frac{ 2 }{ 2^2 } \xi_x - \frac{2}{ 2^2 } \xi_y = \frac{ 1 }{ 2 } \xi_x - \frac{ 1 }{ 2 } \xi_y = \frac{ 1 }{ 2 } \left< (1, -1), \xi \right>
\]

2. Glied:
\begin{align*}
	\sum_{\left| \alpha \right| = 1}^{} \frac{ \partial^{\alpha} f(1, 1) }{ \alpha! } \xi^{\alpha}
	&= \frac{ \partial_x^2 f(1, 1) }{ 2!0! } \xi_x^2
	% + \frac{ \partial_y \partial_x f(1, 1) }{ 1!1! } \xi_x\xi_y
	+ \frac{ \partial_x \partial_y f(1, 1) }{ 1!1! } \xi_x\xi_y
	+ \frac{ \partial_y^2 f(1, 1) }{ 0!2! } \xi_x^2 \\
	&= \frac{ \frac{ 4 }{ 2^3 } }{ 2 } \xi_x^2 + 0 + \frac{ \frac{ 4 }{ 2^3 } }{ 2 } \xi_y^2 \\
	&= \frac{ 1 }{ 4 } \xi_x^2 + \frac{ 1 }{ 4 } \xi_y^2 \\
	&= \frac{ 1 }{ 4 } \left< \xi, \xi \right> \\
	&= \frac{ 1 }{ 4 } \left\| \xi \right\| _2^2
\end{align*}

Also ist die Taylor-Entwicklung der Funktion $ f $ im Punkt $ x_0 = (1, 1) $ bis einschließlich den Gliedern 2. Ordnung:
\[
	T_{x_0} f(x_0 + \xi) = 0 + \frac{ 1 }{ 2 } \left< (1, -1), \xi \right> + \frac{ 1 }{ 4 } \left\| \xi \right\| _2^2
\]
\[
	T_{x_0} f(x) = 0 + \frac{ 1 }{ 2 } \left< (1, -1), (x - x_0) \right> + \frac{ 1 }{ 4 } \left\| (x - x_0) \right\| _2^2
\]

\section{Taylor-Entwicklung II}
\[
	\partial_x f = 0 + (y + \cos (y)) \cos (x)
\]
\[
	\partial_y f = (1 - \sin (y)) \sin (x) + 0
\]
\[
	\partial_x^2 f = - (y + \cos (y)) \sin (x)
\]
\[
	\partial_y\partial_x f = (1 - \sin (y)) \cos (x)
\]
\[
	\partial_x\partial_y f = (1 - \sin (y)) \cos (x)
\]
\[
	\partial_y^2 f = - \cos (y) \sin (x)
\]

0. Glied:
\[
	\sum_{\left| \alpha \right| = 0}^{} \frac{ \partial^{\alpha} f\left( \frac{ \pi }{ 2 } , 0 \right)  }{ \alpha! } \xi^{\alpha} = f\left( \frac{ \pi }{ 2 } , 0 \right)  = (0 + 1) \cdot 1 = 1
\]

1. Glied:
\[
	\sum_{\left| \alpha \right| = 1}^{} \frac{ \partial^{\alpha} f\left( \frac{ \pi }{ 2 } , 0 \right)  }{ \alpha! } \xi^{\alpha}
	= \frac{ \partial_x f\left( \frac{ \pi }{ 2 } , 0 \right) }{ 1! } \xi_x
	+ \frac{ \partial_y f\left( \frac{ \pi }{ 2 } , 0 \right) }{ 1! } \xi_y
	= (0 + 1) \cdot 0 \cdot \xi_x + (1 - 0) \cdot 1 \cdot \xi_y = \left< (0, 1), \xi \right>
\]

2. Glied:
\begin{align*}
	\sum_{\left| \alpha \right| = 2}^{} \frac{ \partial^{\alpha} f\left( \frac{ \pi }{ 2 } , 0 \right)  }{ \alpha! } \xi^{\alpha}
	&= \frac{ \partial_x^2 f\left( \frac{ \pi }{ 2 } , 0 \right) }{ 2!0! } \xi_x^2
	% + \frac{ \partial_y\partial_x f\left( \frac{ \pi }{ 2 } , 0 \right) }{ 1!1! } \xi_x\xi_y
	+ \frac{ \partial_x\partial_y f\left( \frac{ \pi }{ 2 } , 0 \right) }{ 1!1! } \xi_x\xi_y
	+ \frac{ \partial_y^2 f\left( \frac{ \pi }{ 2 } , 0 \right) }{ 0!2! } \xi_y^2 \\
	&= \frac{ - (0 + 1) \cdot 1 }{ 2 } \xi_x^2 + (1 - 0) \cdot 0 \xi_x\xi_y - \frac{ 1 \cdot 1 }{ 2 } \xi_y^2 \\
	&= -\frac{ 1 }{ 2 } \xi_x^2 - \frac{ 1 }{ 2 } \xi_y^2 \\
	&= - \frac{ 1 }{ 2 } \left\| \xi \right\| _2^2
\end{align*}

Also ist die Taylor-Entwicklung der Funktion $ f $ im Punkt $ x_0 = \left( \frac{ \pi }{ 2 } , 0 \right)  $ bis einschließlich den Gliedern 2. Ordnung:
\[
	T_{x_0} f(x_0 + \xi) = 1 + \left< (0, 1), \xi \right> - \frac{ 1 }{ 2 } \left\| \xi \right\| _2^2
\]
\[
	T_{x_0} f(x) = 1 + \left< (0, 1), (x - x_0) \right> - \frac{ 1 }{ 2 } \left\| (x - x_0) \right\| _2^2
\]

\section{Schrankensatz}

\begin{theorem*}[4.29]
        Sei $ \Omega \subset \R ^n $, $ x_0 \in \Omega $, $ f : \Omega \to \R ^m $.
        Sei $ \xi \in \R ^n $ so, dass
        \[
                [x_0, x_0 + \xi] \coloneqq \left\{ x_0 + t \xi : 0 \leq t \leq 1 \right\} \subset \Omega.
        \]
        Ist $ f $ stetig differenzierbar, so
        \[
                f\left( x_0 + \xi \right) - f(x_0) = \int_{0}^{1} \underbrace{Df}_{\in \R ^{m \times n} } (x_0 + t \xi ) \dd t \cdot \xi.
        \]
        Hierbei setzen wir für
        \[
                A : [a, b] \to \R ^{m \times n} ,
        \]
        wobei $ A(x) = \left( a_{ij} (x) \right) _{i = 1, \dotsc, m, j = 1, \dotsc, n}  $:
        \[
                \int_{a}^{b} A(x) \dd x \coloneqq \left( \int_{a}^{b} a_{ij} (x) \dd x \right) _{ij} .
        \]
\end{theorem*}
\begin{proof*}[Thm. 2.29]
        $ g(t) \coloneqq  f(x_0, + t\xi) $. Dann
        \begin{align*}
                f(x_0 + \xi) - f(x_0) &= g(1) - g(0) \\
                                      &= \begin{pmatrix} g_1(1) - g_1(0) \\ \vdots \\ g_m (1) - g_m(0) \end{pmatrix} \\
                                      &= \left( g_k(1) - g_k(0) \right) _{k = 1, \dotsc, m} \\
                                      &\overset{\text{Hauptsatz} }{=} \left( \int_{0}^{1} \frac{ \dd }{ \dd t } g_k (t) \dd t \right) _{k = 1, \dotsc, m}  \\                                      &= \left( \int_{0}^{1} \frac{ \dd }{ \dd t } f_k (x_0 + t \xi) \right)_{k = 1, \dotsc, m}   \\
                                      &\overset{\text{Kettenregel} }{=} \left( \left< \int_{0}^{1} \mathbf{Df_k} (x_0 + \xi) \dd t, \xi \right> \right) _{k = 1, \dotsc, m}  \\
                                      &= \int_{0}^{1} \underbrace{Df}_{\R ^{m \times n} } (x_0 + t \xi) \dd t \cdot \underbrace{\xi}_{\R ^{n \times 1} } \qed        \end{align*}
\end{proof*}
\begin{theorem*}[Schrankensatz]
        In der Situation von Thm 4.29 gilt
        \[
                \left\| f(x_0 + \xi) - f(x_0) \right\| _2 \leq M \left\| \xi \right\| _2,
        \]
        wobei
        \[
                M \coloneqq  \sup_{0 \leq t \leq 1}  \left\| \left( Df \right) \left( x_0 + t \xi \right)  \right\|
        \]
        Operatornorm, Lemma 4.22
\end{theorem*}
\begin{proof*}[Schrankensatz]
        Nach Thm. 4.29:
        \begin{align*}
                \left\| f(x_0 + \xi) - f(x_0) \right\| _2 &= \left\| \int_{0}^{1} \left( Df \right) \left( x_0 + t\xi \right) \dd t \cdot \xi \right\| _2 \\
                ~ \overset{(*)}{\leq } \int_{0}^{1} \left\| Df (x_0 + t\xi) \cdot \xi \right\| _2 \dd t \\
                ~ & \leq \int_{0}^{1} \underbrace{\left| (Df) (x_0 + t\xi) \right| }_{\leq M} \cdot \left| \xi \right| _2 \dd t \\
                ~ & \leq M \cdot \left\| \xi \right\| _2.
        \end{align*}
        \textbf{Zu $ (*) $}: Ist $ v : [a, b] \to \R ^m $ stetig.
        Dann:
        \[
                \left\| \int_{a}^{b} v(t) \dd t \right\| _{2} \leq \int_{a}^{b} \left\| v(t) \right\| _2 \dd t.
        \]
        Sei hierzu
        \[
                \eta \coloneqq \int_{a}^{b} v(t) \dd t = \begin{pmatrix} \int_{a}^{b} v_1(t) \dd t \\ \vdots \\ \int_{a}^{b} v_m (t) \dd t \end{pmatrix}
        \]
        Definiere
        \begin{align*}
                K \coloneqq \left\| \eta \right\| _2 \implies K^2 &= \left\| \eta \right\| _2^2 \\
                ~ &= \left< \eta, \eta \right> \\
                ~ &= \left< \int_{a}^{b} v(t) \dd t, \eta \right> \\
                  &= \int_{a}^{b} \left< v(t), \eta \right> \\
                  & \overset{\text{Cauchy-Schwarz} }{\leq } \int_{a}^{b} \left\| v(t) \right\| _2 \cdot \underbrace{\left\| \eta \right\| _2}_{K} \dd t \\
                  &= K \int_{a}^{b}\left\| v(t) \right\| _2 \dd t.
        \end{align*}
        \OE{} $ K \neq 0 $, kürze durch $ K $.\qed
\end{proof*}

\begin{enumerate}[label=(\alph*)]
	\item Ich denke das ist ein Tippfehler und $ \varphi(1) = y $, weil ja.
		Betrachte
		\[
			g \coloneqq f \circ \varphi_{x, y} 
		\]
		sodass $ g : \R \to \R  $. Sei $ L_g \coloneqq c \sup_{x \in \Omega} \left\| D f(x) \right\|  $
		Dann gilt nach der Kettenregel
		\[
			Dg = (Df)(\varphi_{x, y} ) \cdot D\varphi_{x, y}  \leq \sup_{x \in \Omega} \left\| D f(x) \right\| \cdot c\left\| x - y \right\| = L_g \left\| x - y \right\| 
		\]
		Den Schrankensatz (Theorem 4.30) dürfen wir anwenden mit $ x_0 = 0 $ und $ \xi = 1 $, da $ [x_0, x_0 + \xi] = [0, 1] \subset [0, 1] $.
		Nach diesem gilt:
		\[
			\left\| f(x) - f(y) \right\|_2 = \left\| g(0) - g(1) \right\|_2 = \left\| g(1) - g(0) \right\| _2 \leq M \left\| 1 - 0 \right\| _2
		\]
		mit
		\[
			M \coloneqq \sup_{0 \leq  t \leq 1} \left\| (Dg)(0 + 1 \cdot t) \right\| \leq L_g \left\| x - y \right\| _2
		\]
		Also 
		\[
			\left\| f(x) - f(y) \right\|_2 \leq  L_g \left\| x - y \right\| _2
		\]
	\item Sei $ \Omega \coloneqq B_{2}(0) \setminus \left[ \begin{pmatrix} 0 \\ 0 \end{pmatrix} , \begin{pmatrix} 1 \\ 0 \end{pmatrix}  \right] $.
		Dann gibt es trivialer Weise (?) eine stetige Abbildung $ \psi_{x, y} : [0, 1] \to \Omega $ mit $ \psi_{x, y} (0) = x, \psi_{x, y} (1) = y $.
		Betrachte die Folgen
		\[
			(x_i)_{i \in \N }: x_i = \left( 1, \frac{ 1 }{ i }   \right)
		\]
		\[
			(y_i)_{i \in \N, i > 4 }: y_i = \left( 1, - \frac{ 1 }{ i } \right)
		\]
		So, dass
		\begin{align*}
			\left\| x_i - y_i \right\| _2^2
			&= \left( 1 - 1 \right)^2
			+ \left( \frac{ 1 }{ i } - \left( - \frac{ 1 }{ i }  \right) \right)^2 \\
			&= 0 + \frac{ 4 }{ i } \\
			&\overset{i \to \infty}{\longrightarrow} 0
		\end{align*}
		Angenommen es gibt solche $ \varphi_{x_i, y_i}  $, welche die Anforderungen erfüllt.
		Der Weg von $ x_i $ nach $ y_i $ ist aber mindestens eine Länge von 2, da man einmal um den Schnitt herum muss.
		Nach Definition 3.8 Ist $ \varphi_{x_i, y_i}  $ rektifizierbar mit
		\[
			2 \leq L_{\varphi_{x_i, y_i} } = \int_{0}^{1} \left\| \varphi_{x_i, y_i} ^\prime (t) \right\| \dd t \leq \int_{0}^{1} c \left\| x_i - y_i \right\| \dd t = c \left\| x_i - y_i \right\| \overset{i \to \infty}{\longrightarrow} 0  < 2.
		\]
		Was ein Widerspruch ist.
		Also war die Annahme falsch und es existieren keine solche $ \varphi_{x_i, y_i}  $
	\item Entweder, $ f(x) $ bildet auf den Winkel $ \varphi $ der Polardarstellung ab mit $ \varphi \in (0, 2 \pi ) $, oder
		\[
			f: \Omega \to \R , x \mapsto 
			\begin{cases}
				0, & x_1 \leq 0, \\
				x_1, & x_1 > 0 \wedge x_2 > 0, \\
				-x_1 & x_1 > 0 \wedge x_2 < 0
			\end{cases}
		\]
		Dann wäre
		\begin{align*}
			\partial_1 f(x) &= \begin{cases}
				0, & x_1 \leq 0, \\
				1, & x_1 > 0 \wedge x_2 > 0,\\
				-1 & x_1 > 0 \wedge x_2 < 0
			\end{cases} \\
			\partial_2 f(x) &= 0 \\
			\partial_1^2 f(x) &= 0 \\
			\partial_1\partial_2 f(x) &= 0 \\
			\partial_2\partial_1 f(x) &= 0 \\
			\partial_2^2 f &= 0 \\
		\end{align*}
		partielle Ableitungen sind stetig, da $ \partial_1 f(x_1, x_2) \to 0, x_1 \to 0 $.
		Also $ f $ differenzierbar mit $ \sup_{x \in \Omega} \left\| Df \right\| < \infty $, aber analog zu (b) ist $ f $ nicht Lipschitzstetig
\end{enumerate}


\end{document}

