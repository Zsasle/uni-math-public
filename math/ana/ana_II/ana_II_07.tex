\section{Einführung in Maß-\& Integrationstheorie}
\begin{itemize}
	\item \textbf{Ziel:} Erweiterung der Integrationstheorie von 1-dim auf mehrdimensionale Fall.
		Teaser auf Ana III.
\end{itemize}
\textbf{Idee:}
\begin{itemize}
	\item Ana I: Erkläre Integrale für Treppenfunktion, dann Grenzübergang via $ \sup $-Norm: Regelintegral, sehr restriktiv.
	\item Ana II: Erklärt Integral für einfache Funktionen, dann Grenzübergang auf Integrallevel.\\
		$ \to  $ Ana III: ``Grund'' hierfür:
	\item Treppenfunktion:
		\[
			\sum_{i=1}^{N} \alpha_i \I_{I_i} 
		\]
		$ I_i $ Intervalle, partitionieren $ [a, b] $ 
	\item Einfache Funktionen:
		\[
			\sum_{i=1}^{N} \alpha_i \I_{\Omega_i} 
		\]
		$ \Omega_i $ ``messbare'' Mengen.
\end{itemize}

\subsection{\mathsec{\sigma}{}-Algebren \& Messbarkeit}
\begin{definition}
	Sei $ \Omega  $ eine Menge, $ \mathfrak{A} \subset \mathcal{P} (\Omega) $ (Potenzmenge) heißt \textbf{$ \sigma $-Algebra}, falls
	\begin{enumerate}[label=(\roman*)]
		\item $ \OO , \Omega \in \mathfrak{A}  $ 
		\item $ A \in \mathfrak{A} \implies A^c \in \mathfrak{A}  $ 
		\item $ A_1, A_2, \dotsc, \in \mathfrak{A} \implies \bigcup_{i = 1} ^{\infty} A_i \in \mathfrak{A}  $.
	\end{enumerate}
	$ (\Omega, \mathfrak{A} ) $ heißt \textbf{Messraum}
\end{definition}
\begin{itemize}
	\item \textbf{Unterschied zur Topologie:} Eine Topologie auf $ \Omega $ ist ein System $ \mathcal{T} \subset \mathcal{P} (\Omega) $ mit
		\begin{enumerate}[label=(\roman*)]
			\item $ \OO , \Omega \in \mathcal{T}  $ 
			\item $ A, B \in \mathcal{T} \implies A \cap B \in \mathcal{T}  $ 
			\item $ A_1, A_2, \dotsc \in \mathcal{T} \implies \bigcup_{i = 1} ^{\infty} A_i \in \mathcal{T}  $
		\end{enumerate}
		Topologie kann keine Komplementbildung!
\end{itemize}

\begin{lemma}
	Sei $ \Omega $ eine Menge \& $ \left( A_i \right) _{i \in I}  $ eine Familie von $ o $-Algebren.
	Dann ist $ \bigcap_{i \in  I} A_i $ eine $ \sigma $-Algebra.
\end{lemma}
\begin{proof*}[Lemma \ref{7.2}]
	Übungsaufgabe
\end{proof*}

Damit ist für $ \mathfrak{A} \subset \mathcal{P}  $, dann ist $ o(\mathfrak{A} ) \coloneqq \bigcap_{} \left\{ \mathfrak{B} : \mathfrak{B} \text{ $ \sigma $-Algebra mit $ \mathfrak{A}  \subset \mathfrak{B}  $}  \right\}  $ die kleinste $ \sigma $-Algebra, die $ \mathfrak{A}  $ enthält.

\begin{definition}
	Sei $ (\Omega, \mathcal{T} ) $ ein topologischer Raum. Dann heißt $ \mathfrak{B} (\Omega) \coloneqq \sigma(\mathcal{T} ) $.
	\begin{itemize}
		\item $ \mathfrak{B} (\Omega) \hat{=}  $ kleinsten $ \sigma $-Algebra, die alle offenen Mengen enthält.
		\item Auf $ \R ^n $ bilden die bezüglich der euklidischen Norm offenen Mengen eine Topologie.
			Damit ist $ \mathfrak{B} (\R ^n) $ eine kanonische $ \sigma $-Algebra auf $ \R ^n $.
	\end{itemize}
\end{definition}

\begin{definition}
	Seien $ (\Omega_1, \mathfrak{A}_1 ), (\Omega_2, \mathfrak{A} _2) $ zwei Messräume.
	$ f : \Omega_1 \to \Omega_2 $ heißt $ (\mathfrak{A} _1, \mathfrak{A} _2) $ messbar, falls
	\[
		\forall A \in \mathfrak{A} _2 : f^{-1} (A) \in \mathfrak{A} _1.
	\]
	\begin{itemize}
		\item Oft nützlich: Funktionen mit Werten in $ \overline{\R } = \R  \cup \left\{ + \infty, - \infty \right\}  $.
			Dann $ f : \Omega \to \overline{\R }, (\Omega, \Sigma ) $ Messraum,
			$ \Sigma $-messbar, $ f^{-1} ((t, \infty]) \in \Sigma , \forall t \in \R  $
	\end{itemize}
\end{definition}

\begin{example}[Einfache Funktion]
	Sei $ (\Omega, \Sigma) $ ein Messraum \& $ A_1, \dotsc, A_N \in \Sigma, \alpha_1, \dotsc, \alpha_N \in \R  $.
	Dann heißt
	\[
		x \mapsto f(x) \coloneqq \sum_{i=1}^{N} \alpha_i \I_{A_i}(x)
	\]
	\textbf{einfache Funktion}
\end{example}

\begin{definition}
	Sei $ (\Omega, \mathfrak{A} ) $ ein Messraum.
	Wir nennen $ \mu : \mathfrak{A} \to [0, \infty]  $ ein \textbf{Maß}, falls
	\begin{enumerate}[label=(\roman*)]
		\item $ \mu ( \OO ) = 0 $,
		\item Sind $ A_1, A_2, \dotsc, \in \mathfrak{A}  $ \textbf{paarweise disjunkt}, so
			\[
				\mu \left( \bigcup_{i = 1} ^{\infty} A_i \right) = \sum_{i=1}^{\infty} \mu(A_i).
			\]
			\begin{itemize}
				\item $ \mu (A_1 \cup A_2) = \mu(A_1) + \mu(A_2) - \mu(A_1 \cap A_2) $
				\item $ \mu $ heißt \textbf{Wahrscheinlichkeitsmaß}, falls $ \mu(\Omega) = 1 $
					Falls $ \mu(\Omega) < \infty $, so heißt $ \mu $ \textbf{endliches Maß}
			\end{itemize}
	\end{enumerate}
\end{definition}

Zur ``Volumenmessung'' in $ \R ^n $:
\begin{theorem}
	Es gibt  genau ein Maß $ \mathcal{L} ^n $ auf $ \mathfrak{B} \left( \R ^n \right)  $ ($ \hat{=}  $ Borelsche $ \sigma $-Algebra), das für alle $ a_1 <b_1, \dotsc, a_n < b_n $ gilt:
	\[
		\mathcal{L} ^n \left( (a_1, b_1) \times \dotsb \times (a_n, b_n) \right) = \prod_{i=1}^{n} (b_i - a_i) 
	\]
\end{theorem}

\subsection{Das Lebesgueintegral}
\begin{itemize}
	\item Hervorragende Eigenschaften bezüglich Vertauschen von Grenzprozessen.
	\item \textbf{Recap:} (Regelintegral)\\
		Für Treppenfunktion $ f = \sum_{i=1}^{N} \alpha_1 \I_{I_i} $
		\[
			\int_{0}^{1}f \dd x = \sum_{i=1}^{N} \alpha_i \left| I_i \right| 
		\]
		Regelfunktion $ \hat{=}  $ es gibt eine Folge von Treppenfunktionen, die gegen die Funktionen bezüglich der \textbf{Supremumsnorm} konvergiert.($ \left\| f \right\| _{\infty;[0, 1]} \coloneqq \sup_{x \in [0, 1]} \left| f(x) \right|  $)\\
		Da die Sup-Norm die absolute ``Größe'' von Funktionen kontrolliert, ist dieser Integralbegriff \textbf{zu restriktiv}
\end{itemize}

\textbf{Lebesques Ansatz}\\
Sei $ (\Omega, \Sigma, \mu) $ ein Maßraum (Menge, $ \sigma $-Algebra, Maß).
\begin{enumerate}[label=(L\arabic*)]
	\item Ist
		\[
			f = \sum_{i=1}^{N} \alpha_i \I_{A_i} 
		\]
		einfach, so sei
		\[
			\int _{\Omega} f \dd \mu \coloneqq \sum_{i=1}^{N} \alpha_i \mu (A_i) \qquad (A_i \in \mathfrak{A} )
		\]
		Hierbei $ \alpha_1, \dotsc, \alpha_N \geq 0 $.
	\item Ist $ f : \Omega \to \overline{\R } _+ = \R _{\geq 0} \cup \left\{ \infty \right\}  $ messbar (also $ \Sigma \text{-}  \mathfrak{B} (\overline{\R } _+) $-messbar), so existiert eine folge $ (f_i) $ von einfachen Funktionen mit $ f_i \nearrow f $.
	\item Für $ f : \Omega \to \overline{\R } _+ $ messbar sei nun
		\[
			\int _\Omega f \dd \mu = \lim_{i \to \infty} \int _\Omega f_i \dd \mu,
		\]
		wobei $ (f_i) $ eine Folge von einfachen Funktionen mit $ f_i \nearrow f $ ist (existent nach (L2)).
		Diese Def. von $ \int _\Omega f \dd \mu $ hängt \textbf{nicht} von $ (f_i) $ ab.
		$ \I_\Q  $\\
		\textbf{Lebesque:} $ \mathcal{L} ^(\Q ) = 0 $.\\
		\textbf{Grund:} $ x \in \R  $, so $ \mathcal{L} ^1\left( \left\{ x \right\}  \right) = 0 $.
		Sind $ U, V $ messbar, so $ \mathcal{L} ^1(U) \leq \mathcal{L} ^1(V) $ 
		Für $ \varepsilon > 0 $ ist $ \mathcal{L} ^1((x - \varepsilon, x + \varepsilon )) = 2 \varepsilon  \implies \mathcal{L} ^1(\left\{ x \right\} ) \leq \mathcal{L} ^1 ((x - \varepsilon ), (x + \varepsilon )) = 2 \varepsilon \searrow 0 $.
		Damit nach $ \sigma $-Add. 
		\[
			\left( \mu \left( \bigcup_{i = 1} ^\infty A_i \right) = \sum_{i=1}^{\infty} \mu (A_i), \text{ falls  die $ A_i $'s messbar + pw. disjunkt}  \right) :
		\]
		\[
			\Q  = \bigcup_{i = 1} ^\infty \left\{ x_i \right\} \implies \mathcal{L} ^1(\Q ) = \sum_{i=1}^{\infty} \mathcal{L} ^1 ( \left\{ x_i \right\} ) = 0.
		\]
		Damit
		\[
			\int_{\R }^{} \I_{\Q } \dd \mathcal{L} ^1 = 0, \int_{\R }^{} \I_{\R  \setminus \Q } \dd \mathcal{L} ^1 = +\infty
		\]
	\item \textbf{Letztlich:}\\
		Ist $ f : \Omega \to \overline{\R }  $ messbar, so setze $ f^+ \coloneqq \max \left\{ f, 0 \right\}  $, $ f^- \coloneqq \max \left\{ -f, 0 \right\}  $ (womit $ f = f^+ - f^- $).
		Wir definieren, falls entweder
		\[
			\int _\Omega f^+ \dd \mu, \int _\Omega f^{-} \dd \mu, \qquad \int _\Omega f \dd \mu \coloneqq \int _\Omega f^+ \dd \mu - \int_{\Omega}^{} f^- \dd \mu.
		\]
		Messbare Funktionen.
		$ f : \Omega \to \overline{\R }  $ mit $ \int _\Omega f^+ \dd \mu < \infty $ oder $ \int _\Omega f^- \dd \mu < \infty $ heißen \textbf{quasiintegrabel}.
		Ist $ \int_{\Omega}^{} \left| f \right| \dd \mu < \infty $, so heißt $ f $ \textbf{integrabel}.
\end{enumerate}

\subsection{Fundamente der Lebesgueintegral}
\begin{itemize}
	\item Vertauschen von Grenzprozessen.
		Für das \textbf{Regelintegral}: $ [a, b], f_i \to f $ gleichmäßig $ \implies \int_{a}^{b} f_i(x) \dd x \to \int_{a}^{b} f(x) \dd x $
\end{itemize}
\begin{theorem*}[Dominierte Konvergenz, Lebesgue]
	Sei $ (\Omega, \Sigma, \mu) $ Maßraum, sowie $ f, f_1, f_2, \dotsc : \Omega \to \R  $ integrabel und $ f_i \to f $ \textbf{punktweise}.
	Weiter gebe es ein integrables $ F: \Omega \to \R  $ mit $ \forall i \in \N : \left| f_i \right| \leq \left| F \right|  $.
	Dann gilt
	\[
		\int_\Omega \underbrace{f}_{= \lim_{i \to \infty} f_i} \dd \mu = \lim_{i \to \infty} \int _\Omega f_i \dd \mu.
	\]
	\begin{itemize}
		\item $ \Omega = \R  $, $ \Sigma = \mathfrak{B} (\R ) $, $ \mu = \mathcal{L} ^1 $ , $ f_i = \I_{(i, i + 1)}  $, $ f = 0 $, dann $ f_i \to f $ punktweise.\\
			\begin{tikzpicture}
				\begin{axis}[
					xmin= -0, xmax= 4,
					ymin= -0, ymax = 2,
					axis lines = middle,
					xtick = { 1, 2, 3 },
					ytick = { },
					xticklabels = {$ i $, $ i + 1 $, $ i + 2 $},
					yticklabels = {},
					height = 4cm,
					width = 8cm,
				]
					\addplot[domain=1:2, samples=100]{1};
				\end{axis}
			\end{tikzpicture}\\
			$ \forall i \in \N : \int _\Omega f_i \dd \mu = \int _{\R } \I_{i, i + 1} \dd \mathcal{L} ^1 = 1  $. \& $ \int _\Omega f \dd \mu = 0 $. $ (1 \nrightarrow 0) $
	\end{itemize}
	Eine integrierbare Majorante $ F $ muss notwendigerweise für Große $ x $ einmal ``unter'' $ f_i $ liegen. Unmöglich.
\end{theorem*}

\begin{itemize}
	\item $ \Omega = (0, 1), \Sigma = \mathfrak{B}((0, 1)), \mu = \mathcal{L}^1  $. $ f_i \coloneqq i \I_{\left(0, \frac{ 1 }{ i } \right)}  $.
		$ \forall i \in \N : \int _\Omega f_i \dd \mu = 1 $.
		$ f_i \to 0 $ punktweise überall, und $ \int _\Omega f \dd \mu = 0 $. ($ 1 \nrightarrow 0 $)
		\[
			\underbrace{\int _{\R } f \dd \mathcal{L} ^1}_{\text{``Masse''}}  = 0 \neq \lim_{i \to \infty} \underbrace{\int _{\R } f_i \dd \mathcal{L} ^1}_{\text{``Masse''} } \quad \text{Masse geht verloren} 
		\]
	\item Kann hierbei auch im Limes dazukommen? \textbf{Nein}, wegen \textbf{Lemma von Fatou}.
\end{itemize}
	
\begin{theorem}[Monotone Konvergenz, Beppo Levi]
	Sei $ \left( \Omega, \Sigma, \mu \right)  $ Maßraum, $ f_1, f_2, f_3, \dotsc : \Omega \to \R  $ messbar mit $ f_i \nearrow f $ für ein $ f : \Omega \to \R  $ messbar.
	So gilt
	\[
		\int _\Omega f \dd \mu = \lim_{i \to \infty} \int _\Omega f_i \dd \mu.
	\]
	{$ [ f_i \nearrow f : \forall i \in \N : \forall x \in \Omega : f_i (x) \leq f_{i + 1} (x) \& \forall x \in \Omega : f(x) \lim_{i \to \infty} f_i(x)] $}
	\begin{itemize}
		\item Keine integrierbare Majorante gefordert! Speziell darf $ \int _\Omega f \dd \mu = \infty $ passieren.
		\item Stattdessen bei \textbf{punktweiser Konvergenz}:
			\[
				\left| f \right| \leq \lim_{i \to \infty} \left| f_i \right| \leq \left| F \right| \& F \text{ integrierbar} ,
			\]
			also auch $ f $ integrierbar, also $ \int_{\Omega}^{} \left| f \right| \dd \mu < \infty $
	\end{itemize}
\end{theorem}

\begin{corollary}[Lemma von Fatou]
	Sei $ (\Omega, \Sigma, \mu) $ ein Maßraum \& $ f, f_1, \dotsc : \Omega \to \R  $ messbar mit $ f = \lim_{i \to \infty} f_i $ punktweise.
	Angenommen, $ \forall i \in \N  : f_i \geq 0 $.
	Dann
	\[
		\underbrace{ \int_{\Omega}^{} f \dd \mu }_{ \int _\Omega \liminf_{i \to \infty} f_i \dd \mu } \leq \liminf_{i \to \infty} \int _\Omega f_i \dd \mu
	\]
\end{corollary}
\begin{proof*}[Corollary \ref{7.9}]
	\[
		g_k \coloneqq \inf_{l \geq k} f_l \nearrow \liminf_{i \to \infty} f_i.
	\]
	Dann
	\[
		\int_{\Omega}^{}g_k \dd \mu \overset{k \to \infty}{\longrightarrow} \int \underbrace{\liminf_{i \to \infty} f_i}_{f} \dd \mu
	\]
	\[
		\int _\Omega f \dd \mu \leq \lim_{k \to \infty} \int _\Omega \underbrace{\inf_{l \geq k} f_j}_{f_k} \dd \mu = \lim_{k \to  \infty} \int_{\Omega}^{} f_k \dd \mu.\qed
	\]
\end{proof*}

\begin{itemize}
	\item 
		\[
			f_i = \I_{[i, i + 1]} \geq 0, \quad \int _\Omega f \dd \mu = 0 < 1 = \liminf_{i \to \infty} \underbrace{\int _\Omega f_i \dd \mu}_{1}.
		\]
	\item 
		\[
			f_i = i \I_{\left(0, \frac{ 1 }{ i } \right)} , \Omega = \R , \Sigma = \mathfrak{B} (\R ), \mu = \mathcal{L} ^1.
		\]
		$ f_i \to 0 $ punktweise
		\[
			\int _\Omega f \dd \mu = 0 < 1 = \liminf_{i \to \infty} \int _\Omega f_i \dd \mu
		\]
\end{itemize}

\begin{note*}
	Wir können \textbf{nicht} ohne Weiteres auf $ f_i \geq 0 $ verzichten.
	$ f_i \coloneqq - \frac{ 1 }{ i } \I_{(0, i)}  $, so $ f_i \to 0 $ (gleichmäßig), also $ f \equiv 0 $.
	\[
		\int _\Omega f \dd \mu = 0 > -1 = \liminf_{i \to \infty} \int _\Omega f_i \dd \mu.
	\]
	(Hierbei $ \Omega = \R , \Sigma = \mathfrak{B} (\R ), \mu = \mathcal{L} ^1 $)

	Ausweg: Man nimmt an, dass $ \exists H : \Omega \to \R _{\geq 0}  $ integrierbar mit $ \forall i \in \N : f_i \geq - H $.
		Denn dann $ 0 \leq H + f_i \overset{i \to \infty}{\longrightarrow} H + f $ punktweise, also nach
		\[
			\int _\Omega H + f \dd \mu \leq  \liminf_{i \to \infty} \int _\Omega H + f_i.
		\]
		\[
			\underbrace{\int _\Omega H \dd \mu}_{\in \R } + \int _\Omega f \dd \mu \leq \underbrace{\int _\Omega H \dd \mu}_{\in \R } + \liminf_{i \to \infty} \int _\Omega f_i \dd \mu.
		\]
		\[
			\iff \int _\Omega f \dd \mu \leq  \liminf_{i \to \infty} \int _\Omega f_i \dd \mu.
		\]
\end{note*}

\begin{itemize}
	\item $ \R ^2 = \R \times \R  $; Lebesguemaß auf $ \R ^2 $: Eindeutig bestimmtes Maß auf $ \mathfrak{B} (\R ): \mathcal{L} ^2 \left( \left( a, b \right) \times \left( c, d \right)  \right) = \left( d - c \right) \cdot (b - a) $. $ (\forall a, b, c, d) $
		\[
			\int_{\R ^2}^{} f(x, y) \dd \mathcal{L} ^2 (x, y) \overset{?}{=} \int_{\R }^{} \int _\R f(x, y) \dd \mathcal{L} ^1(x) \dd \mathcal{L} ^1(y).
		\]
		{[Guido Fubini]}
	\item Ein Maßraum heißt \textbf{$ \sigma $-endlich}, falls 
		\[
			\exists V_1, V_2, \dotsc \in \Sigma : V_1 \subset V_2 \subset \dotsb, \Omega = \bigcup_{k \in \N } V_k \; \& \; \forall k \in \N : \mu (V_k) < \infty.
		\]
		Dies ist z.B. erfüllt für alle endlichen Maße, aber auch:
		$ \R ^k $ mit $ \Sigma = \mathfrak{B} (\R ^n), \mu = \mathcal{L} ^n. $ 
		\[
			\R ^n = \bigcup_{k \in \N } [-k, k]^n \; \& \; \forall k \in \N : \mathcal{L} ^n \left( [-k, k]^n \right) = (2k)^n < \infty.
		\]
	\item Für zwei Maßräume $ \left( \Omega_1, \Sigma_1, \mu_1 \right) , \left( \Omega_2, \Sigma_2, \mu_2 \right)  $ sei $ \Sigma_1 \otimes \Sigma_2 $ die von allen Mengen der Form $ A \times B, A \in \Sigma_1, B \in \Sigma_2 $ erzeugt $ \sigma $-Algebra auf $ \Omega_1 \times \Omega_2 $.
		Wir nennen $ \mu : \Sigma_1 \otimes \Sigma_2 \to [0, \infty] $ ein Produktmaß bezüglich $ \mu_1 \& \mu_2 $, falls
		\[
			\forall A \in \Sigma_1: \forall B \in \Sigma_2 : \mu (A \times B) = \mu_1(A) \mu(B)
		\]
		Sind $ (\Omega_1, \Sigma_1, \mu_1), \left( \Omega_2, \Sigma_1, \mu_2 \right)  $ $ \sigma $-endlich, so ist ein Produktmaß eindeutig durch $ \mu_1, \mu_2 $ festgelegt.
\end{itemize}

\begin{theorem}[Fubini]
	Seien $ \left( \Omega_1, \Sigma_1, \mu_1 \right) , \left( \Omega_2, \Sigma_2, \mu_2 \right)  $ zwei $ \sigma $-endliche Maßräume \& sei $ f: \Omega_1 \times \Omega_2 \to \R  $ $ \Sigma_1 \otimes \Sigma_2 $-messbar \& auf $ \Omega_1 \times \Omega_2 $ integrierbar,
	\[
		\int _{\Omega_1 \times \Omega_2} \left| f \right| \dd \mu < \infty,
	\]
	wobei $ \mu \coloneqq \mu_1 \otimes \mu_2 $ \textbf{das} Produktmaß von $ \mu_1, \mu_2 $ ist.
	Dann ist für $ \mu_1 $-fast alle $ x \in \Omega_1 $ die Funktion $ \Omega_2 \ni y \mapsto f(x, y) \in \R  $ $ \Sigma_2 $-messbar, für $ \mu_2 $-fast alle $ y \in \Omega_2 $ die Funktion $ \Omega_1 \ni x \mapsto f(x, y) \in \R  $ $ \Sigma_1 $-messbar, und es gilt:
	\[
		\int _{\Omega_1 \times  \Omega_2} f(x, y) \dd \left( \mu_1 \otimes \mu_2 \right) \left( x, y \right) = \int _{\Omega_1} \int _{\Omega_2} f(x, y) \dd \mu_2(y) \dd \mu_1(x) = \int _{\Omega_2} \int _{\Omega_1} f(x, y) \dd\mu_1(x) \dd \mu_2 (y).
	\]
\end{theorem}

\begin{theorem}[Transformationssatz ($ \hat{=} $ Substitutionsregel aus Ana. I)]
	Sei $ \Omega \subset \R ^n $ offen, $ \Phi: \Omega \to \Phi(\Omega) $ ein $ C^1 $-\textbf{Diffeomorphismus} ($ \Phi: \Omega \to \Phi(\Omega) $ ist $ C^1 $, $ \Phi $ ist bijektiv \& $ \Phi^{-1} : \Phi(\Omega) \to \Omega $ ist $ C^1 $).
	Dann ist eine Funktion $ f : \Phi(\Omega) \to \R  $ bezüglich $ \mathcal{L} ^n $ integrierbar genau dann, wenn $ \Omega \ni x \mapsto f (\Phi(x)) \left| \det D \Phi(x) \right|  $ integrierbar bezüglich $ \mathcal{L} ^1 $ ist.
	In diesem Fall gilt:
	\[
		\int _{\Phi(\Omega)}  f \dd \mathcal{L} ^n = \int _{\Omega} f (\Phi) \left| \det D \Phi \right| \dd \mathcal{L} ^n
	\]
\end{theorem}

\begin{example}
	\[
		\int_{-\infty}^{\infty} e^{- \frac{ x^2 }{ 2 } } \dd x = \sqrt{2 \pi } .
	\]
	\begin{itemize}
		\item 
			\begin{align*}
				\left( \int _\R e^{-x^2}  \right) ^2 &= \int_{\R }^{} e^{- \frac{ x^2 }{ 2 } } \dd \mathcal{L} ^1(x) \int_{\R }^{} e^{- \frac{ y^2 }{ 2 } } \dd \mathcal{L} ^1 (y) \\
				~&\overset{\text{Fubini} }{=} \int_{\R } \int _{\R } e^{\frac{ - x^2 - y^2 }{ 2 } } \dd \mathcal{L} ^1(x) \dd \mathcal{L} ^1(y) \\
				~&\overset{z = (a, b)}{=} \int _{\R ^2} e^{- \frac{\left| z \right| ^2}{ 2 } } \dd \mathcal{L} ^2(z)
			\end{align*}
	\end{itemize}
	
\end{example}


