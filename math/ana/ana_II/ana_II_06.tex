\section{Der lokale Umkehrsatz \& implizite Funktionen}
\begin{itemize}
	\item $ f : \R ^n \to \R ^n $ differenzierbar.
		Was müssen an $ x_0 \in \R ^n $ fordern, so dass es eine offene Menge $ V $ mit $ x_0 \in V $ \&
		\[
			\underbrace{f : V}_{f|_V} \to f(V)
		\]
		ist bijektiv mit differenzierbarer Umkehrfunktion?
\end{itemize}
\begin{itemize}
	\item Falls $ y \in f(V) $, löse dann $ \exists ! x \in V : f(x) = y $.
\end{itemize}
\textbf{Idee:}
\[
	f(x) = \underbrace{f(x_0) + Df(x_0) (x - x_0)}_{\text{linearisertes $ f $} } + \left( \underbrace{\text{Rest} }_{\text{vernachlässigbar} } \right) 
\]
\textbf{lokal:}
\[
	f(x) \hat{=} f(x_0) + Df(x_0) (x - x_0) = y
\]
Wollen wir dies lösen, also
\[
	Df(x_0) (x - x_0) = y - f(x_0),
\]
so muss $ Df(x_0) $ invertierbar sein.
Frage: Auch hinreichend?

\begin{lemma}
	Sei $ \operatorname{GL}(n) \coloneqq \left\{ A \in \R ^{n \times n} : A \text{ invertierbar}  \right\}  $ ist offen in $ \R ^{n \times n}  $ bezüglich jeder Norm, und inv: $ \operatorname{GL}(n) \ni A \mapsto A^{-1} \in \operatorname{GL}(n) $ ist stetig differenzierbar.
\end{lemma}
\begin{proof*}[Lem \ref{6.1}]
	\[
		A = \left( a_{ij}  \right) _{i, j = 1, \dotsc, n} \mapsto \det (A)
	\]
	ist ein Polynom, also stetig.
	Damit $ \operatorname{GL}(n) = \det ^{-1} \underbrace{\left( \R \setminus \left\{ 0 \right\}  \right) }_{\text{offen} } $ offen.
	Differenzierbarkeit folt aus der \textsc{Cramer}schen Regel.
	Nämlich
	\[
		A^{-1} = \frac{ 1 }{ \det(A) } \operatorname{cof}(A)^{T} 
	\]
	\[
		\operatorname{cof}(A) = m_{ij} = (-1)^{i + j} M_{ij} 
	\]
	mit $ M_{ij}  $, der $ (i, j) $-te Minore ($ \det $ der Matrix, ohne $ i $-te Zeile, $ j $-te Spalte).
	(siehe Lin. Alg.)
	Damitist $ A \mapsto \frac{ 1 }{ \det (A) } \operatorname{cof}(A) $ ist eine $ \R ^{n \times n}  $ - wertige Abbildung, deren einzelne Komponeneten rationale Funktionen in den Einträgen von $ A $ sind. $ \to  $ \textbf{Beh.}
\end{proof*}

\begin{theorem}
	Sei $ \Omega_1 \subset \R ^n $ offen, $ \Omega_2 \subset \R ^m $ offen, $ f : \Omega_1 \to \Omega_2 $ bijektiv.
	Sei $ x_0 \in \Omega_1 $ \& $ f $ in $ x_0 $ differenzierbar, $ f^{-1}  $ in $ f(x_0) $ differenzierbar.
	Dann $ m = n $; $ Df(x_0) $ inv \& $ \left( Df^{-1}  \right) \left( f(x_0) \right) = \left( Df(x_0) \right) ^{-1}  $ 
	\[
		f^{-1} (f(x)) = x,
		f(f^{-1} (x)) = x 
		\implies \left( Df^{-1} \right) (x) \left( Df \right) \left( f^{-1} (x) \right) = 1
	\]
\end{theorem}
\begin{proof*}[Thm. \ref{6.2}]
	\[
		A \coloneqq Df(x_0), B \coloneqq \left( Df^{-1}  \right) \left( f(x_0) \right) .
	\]
	\[
		f \circ f^{-1} = \id_{\Omega_2} \& f^{-1} \circ f = \id_{\Omega_1} \overset{\text{Kettenregel} }{\implies } BA = E_n, AB = E_m
	\]
	Mit $ E_n $ ist die $ (n \times n ) $-Einheitsmatrix,
	und $ E_m $ ist die $ (m \times m ) $-Einheitsmatrix.\qed
\end{proof*}

\textbf{Frage:} Wann ist $ f^{-1}  $ differenzierbar?

\begin{lemma}
	Seien $ \Omega_1, \Omega_2 \subset \R ^n $ offen, $ f : \Omega_1 \to \Omega_2 $ bijektiv \& stetig differenzierbar.
	Weiter sei $ f^{-1}  $ stetig \& $ \forall x \in \Omega_1 : Df(x) \in \GL(n) $.
	Dann ist $ f^{-1}  $ \textbf{stetig differenzierbar}.
\end{lemma}
\begin{proof*}[Lem \ref{6.3}]
	Zu zeigen: $ f^{-1}  $ ist in jedem $ b = f(x_0), x_0 \in \Omega_1 $ differenzierbar
	\begin{description}
		\item[1. Reduktion] $ x_0 = 0, f(x_0) = 0. $ 
			Ansonsten betrachte $ g(x) = f(x_0 + x) - f(x_0) $
		\item[2. Reduktion] Sei $ C \coloneqq \left( Df(0) \right) ^{-1}  $. Wir zeigen die Aussage für $ \left( Cf \right) ^{-1}  $,
			denn $ D \left( \left( C f \right)^{-1}   \right) = C^{-1}  \left( Df^{-1}  \right)  $ (+ $ C $ invertierbar!)
		\item[3. Reduktion] Damit \OE{} $ Df(0) = E_n $.
	\end{description}
	\begin{itemize}
		\item $ f $ in $ 0 $ differenzierbar $ \implies  \exists  r : \Omega_1 \overset{\text{stetig} }{\to }\R ^n : r(0) = 0 $ 
			\& $ \forall x \in \Omega_1 : f(x) \overset{\text{\textbf{3. Red.}} }{=} f(0) + E_n x + r(x) \left\| x \right\| _2 $ 
		\item Setze
			\[
				s(y) \coloneqq
				\begin{cases}
					-r(x) \frac{ \left\| x \right\| _2 }{ \left\| y \right\| _2 } , & y \neq 0, y = f(x)\\
					0, & \text{sonst.} 
				\end{cases}
				{ \color{gadse-dark-green}
					\implies -r(x) \left\| x \right\| _2 = s(y) \left\| y \right\| _2
				}
			\]
			$ f(0) = 0 $. Also $ f^{-1} (0) = 0 $.
			Für alle $ y = f(x), y \neq 0 $.
			\begin{align*}
				f^{-1} (y) = x &= f(x) - r(x) \left\| x \right\| _2\\
				~ &= y + s(y) \left\| y \right\| _2 \\
				~ &= \underbrace{f^{-1} (0)}_{0} + E_n y + s(y) \left\| y \right\| _2
			\end{align*}
			Dies gilt auch für $ y = 0 $.
			Also ist $ f^{-1}  $ in $ 0 $ differenzierbar.
			Also auch für alle $ x_0 \in \Omega_1 $.
	\end{itemize}
	---\\
	Zunächst gilt
	\[
		f^{-1} \circ f = id_{\Omega_1} ,
	\]
	damit folgt nach Kettenregel
	\[
		\left( Df^{-1}  \right) \left( f(x) \right) \left( Df(x) \right) = E_n,
	\]
	also
	\[
		\left( Df^{-1}  \right) \left( f(x) \right) = \left( Df(x) \right) ^{-1} 
	\]
	mit $ y = f(x) $ folgt, dass
	\[
		Df^{-1} = \underbrace{\operatorname{inv}}_{\text{Lem. \ref{6.1} stetig}}  \circ \underbrace{Df}_{\text{stetig nach Vor.} }  \circ \underbrace{f^{-1}}_{\text{stetig nach I teil des Bew.}}
	\]
	Also $ Df^{-1}  $ stetig\qed
\end{proof*}

\begin{definition}
	Seien $ \Omega_1, \Omega_2 \subset \R ^n $ offen.
	Eine stetig differenzierbare, bijektive Abbildung $ f: \Omega_1 \to \Omega_2 $ heißt \textbf{Diffeomorphismus}, falls 
	\[
		f^{-1} : \Omega_2 \to \Omega_1
	\]
	ebenfalls stetig differenzierbar ist.
\end{definition}

\begin{theorem}[Lokaler Umkehrsatz]
	Sei $ \Omega \subset \R ^n $ offen und $ f: \Omega \to \R ^n $ stetig differenzierbar.
	Sei $ x_0 \in \Omega $ so, dass $ Df(x_0) $ invertierbar ist.
	Dann gibt es eine offene Umgebung $ U $ von $ x_0 $, sodass $ f(U) $ eine offene Umgebung von $ f(x_0) $ ist und $ f|_U : U \to f(U) $ ein Diffeomorphismus ist: $ f|_U $ ist stetig differenzierbar und $ \left( f|_{U} \right)^{-1}  $ ist stetig differenzierbar.
\end{theorem}
\begin{proof*}[Theorem \ref{6.5}]
	\OE{} Sei \OE{} $ x_0 = 0 $ und $ f(x_0) = 0 $, ansonsten betrachte $ x \mapsto f\left( x + x_0 \right) - f(x_0) $.
	Betrachte mit $ C \coloneqq \left( Df(0) \right) ^{-1}  $ den Isomorphismus $ T : x \mapsto Cx $.
	Nach Lemma \ref{5.3} gilt
	\[
		D\left( T \circ f \right) \left( 0 \right) = C \cdot Df(0) = E_n.
	\]
	Also können wir annehmen, dass $ x_0 = f(x_0) = 0 $ und $ Df(0) = E_n $
	Wir benutzen:
	\begin{itemize}
		\item den Schrankensatz
		\item den Banachschen Fixpunktsatz
	\end{itemize}
	\begin{description}
		\item[Schritt 1] Wir wollen
			\[
				f(x) = y
			\]
			lösen.
			Setzen für $ y \in \R ^n $ und $ x \in \Omega $ 
			definiere
			\[
				g_y (x) \coloneqq y + x - f(x).
			\]
			Wir wollen
			\[
				g_y(x) = x
			\]
			lösen.
			Nach Voraussetzung ist $ Df $ stetig in $ x_0 = 0 $. Daher gibt es ein $ r> 0 $, sodss
			\[
				K \coloneqq \overline{B_{2r}}(0) \subset \Omega,
			\]
			sowie
			\[
				\forall x \in K : \left\| E_n - Df(x) \right\| < \frac{ 1 }{ 2 } 
			\]
			Da $ Dg_y(x) = E_n Df(x) $, folgt mit Schrankensatz
			\[
				\forall x_1, x_2 \in K : \left\| g_y(x_1) - g_y(x_2) \right\| \leq \max_{x \in K} \left\| E_n - Df(x) \right\| \left\| x_1 - x_2 \right\| _2 \leq  \frac{ 1 }{ 2 } \left\| x_1 - x_2 \right\| _2
			\]
			Daraus folgt, dass $ \forall x \in K, y \in B_{r}(0) : \left\| g_y (x) \right\| _2 \leq \left\| g_y(x) - g_y(0) \right\| _2 + \left\| g_y(0) \right\| _2 \leq \frac{ 1 }{ 2 } \left\| x \right\| _2 + \left\| y \right\| _2 < r + r = 2r $.\\
			$ \implies g_y : K \to K $ ist kontrahierende selbstabbildung.
			Da $ K $ abgeschlossen und beschränkt ist, ist $ K $ kompakt, als BFS anwendbar.
			\begin{itemize}
				\item Ziel erreicht: Für jedes $ y \in B_{r}(0)  $ gibt es genau ein $ x \in K $ so, dass $ g_y(x) = x $, also $ f(x) = y $ mit $ B_{2r}(0)  $.
			\end{itemize}
			Für $ y \in B_{r}(0)  $ definieren wir $ h(y) $ also das eindeutig bestimmte $ x \in B_{2r}(0)  $, also
			\[
				h(y) \coloneqq  x,
			\]
			\[
				V \coloneqq B_{r}(0) 
			\]
			und 
			\[
				U \coloneqq f^{-1} (V) \cap B_{2r}(0) 
			\]
			Dies definiert eine Umkehrabbildung
			\[
				h : V \to U
			\]
			zu $ f|_U : U \to f(U) $
		\item[Schritt 2]
			\textbf{$ h $ ist stetig}\\
			Seien $ y_1, y_2 \in V $.
			Definiere $ x_1 = h(y_1) $ und $ x_2 \coloneqq h(y_2) $.
			Da $ g_y(x) = x $ folgt $ x_1 - x_2 = g_0(x_1) - g_0(x_2) + f(x_1) - f(x_2) $.
			D.h.
			\[
				\left\| x_1 - x_2 \right\| _2 \leq \frac{ 1 }{ 2 } \left\| x_1 - x_2 \right\| _2 + \left\| f(x_1) - f(x_2) \right\| _2,
			\]
			also
			\[
				\left\| h(y_1) - h(y_2) \right\| _2 \leq 2 \left\| y_1 - y_2 \right\| _2.
			\]
			Also $ h $ stetig.
		\item[Schritt 3]
				Invertierbarkeit von $ Df(x) $ auf $ U $.
				Sei $ x \in U $, dann gilt $ \left\| x \right\| _2 < 2r $.
				Dann gilt
				\[
					\forall v \in \R ^n : \left\| v - Df(x) v \right\| _2 \leq  \frac{ 1 }{ 2 } \left\| v \right\| _2
				\]
				Angenommen $ v \in \Kern \left( Df(x) \right)  $.
				Dann $ \left\| v \right\| _2 \leq \frac{ 1 }{ 2 } \left\| v \right\| _2 $, also $ v = 0 $.
				Also $ Df(x) $ invertierbar.
				Behauptung folgt aus Lemma \ref{6.3} \qed
	\end{description}
\end{proof*}

\begin{corollary}[Satz von der Offenheit]
	Sei $ \Omega \subset \R ^n $ offen und $ f: \Omega \to \R ^n $ eine stetig differenzierbare Funktion, sodass $ Df(x) $ für alle $ x \in \Omega $ invertierbar ist.
	Dann ist $ f(\Omega) $ offen.
\end{corollary}
\begin{proof*}[Korollar \ref{6.6}]
	Nach lokalem Umkehrsatz gibt es für jedes $ x \in \Omega $ eine offene Umgebung $ U_x $, sodass $ f\left( U_x \right)  $ offen ist.
	D.h.  
	\[
		f(\Omega) = \bigcup_{x \in \Omega} f\left( U_x \right) 
	\]
	ist offen.\qed
\end{proof*}

\begin{example}
	\[
		f(x_1, x_2) = \left(x_1x_2^2, x_1^{x_2} \right)
	\]
	auf
	\[
		\Omega = \left( 0, \infty \right) \times \R .
	\]
	Wir beachten, dass $ x_1^{x_2} = \exp \left( x_2 \log x_1 \right)  $.
	Damit gilt
	\[
		Df(x) =
		\begin{pmatrix} 
			x_2^2 & 2x_1x_2 \\
			x_2x_1^{x_2 - 1} & \log \left( x_1 \right) x_1^{x_2}
		\end{pmatrix} ,
	\]
	im Punkt $ (1, 2) $ gibt
	\[
		Df(1, 2) = \begin{pmatrix} 4 & 4 \\ 2 & 0 \end{pmatrix} 
	\]
	ist invertierbar.
	$ \implies  $ Lokal umkehrbar.
\end{example}

\begin{corollary}
	Seien $ U \subseteq \R ^n $ offen und
	\[
		f : U \to \R ^n
	\]
	eine injektiv, stetig differenzierbar mit $ Df(x) $ invertierbar für alle $ x \in U $.
	Dann ist $ f(U) $ offen und
	\[
		f^{-1} : f(U) \to \R ^n
	\]
	stetig differenzierbar.
\end{corollary}
\begin{proof*}[Korollar ref{6.8}]
	Nach Korollar \ref{6.6}: $ f(U) $ offen.
	Sei nun $ b \in f(U) $ und sei $ a = f^{-1} (b) $.
	Nach Satz \ref{6.5} existiert eine offene Umgebung $ W \subseteq f(U) $ von b, sodass $ f_{|W} ^{-1}  $ stetig differenzierbar ist.\qed
\end{proof*}

\begin{example}
	Sei $ f: \left( 0, \infty \right) \times \left( 0, 2 \pi  \right) \to \R ^2 $ definiert als
	\[
		f(r, \varphi) = \left( r \cos \varphi, r \sin \varphi \right) 
	\]
	$ f $ ist injektiv und stetig differenzierbar.
	Es ist
	\[
		Df\left( r, \varphi \right) =
		\begin{pmatrix} 
			\cos \varphi & - r \sin \varphi \\
			\sin \varphi &   r \cos \varphi
		\end{pmatrix} 
	\]
	Also $ \det \left( Df\left( r, \varphi \right)  \right) = r \cos ^2 \varphi + r \sin ^2 \varphi = r > 0 $.
	Also ist $ Df\left( r, \varphi \right)  $ invertierbar und somit $ g = f^{-1}  $ stetig differenzierbar auf $ \R \setminus \left\{ 0 \right\}  $.
	Für $ (x, y) = f(r, \varphi) $ gilt
	\[
		Dg(x, y) = \left( Df(r, \varphi) \right) ^{-1} =
		\begin{pmatrix} 
			\cos \varphi & \sin \varphi \\
			- \frac{ 1 }{ r } \sin \varphi & \frac{ 1 }{ r } \cos \varphi
		\end{pmatrix} 
		=
		\begin{pmatrix} 
			\frac{ x }{ \left\| (x, y) \right\|  } & \frac{ y }{ \left\| (x, y) \right\|  } \\
			- \frac{ y }{ \left\| (x, y) \right\| ^2 } & \frac{ x }{ \left\| (x, y) \right\| ^2 } 
		\end{pmatrix} 
	\]
\end{example}

Sei nun $ F : \R ^2 \to \R  $ mit $ F\left( a, b \right) = 0 $.
Wir suchen für die eindeutige Lösbarkeit der Gleichung
\[
	F(x, y) = 0
\]
Werte $ y $ in Abhängigkeit von $ x $ in einer Umgebung von $ (a, b) $

\begin{example}
	Für
	\[
		F\left( x, y) \right) = x - y^2.
	\]
	Es ist $ F(1, 1) = 0 $ 
	Definiere
	\[
		g : \left( 0, \infty \right) \to \R 
	\]
	durch
	\[
		g(x) = \sqrt{x} 
	\]
	Dann gilt $ f(x, g(x)) = 0 $ in einer Umgebung von $ (1, 1) $
\end{example}

\begin{definition*}[Notation]
	Für  
	\[
		g : X \to Y
	\]
	sei $ \Graph(g) = \left\{ \left( x, g(x) \right) : x \in X \right\}  $ der Graph von $ g $
\end{definition*}

\begin{note*}
	Seien $ \left( u, v \right) \in \R ^k \times \R ^m $ und $ W $ Umgebung von $ (u, v) $.
	Dann existieren Umgebungen $ W_0 $ von $ u $ und $ W_1 $ von $ v $ mit $ W_0 \times W_1 \subseteq W $.
	\begin{proof*}
		Für $ x = (x_1, \dotsc, x_k) \in \R ^k $ und $ y = \left( y_1, \dotsc, y_m \right) \in \R ^m $ ist
		\[
			\left\| (x, y) \right\| _2^2 = \left\| x \right\| _2^2 + \left\| y \right\| _2^2 \leq \left( \left\| x \right\| _2 + \left\| y \right\| _2 \right) ^2,
		\]
		also 
		\[
			\left\| (x, y) \right\| _2 \leq \left\| x_2 \right\| + \left\| y \right\| _2
		\]
		Wähle also $ \varepsilon > 0 $ mit $ B_{\varepsilon}(\left( u, v \right) ) \subseteq W $.
		Dann ist
		\[
			B_{\frac{ \varepsilon }{ 2 } }(u) \times B_{\frac{ \varepsilon }{ 2 } }(v) \subseteq B_{\varepsilon}((u, v)) \subseteq W.\qed
		\]
	\end{proof*}
\end{note*}

\begin{definition*}[Notation]
	Seien
	\[
		U_1 \subseteq \R ^k \text{ und} 
	\]
	\[
		U_2 \subseteq \R ^m
	\]
	offen und 
	\[
		F : U_1 \times U_2 \to \R ^m
	\]
	differenzierbar.
	Für $ (a, b) \in U_1 \times U_2 $ setze
	\[
		\frac{ \partial F }{ \partial x } (a, b)
		=
		\begin{pmatrix} 
			D_1 F_1(a, b) & \hdots & D_k F_1(a, b) \\
			\vdots & \ddots & \vdots \\
			D_1 F_m (a, b) & \hdots & D_k F_m(a, b)
		\end{pmatrix} 
		\text{ und} 
	\]
	\[
		\frac{ \partial F }{ \partial y } (a, b)
		=
		\begin{pmatrix} 
			D_{k + 1}  F_1(a, b) & \hdots & D_{k + m}  F_1(a, b) \\
			\vdots & \ddots & \vdots \\
			D_{k + 1}  F_m (a, b) & \hdots & D_{k + m}  F_m(a, b)
		\end{pmatrix} 
		.
	\]
	Es gilt also
	\[
		DF(a, b) = 
		\left( 
			\begin{array}{c|c}
				\frac{ \partial F }{ \partial x } (a, b) & \frac{ \partial F }{ \partial y } (a, b)
			\end{array}
		\right) 
	\]
	Somit ist für $ \left( u, v \right) \in \R ^k \times \R ^m $ 
	\[
		DF(a, b) (u, v) = \frac{ \partial F }{ \partial x } (a, b) \cdot u + \frac{ \partial F }{ \partial y } (a, b) \cdot v.
	\]
\end{definition*}

\begin{theorem}[Implizite Funktionen]
	Seien
	$
		U_1 \subseteq \R ^k
	$
	und
	$
		U_2 \subseteq \R ^m
	$
	offen und sei
	\[
		F : U_1 \times U_2 \to \R ^m
	\]
	stetig differenzierbar.
	Weiter sei
	$
		(a, b) \in U_1 \times U_2
	$
	mit
	\[
		F(a, b) = 0
	\]
	und
	\[
		\frac{ \partial F }{ \partial y } (a, b)
	\]
	sei invertierbar.
	Dann gibt es offene Umgebungen $ V_1 \subseteq U_1 $ von $ a $ und $ V_2 \subseteq U_2 $ von $ b $, sowie eine stetig differenzierbare Funktion
	\[
		g : V_1 \to V_2
	\]
	mit
	\[
		\Graph(g) = \left\{ (x, y) \in V_1 \times V_2 : F(x, y) = 0 \right\} \quad \left( \text{d.h. } F(x, g(x)) = 0 \right) 
	\]
	Weiterhin gilt
	\[
		Dg(a) =
		- \left(
			\frac{ \partial F }{ \partial y } (a, b)
		\right)^{-1} 
		\frac{ \partial F }{ \partial x } (a, b)
	\]
\end{theorem}
\begin{proof*}[Satz \ref{6.11}]
	Definiere
	\[
		H : U_1 \times U_2 \to \R ^{k + m} 
	\]
	durch
	\[
		H(x, y) = \left( x, F(x, y) \right) .
	\]
	$ H $ ist stetig differenzierbar und es gilt
	\[
		DH(a, b) =
		\begin{pmatrix} 
			E_k & 0 \\
			\frac{ \partial F }{ \partial x  } (a, b) & \frac{ \partial F }{ \partial y } (a, b)
		\end{pmatrix} .
	\]
	Es ist
	\[
		\det \left( DH(a, b) \right) = \underbrace{\det \left( \frac{\partial F}{ \partial y } (a, b) \right) }_{\neq 0 \text{ nach Annahme} } \cdot  \det(E_k) \neq 0
	\]
	Also ist $ DH(a, b) $ invertierbar $ \implies  $ Lokaler Umkehrsatz:
	\begin{itemize}
		\item Es gibt eine offene Umgebung $ V \subseteq  U_1 \times  U_2 $ von $ (a, b) $ und $ W \subseteq \R ^{k + m}  $ von $ H(a, b) $ derart, dass $ h \coloneqq H_{|V}  $ eine Bijektion von $ V $ nach $ W $ ist, für die $ h^{-1}  $ stetig differenzierbar ist.
	\end{itemize}
	Sei 
	\[
		h^{*} : W \to \R ^m
	\]
	die Funktion, die
	\[
	h^{-1}(u, v) = \left( u, h^{*} (u, v) \right)
	\]
	für alle $ (u, v) \in W $ erfüllt.
	Dann
	\begin{equation}
		\label{eq:6.11.1}
		\tag{$ * $}
		\forall (x, y) \in V : \left( F(x, y) = 0 \iff H(x, y) = (x, 0) \iff y = h^{*} (x, 0) \right) 
	\end{equation}
	Nach Bemerkung existieren offene Umgebungen $ W_1 $ von $ a $ und $ V_2 $ von $ b $ mit $ W_1 \times V_2 \subseteq V $.
	Nun ist $ h^{*} (a, 0) = b $ und $ x \mapsto h^{*} (x, 0) $ ist stetig.
	Also existiert eine offene Umgebung $ V_1 \subseteq W_1 $ von $ a $ mit $ h^{*} (x, 0) \in V_2 $ für alle $ x \in V_1 $.
	Definiere
	\[
		g : V_1 \to V_2 \quad \text{durch} \quad g(x) = h^{*} (x, 0).
	\]
	$ h^{-1}  $ ist stetig differenzierbar, also ist es auch $ g $ und es gilt
	\[
		\Graph(g) = \left\{ (x, y) \in V_1 \times V_2 : F(x, y) = 0 \right\} ,
	\]
	denn
	\begin{description}
		\item[``$ \subseteq  $''] 
			Sei $ x \in V_1 $.
			Dann ist $ (x, g(x)) \in V_1 \times V_2 $ und $ g(x) = h^{*} (x, 0) $, also
			\[
				F(x, g(x)) = 0
			\]
			nach \eqref{eq:6.11.1}.
		\item[``$ \subseteq $'']
			Sei $ (x, y) \in V_1 \times V_2 $ mit $ F(x, y) = 0 $.
			Dann $ (x, y) \in V $, nach \eqref{eq:6.11.1} $ y = g(x) $
	\end{description}
	Nun zum Zusatz:
	Es ist $ F(a, g(x)) = 0 $, da $ g(a) = b $.
	Also gilt für $ g(x) = F(x, g(x)) $, dass
	\[
		0 = Dg(a) = DF(a, b) \begin{pmatrix} E_k \\ Dg(a) \end{pmatrix} = \frac{ \partial F }{ \partial x } (a, b) + \frac{ \partial F }{ \partial y } (a, b) Dg(a)
	\]
	und daher
	\[
		Dg(a) = - \left( \frac{ \partial F }{ \partial y } (a, b) \right) ^{-1}  \frac{ \partial F }{ \partial x } (a, b) \qed
	\]
	
\end{proof*}


