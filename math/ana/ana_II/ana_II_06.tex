\section{Der lokale Umkehrsatz \& implizite Funktionen}
\begin{itemize}
	\item $ f : \R ^n \to \R ^n $ differenzierbar.
		Was müssen an $ x_0 \in \R ^n $ fordern, so dass es eine offene Menge $ V $ mit $ x_0 \in V $ \&
		\[
			\underbrace{f : V}_{f|_V} \to f(V)
		\]
		ist bijektiv mit differenzierbarer Umkehrfunktion?
\end{itemize}
\begin{itemize}
	\item Falls $ y \in f(V) $, löse dann $ \exists ! x \in V : f(x) = y $.
\end{itemize}
\textbf{Idee:}
\[
	f(x) = \underbrace{f(x_0) + Df(x_0) (x - x_0)}_{\text{linearisertes $ f $} } + \left( \underbrace{\text{Rest} }_{\text{vernachlässigbar} } \right) 
\]
\textbf{lokal:}
\[
	f(x) \hat{=} f(x_0) + Df(x_0) (x - x_0) = y
\]
Wollen wir dies lösen, also
\[
	Df(x_0) (x - x_0) = y - f(x_0),
\]
so muss $ Df(x_0) $ invertierbar sein.
Frage: Auch hinreichend?

\begin{lemma}
	Sei $ \operatorname{GL}(n) \coloneqq \left\{ A \in \R ^{n \times n} : A \text{ invertierbar}  \right\}  $ ist offen in $ \R ^{n \times n}  $ bezüglich jeder Norm, und inv: $ \operatorname{GL}(n) \ni A \mapsto A^{-1} \in \operatorname{GL}(n) $ ist stetig differenzierbar.
\end{lemma}
\begin{proof*}[Lem \ref{6.1}]
	\[
		A = \left( a_{ij}  \right) _{i, j = 1, \dotsc, n} \mapsto \det (A)
	\]
	ist ein Polynom, also stetig.
	Damit $ \operatorname{GL}(n) = \det ^{-1} \underbrace{\left( \R \setminus \left\{ 0 \right\}  \right) }_{\text{offen} } $ offen.
	Differenzierbarkeit folt aus der \textsc{Cramer}schen Regel.
	Nämlich
	\[
		A^{-1} = \frac{ 1 }{ \det(A) } \operatorname{cof}(A)^{T} 
	\]
	\[
		\operatorname{cof}(A) = m_{ij} = (-1)^{i + j} M_{ij} 
	\]
	mit $ M_{ij}  $, der $ (i, j) $-te Minore ($ \det $ der Matrix, ohne $ i $-te Zeile, $ j $-te Spalte).
	(siehe Lin. Alg.)
	Damitist $ A \mapsto \frac{ 1 }{ \det (A) } \operatorname{cof}(A) $ ist eine $ \R ^{n \times n}  $ - wertige Abbildung, deren einzelne Komponeneten rationale Funktionen in den Einträgen von $ A $ sind. $ \to  $ \textbf{Beh.}
\end{proof*}

\begin{theorem}
	Sei $ \Omega_1 \subset \R ^n $ offen, $ \Omega_2 \subset \R ^m $ offen, $ f : \Omega_1 \to \Omega_2 $ bijektiv.
	Sei $ x_0 \in \Omega_1 $ \& $ f $ in $ x_0 $ differenzierbar, $ f^{-1}  $ in $ f(x_0) $ differenzierbar.
	Dann $ m = n $; $ Df(x_0) $ inv \& $ \left( Df^{-1}  \right) \left( f(x_0) \right) = \left( Df(x_0) \right) ^{-1}  $ 
	\[
		f^{-1} (f(x)) = x,
		f(f^{-1} (x)) = x 
		\implies \left( Df^{-1} \right) (x) \left( Df \right) \left( f^{-1} (x) \right) = 1
	\]
\end{theorem}
\begin{proof*}[Thm. \ref{6.2}]
	\[
		A \coloneqq Df(x_0), B \coloneqq \left( Df^{-1}  \right) \left( f(x_0) \right) .
	\]
	\[
		f \circ f^{-1} = \id_{\Omega_2} \& f^{-1} \circ f = \id_{\Omega_1} \overset{\text{Kettenregel} }{\implies } BA = E_n, AB = E_m
	\]
	Mit $ E_n $ ist die $ (n \times n ) $-Einheitsmatrix,
	und $ E_m $ ist die $ (m \times m ) $-Einheitsmatrix.\qed
\end{proof*}

\textbf{Frage:} Wann ist $ f^{-1}  $ differenzierbar?

\begin{lemma}
	Seien $ \Omega_1, \Omega_2 \subset \R ^n $ offen, $ f : \Omega_1 \to \Omega_2 $ bijektiv \& stetig differenzierbar.
	Weiter sei $ f^{-1}  $ stetig \& $ \forall x \in \Omega_1 : Df(x) \in \GL(n) $.
	Dann ist $ f^{-1}  $ \textbf{stetig differenzierbar}.
\end{lemma}
\begin{proof*}[Lem \ref{6.3}]
	Zu zeigen: $ f^{-1}  $ ist in jedem $ b = f(x_0), x_0 \in \Omega_1 $ differenzierbar
	\begin{description}
		\item[1. Reduktion] $ x_0 = 0, f(x_0) = 0. $ 
			Ansonsten betrachte $ g(x) = f(x_0 + x) - f(x_0) $
		\item[2. Reduktion] Sei $ C \coloneqq \left( Df(0) \right) ^{-1}  $. Wir zeigen die Aussage für $ \left( Cf \right) ^{-1}  $,
			denn $ D \left( \left( C f \right)^{-1}   \right) = C^{-1}  \left( Df^{-1}  \right)  $ (+ $ C $ invertierbar!)
		\item[3. Reduktion] Damit \OE{} $ Df(0) = E_n $.
	\end{description}
	\begin{itemize}
		\item $ f $ in $ 0 $ differenzierbar $ \implies  \exists  r : \Omega_1 \overset{\text{stetig} }{\to }\R ^n : r(0) = 0 $ 
			\& $ \forall x \in \Omega_1 : f(x) \overset{\text{\textbf{3. Red.}} }{=} f(0) + E_n x + r(x) \left\| x \right\| _2 $ 
		\item Setze
			\[
				s(y) \coloneqq
				\begin{cases}
					-r(x) \frac{ \left\| x \right\| _2 }{ \left\| y \right\| _2 } , & y \neq 0, y = f(x)\\
					0, & \text{sonst.} 
				\end{cases}
				{ \color{gadse-dark-green}
					\implies -r(x) \left\| x \right\| _2 = s(y) \left\| y \right\| _2
				}
			\]
			$ f(0) = 0 $. Also $ f^{-1} (0) = 0 $.
			Für alle $ y = f(x), y \neq 0 $.
			\begin{align*}
				f^{-1} (y) = x &= f(x) - r(x) \left\| x \right\| _2\\
				~ &= y + s(y) \left\| y \right\| _2 \\
				~ &= \underbrace{f^{-1} (0)}_{0} + E_n y + s(y) \left\| y \right\| _2
			\end{align*}
			Dies gilt auch für $ y = 0 $.
			Also ist $ f^{-1}  $ in $ 0 $ differenzierbar.
			Also auch für alle $ x_0 \in \Omega_1 $.
	\end{itemize}
	---\\
	Zunächst gilt
	\[
		f^{-1} \circ f = id_{\Omega_1} ,
	\]
	damit folgt nach Kettenregel
	\[
		\left( Df^{-1}  \right) \left( f(x) \right) \left( Df(x) \right) = E_n,
	\]
	also
	\[
		\left( Df^{-1}  \right) \left( f(x) \right) = \left( Df(x) \right) ^{-1} 
	\]
	mit $ y = f(x) $ folgt, dass
	\[
		Df^{-1} = \underbrace{\operatorname{inv}}_{\text{Lem. \ref{6.1} stetig}}  \circ \underbrace{Df}_{\text{stetig nach Vor.} }  \circ \underbrace{f^{-1}}_{\text{stetig nach I teil des Bew.}}
	\]
	Also $ Df^{-1}  $ stetig\qed
\end{proof*}

\begin{definition}
	Seien $ \Omega_1, \Omega_2 \subset \R ^n $ offen.
	Eine stetig differenzierbare, bijektive Abbildung $ f: \Omega_1 \to \Omega_2 $ heißt \textbf{Diffeomorphismus}, falls 
	\[
		f^{-1} : \Omega_2 \to \Omega_1
	\]
	ebenfalls stetig differenzierbar ist.
\end{definition}

\begin{theorem}[Lokaler Umkehrsatz]
	Sei $ \Omega \subset \R ^n $ offen und $ f: \Omega \to \R ^n $ stetig differenzierbar.
	Sei $ x_0 \in \Omega $ so, dass $ Df(x_0) $ invertierbar ist.
	Dann gibt es eine offene Umgebung $ U $ von $ x_0 $, sodass $ f(U) $ eine offene Umgebung von $ f(x_0) $ ist und $ f|_U : U \to f(U) $ ein Diffeomorphismus ist: $ f|_U $ ist stetig differenzierbar und $ \left( f|_{U} \right)^{-1}  $ ist stetig differenzierbar.
\end{theorem}
\begin{proof*}[Theorem \ref{6.5}]
	\OE{} Sei \OE{} $ x_0 = 0 $ und $ f(x_0) = 0 $, ansonsten betrachte $ x \mapsto f\left( x + x_0 \right) - f(x_0) $.
	Betrachte mit $ C \coloneqq \left( Df(0) \right) ^{-1}  $ den Isomorphismus $ T : x \mapsto Cx $.
	Nach Lemma \ref{5.3} gilt
	\[
		D\left( T \circ f \right) \left( 0 \right) = C \cdot Df(0) = E_n.
	\]
	Also können wir annehmen, dass $ x_0 = f(x_0) = 0 $ und $ Df(0) = E_n $
	Wir benutzen:
	\begin{itemize}
		\item den Schrankensatz
		\item den Banachschen Fixpunktsatz
	\end{itemize}
	\begin{description}
		\item[Schritt 1] Wir wollen
			\[
				f(x) = y
			\]
			lösen.
			Setzen für $ y \in \R ^n $ und $ x \in \Omega $ 
			definiere
			\[
				g_y (x) \coloneqq y + x - f(x).
			\]
			Wir wollen
			\[
				g_y(x) = x
			\]
			lösen.
			Nach Voraussetzung ist $ Df $ stetig in $ x_0 = 0 $. Daher gibt es ein $ r> 0 $, sodss
			\[
				K \coloneqq \overline{B_{2r}}(0) \subset \Omega,
			\]
			sowie
			\[
				\forall x \in K : \left\| E_n - Df(x) \right\| < \frac{ 1 }{ 2 } 
			\]
			Da $ Dg_y(x) = E_n Df(x) $, folgt mit Schrankensatz
			\[
				\forall x_1, x_2 \in K : \left\| g_y(x_1) - g_y(x_2) \right\| \leq \max_{x \in K} \left\| E_n - Df(x) \right\| \left\| x_1 - x_2 \right\| _2 \leq  \frac{ 1 }{ 2 } \left\| x_1 - x_2 \right\| _2
			\]
			Daraus folgt, dass $ \forall x \in K, y \in B_{r}(0) : \left\| g_y (x) \right\| _2 \leq \left\| g_y(x) - g_y(0) \right\| _2 + \left\| g_y(0) \right\| _2 \leq \frac{ 1 }{ 2 } \left\| x \right\| _2 + \left\| y \right\| _2 < r + r = 2r $.\\
			$ \implies g_y : K \to K $ ist kontrahierende selbstabbildung.
			Da $ K $ abgeschlossen und beschränkt ist, ist $ K $ kompakt, als BFS anwendbar.
			\begin{itemize}
				\item Ziel erreicht: Für jedes $ y \in B_{r}(0)  $ gibt es genau ein $ x \in K $ so, dass $ g_y(x) = x $, also $ f(x) = y $ mit $ B_{2r}(0)  $.
			\end{itemize}
			Für $ y \in B_{r}(0)  $ definieren wir $ h(y) $ also das eindeutig bestimmte $ x \in B_{2r}(0)  $, also
			\[
				h(y) \coloneqq  x,
			\]
			\[
				V \coloneqq B_{r}(0) 
			\]
			und 
			\[
				U \coloneqq f^{-1} (V) \cap B_{2r}(0) 
			\]
			Dies definiert eine Umkehrabbildung
			\[
				h : V \to U
			\]
			zu $ f|_U : U \to f(U) $
		\item[Schritt 2]
			\textbf{$ h $ ist stetig}\\
			Seien $ y_1, y_2 \in V $.
			Definiere $ x_1 = h(y_1) $ und $ x_2 \coloneqq h(y_2) $.
			Da $ g_y(x) = x $ folgt $ x_1 - x_2 = g_0(x_1) - g_0(x_2) + f(x_1) - f(x_2) $.
			D.h.
			\[
				\left\| x_1 - x_2 \right\| _2 \leq \frac{ 1 }{ 2 } \left\| x_1 - x_2 \right\| _2 + \left\| f(x_1) - f(x_2) \right\| _2,
			\]
			also
			\[
				\left\| h(y_1) - h(y_2) \right\| _2 \leq 2 \left\| y_1 - y_2 \right\| _2.
			\]
			Also $ h $ stetig.
		\item[Schritt 3]
				Invertierbarkeit von $ Df(x) $ auf $ U $.
				Sei $ x \in U $, dann gilt $ \left\| x \right\| _2 < 2r $.
				Dann gilt
				\[
					\forall v \in \R ^n : \left\| v - Df(x) v \right\| _2 \leq  \frac{ 1 }{ 2 } \left\| v \right\| _2
				\]
				Angenommen $ v \in \Kern \left( Df(x) \right)  $.
				Dann $ \left\| v \right\| _2 \leq \frac{ 1 }{ 2 } \left\| v \right\| _2 $, also $ v = 0 $.
				Also $ Df(x) $ invertierbar.
				Behauptung folgt aus Lemma \ref{6.3} \qed
	\end{description}
\end{proof*}

\begin{corollary}[Satz von der Offenheit]
	Sei $ \Omega \subset \R ^n $ offen und $ f: \Omega \to \R ^n $ eine stetig differenzierbare Funktion, sodass $ Df(x) $ für alle $ x \in \Omega $ invertierbar ist.
	Dann ist $ f(\Omega) $ offen.
\end{corollary}
\begin{proof*}[Korollar \ref{6.6}]
	Nach lokalem Umkehrsatz gibt es für jedes $ x \in \Omega $ eine offene Umgebung $ U_x $, sodass $ f\left( U_x \right)  $ offen ist.
	D.h.  
	\[
		f(\Omega) = \bigcup_{x \in \Omega} f\left( U_x \right) 
	\]
	ist offen.\qed
\end{proof*}

\begin{example}
	\[
		f(x_1, x_2) = \left(x_1x_2^2, x_1^{x_2} \right)
	\]
	auf
	\[
		\Omega = \left( 0, \infty \right) \times \R .
	\]
	Wir beachten, dass $ x_1^{x_2} = \exp \left( x_2 \log x_1 \right)  $.
	Damit gilt
	\[
		Df(x) =
		\begin{pmatrix} 
			x_2^2 & 2x_1x_2 \\
			x_2x_1^{x_2 - 1} & \log \left( x_1 \right) x_1^{x_2}
		\end{pmatrix} ,
	\]
	im Punkt $ (1, 2) $ gibt
	\[
		Df(1, 2) = \begin{pmatrix} 4 & 4 \\ 2 & 0 \end{pmatrix} 
	\]
	ist invertierbar.
	$ \implies  $ Lokal umkehrbar.
\end{example}


