\section{Reihen und deren Konvergenz}
\subsection{Reihen, Konvergenz und absolute Konvergenz}
\begin{subdefinition}
	Sei $ (a_n)_{n\in\N} \subset \R $ eine Folge. Dann heißt die Folge $ (s_k)_{k\in\N} $ mit
	\[ s_k \coloneqq \sum_{n = 1}^{k} a_n \]
	die \textbf{Reihe} (u $ (a_n)_{n\in\N} $ assoziiert):
	\[ \sum_{k=1}^{\infty} a_n. \]
	Die Reihe $ \sum_{n = 1}^{\infty} a_n $ heißt \textbf{konvergent}, falls die Folge $ (s_k)_{k\in\N} $ konvergiert, und wir bezeichnen dan mit $ \sum_{n=1}^{\infty} a_n $ auch ihren Limes. Andernfalls heißt die \textbf{Reihe divergent}.
\end{subdefinition}
Verschärfung:
\begin{subdefinition}
	Die Reihe $ \sum_{n = 1}^{\infty} a_n $ heißt \textbf{absolut konvergent}, falls $ \sum_{n=1}^{\infty} a_n $ konvergiert.
\end{subdefinition}
\begin{sublemma}
	Absolute Konvergenz impliziert Konvergenz.
	\begin{subproof*}
		Ist $ \sum_{n=1}^{\infty}|a_n| $ konvergent, so ist $ \left( \sum_{n=1}^{\infty}|a_n| \right)_{k\in\N} $ Cauchy.
		\[ \implies \forall \varepsilon > 0 : \exists k_0 \in \N \forall k \geq l \geq k_0 :\]
		\begin{align*}
			\sum_{n=l+1}^{k}|a_n| < &\varepsilon\\
			\implies &\left| \sum_{n=1}^{k}a_n - \sum_{n=1}^{l}a_n \right|\\
			= &\left| \sum_{n = l+1}^{k} a_n \right|\\
			\overset{\text{DUG}}{\leq} &\sum_{n=l+1}^{k} |a_n|\\
			\leq &\varepsilon
		\end{align*}
		Also $ \left(\sum_{n=1}^{k} a_n \right)_k $ Cauchy, also fergit wg Voll. ax.\qed
	\end{subproof*}
\end{sublemma}
\begin{subexample}[Geometrische Reihe]
	Sei $ q \in \R $. Dann konvergiert
	\[ \sum_{n=1}^{\infty} q^n \]
	genau dann, wenn $ | q | < 1 $. Aus Kapitel 1 wissen wir, dass
	\[ \sum_{n=0}^{N} q^n = \frac{1 - q^{N+1}}{1 - q}, \text{ also} \]
	\[ \sum_{n=1}^{N} q^n = \frac{q - q^{N+1}}{1 - q} \overset{N\to\infty}{\to} \frac{q}{1-q}. \]
	Speziell konvergiert die Reihe (absolut).
	\begin{itemize}
		\item \[ q=1: \sum_{n=1}^{N} q^n = N \to \infty, N \to infty. \]
		\item \[ q>1: \sum_{n=1}^{N} q^n, \text{ also } \sum_{n=1}^{N} \to \infty, N \to infty. \]
		\item \[ q\leq-1 \rightsquigarrow \text{ Alternation, keine Konvergenz} \]
	\end{itemize}
\end{subexample}
\begin{subexample}
	\[ \sum_{n=1}^{\infty} \frac{1}{n^2} = \frac{1}{1} + \frac{1}{4} + \frac{1}{9} + \frac{1}{16} + \dotsc = \frac{\pi^2}{6} \]
	\begin{itemize}
		\item \[ \sum_{n=2}^{\infty} \frac{1}{n(n-1)} \text{ konvergiert}\]
			\[ \frac{1}{n(n-1)} = \frac{A}{n} + \frac{B}{n-1} = \frac{A(n-1) + Bn}{n(n-1)} = \frac{\overbrace{-A}^{1} + \overbrace{(A+B)}^{=0}n}{n(n-1)} \]
			\begin{align*}
				\sum_{n=2}{N} \frac{1}{n(n-1)} &= \sum_{n=2}^{N} \left( - \frac{1}{n} + \frac{1}{n-1} \right)\\
				~&= \left( \frac{1}{1} - \frac{1}{2} \right) + \left( \frac{1}{2} - \frac{1}{3} \right) + \dotsb + \frac{1}{N-1} - \frac{1}{N}\\
				~&= 1 - \frac{1}{N} \overset{\N\to\infty}{\to} 1.
			\end{align*}
			Damit:
			\[ \forall N \in \N, N \geq 2: \sum_{n=1}^{N} \frac{1}{n^2} = 1 + \sum_{n=2}^{N} \frac{1}{n^2} \leq 1 + \underbrace{\sum_{n=2}^{N} \frac{1}{n(n-1)}}_{\text{beschränkt in N}} \implies \left(\sum_{n=1}^{N} \frac{1}{n^2} \right)_{N\in\N} \text{ beschränkt} \]
	\end{itemize}
\end{subexample}
\begin{subexample}[Harmonische Reihe]
	\[ \sum_{n=1}^{\infty} \frac{1}{n} \text{\textbf{divergent}}.\]
	\[ \sum_{n=1}^{\infty} \frac{1}{n} = 1 + \sum_{k=0}^{\infty} \underbrace{\sum_{n=2^{k} + 1}^{2^{k+1}} \frac{1}{n}}_{2^k-\text{Summanden}} \leq 1 + \underbrace{\sum_{n=2^{k} + 1}^{2^{k+1}} \frac{1}{2^{k+1}}}_{\frac{2^k}{2^{k+1}}} = 1 + \sum_{k=0}^{\infty}\frac{1}{2} = \infty \]
\end{subexample}

\begin{sublemma}
	Konvergiert
	\[ \sum_{n=1}^{\infty} a_n, \]
	so ist $ (a_n) $ eine Nullfuloge.
\end{sublemma}
\begin{subcorollary}[Trivialkriterium]
	Ist $(a_n)$ \textbf{keine Nullfolge}, so divergiert
	\[ \sum_{n=1}^{\infty} a_n \]
\end{subcorollary}

\begin{subproof*}[Lem \ref{5.1.7}]
	Nach Voraussetzung ist
	\[ \left( \sum_{n=1}^{k} a_n \right)_{k\in\N} \]
	Cauchy.
	Sei $ \varepsilon > 0 $ beliebig, so $ \exists k_0 \in \N \forall k, l \geq k_0 : \left| \sum_{n=1}^{k} a_n - \sum_{n=1}^{l}a_n \right| < \varepsilon \overset{k = l+1}{\rightsquigarrow} \forall l \geq k_0 : | a_{l+1} < \varepsilon \implies (a_n) \text{ Nullfolge} $
\end{subproof*}

\subsection{Konvergenzkriterien}
\begin{itemize}
	\item Leibniz $\rightsquigarrow$ alternierende Reihen
		\[ \sum_{n=1}^{\infty} (-1)^n a_n \]
\end{itemize}
\begin{subproposition}[Leibnizkriterium]
	Ist $ (a_n) \subset \R $ \textbf{monoton fallende Nullfolge}, so \textbf{konvergiert} die alternierende Reihe
	\[ \sum_{n=1}^{\infty} (-1)^n a_n \]
\end{subproposition}
Bemerkung:
Satz \ref{5.2.1} sagt \textbf{nichts} über absolute Konvergenz. Denn sei $ (a_n) = \left ( \frac{1}{n} \right ) $. Dann ist $(a_n)$ monoton fallende Nullfolge
\[ \sum_{n = 1}^{\infty} \left| (-1)^n a_n \right| = \sum_{n = 1}^{\infty} \frac{1}{n} = \infty \]
\begin{subproof*}[(Satz \ref{5.2.1})]
	Für
	\[k \in \N: s_k \coloneqq \sum_{n = 1}^{k} (-1)^n a_n \]
	\begin{itemize}
		\item \textbf{gerade Indices:} $ k = 2j, j \in \N $.
			\begin{align*}
				s_{2j} &= - a_1 + a_2 - a_3 + a_4 - a_5 + \dotsb + \overbrace{(-1)^{2j - 1}}^{=-1} + a_{2j} \\
				~&= -a_1 + a_2 - a_3 \dotsb - a_{2j - 1} + a_{2j} - a_{2j + 1} + \underbrace{\underbrace{a_{2j + 2}}_{a_{2(j+1)}}}_{\leq 0}
			\end{align*}
			$\implies s_{2j} \geq s_{2(j+1} $\\
			und $ s_2j \geq -a_1 + a_{2j} $ und $ a_{2j} \to 0 \implies (s_{2j}) $ nach unten beschränkt. Satz über monotone beschränkte Folgen: $ (s_{2j}) $ konvergiert $ s_{2j} \to s $
		\item \textbf{Analog:} $(s_{2j + 1})$ monoton wachsend und nach oben beschränkt $ \implies (s_{2j + 1}) $ konvergiert, $ s_{2j + 1} \to s^\prime $
	\end{itemize}
	\begin{itemize}
		\item $ S = s^\prime: | s - s^\prime | = lim_{j\to\infty} \underbrace{\left| s_{2j + 1} - s_{2j} \right |}_{ \left | \sum_{k=1}^{2j + 1} (-1)^k a_k - \sum_{k=1}^{2j} (-1)^k a_k \right | } = \lim_{j\to\infty} \left| (-1)^{2j + 1} a_{2j+1} \right| = \lim_{j\to\infty} \left | a_{2j + 1} \right| = 0. $
	\end{itemize}
	Zu zeigen : Die ganze Reihe konvergiert gegen $ s $:\\
	Sei $ \varepsilon > 0 $:
	\begin{description}
		\item[Fall 1:]
			\[ \exists k_1 \in \N \forall k \geq k_1 : \left | \sum_{n = 1}{2k} (-1)^n a_n - s \right | < \varepsilon \]
		\item[Fall 2:]
			\[ \exists k_2 \in \N \forall k \geq k_2 : \left | \sum_{n = 1}{2k + 1} (-1)^n a_n - s \right | < \varepsilon \]
	\end{description}
	Sei nun $ N \coloneqq \max\{2k_1, 2k_2 + 1\} $. Dann $ \forall j \geq N: $
			\[ \left | \sum_{n = 1}{j} (-1)^n a_n - s \right | < \varepsilon. \]
			Damit folgt die Behauptung \qed
\end{subproof*}
\begin{subexample}[Alternierende harmonische Reihe]
	\[ \sum_{n=1}^{\infty} (-1)^n \underbrace{\frac{1}{n}}_{a_n} \]
	konvergiert nach Leibniz, da $ (a_n) $ eine monoton fallende Nullfolge
	{\color{red}
	\[ \sum_{n = 1}^{\infty} (-1)^n \frac{9n^2 - n + 100}{20n^3 + n^2 + 4} \]
	dann muss man feststellen, dass gegen Null und ab einem gewissen Zeitpunkt monoton fallend
	}
\end{subexample}

\textbf{Ab jetzt: Kriterien für absolute Konvergenz}
\begin{subproposition}[(Majorantenkriterium/Minorantenkriterium)]
	Seien $ (a_n), (b_n) \in \R $ so, dass
	\begin{enumerate}[label=(\alph*)]
		\item $ | a_n | \leq | b_n | \forall n \in \N $ und
			\[ \sum_{ n = 1 }^{\infty} | b_n | < \infty. \]
			Dann konvergiert $ \sum_{n = 1}^{\infty} a_n $ absolut.
		\item $ | a_n | \leq | b_n | \forall n \in \N $ und
			\[ \sum_{ n = 1 }^{\infty} | a_n | = \infty. \]
			Dann divergiert $ \sum_{n = 1}^{\infty} a_n $.
	\end{enumerate}
\end{subproposition}

\begin{subproof*}[(Satz \ref{5.2.3})]
	Sei $ \varepsilon > 0 $. Da
	\[ \sum_{n = 1 }^{\infty} |b_n | < \infty \]
	gibt es
	\[ N \in \N : \forall k, m \geq N : \sum_{n=k}^{m} |b_n| < \varepsilon \text{ (Cauchyfolge/Partialsummen der Beträge)} \]
	Daher auch
	\[ \sum_{n=k}^{m} |a_n| \overset{\text{Vor.}}{\leq} \sum_{n=k}^{m} | b_n| < \varepsilon. \]
	Also ist 
	\[ \left ( \sum_{n = 1}^{N} |a_n| \right)_{N\in \N} \]
	Cauchy, und damit folgt (a) nach Vollständigkeit von $ \R $. (b) Analog.\qed
\end{subproof*}
{\color{yellow}\textbf{Vergleich mit geometrischen Reihen}}
\begin{subproposition}[(Quotientenkriterium)]
	Sei $ (a_n) \subset \R $
	\begin{enumerate}[label=(\alph*)]
		\item Es gebe $ 0 \leq q < 1 \wedge N \in \N $ so, dass
			\[ a_n \neq 0 \wedge \left | \frac{a_{n + 1}}{a_n} \right | \leq q \quad \forall n \geq N \]
			Dann konvergiert
			\[ \sum_{n=1}^{\infty} a_n \]
			\textbf{absolut}.

		\item Es gebe $ 1 \leq q < \infty \wedge N \in \N $ so, dass
			\[ a_n \neq 0 \wedge \left | \frac{a_{n + 1}}{a_n} \right | \leq q \quad \forall n \geq N \]
			Dann divergiert
			\[ \sum_{n=1}^{\infty} a_n. \]
	\end{enumerate}
\end{subproposition}

\begin{subproof*}[(Satz \ref{5.2.4})]
	\begin{enumerate}[label=(\alph*)]
		\item Zuerst:
			\[ a_n \neq 0, \underbrace{\left | \frac{a_{n + 1}}{a_n} \right | \leq q}_{|a_{n+1} \leq q|a_n|} ( < 1 ) \quad \forall n \geq N \]
			\textbf{Beh.:} $ \forall j \in \N_0 : | a_{N+j} | \leq q^j|a_N| $.
			\begin{description}
				\item[I.A.] $ j = 0 $ yus is correct
				\item[I.S.] $ j \curvearrowright j + 1 $.
					\[ |a_{N + j + 1}| \overset{\text{Nach Vor.}}{\leq} q | a_{N + j} | \overset{\text{I.V.}}{\leq} q^{j+1} | a_N | \qed \]
			\end{description}
			\begin{align*}
				\sum_{n=1}^{\infty} &= \underbrace{\sum_{n = 1}^{N - 1} | a_n |}_{<\infty} + \sum_{n = N}^{\infty} | a_n |\\
				~&= \left( \sum_{n = 1}^{N- 1} \right) + \sum_{j = 0}^{\infty}|a_{N+j}|\\
				~&= \underbrace{\left( \sum_{n = 1}^{N- 1} \right)}_{<\infty} + \underbrace{|a_N|\sum_{j = 0}^{\infty}q^j}_{<\infty \; q < 1 \; \text{geometrische Reihe}}\\
			\end{align*}
		\item Via Induktion.: $ |a_n | \neq 0, \forall j \in \N_0 : |a_{N+1} | \geq q^j|a_N| $\\
			$ | a_{N+1}| \geq q^j | a_N | \overset{j\to\infty}{\not\to} 0. \implies (a_n) $ keine NUllfolge $ \implies $ [Trivialkriterium] \qed
	\end{enumerate}
\end{subproof*}

\begin{subcorollary}
	Ist $ (a_n) \subset \R $ so, dass $ \exists N \in \N : \forall n \geq N : a_n \neq 0 $. Konvergiert
	\[ \left ( \left | \frac{a_{n+1}}{a_n} \right | \right ) \]
	mit 
	\[ \lim_{n\to\infty} ... \]
\end{subcorollary}
\begin{subexample}
	\[ \sum_{n = 1}^{\infty} \underbrace{\frac{n!}{2^n}}_{=a_n} \]
	1.
	\[ \forall n \in \N : a_n \neq 0 \]
	2.
	\[ \lim_{n\to\infty} \left | \frac{a_{n+1}}{a_n} \right | = ... \]
\end{subexample}

\begin{subproposition}[(Wurzelkriterium)]
	Sei $( a_n) \subset \R $. Dann:
	\begin{enumerate}[label=(\roman*)]
		\item Es gebe $ 0 \leq q < 1 \wedge N \in \N $ mit $ \sqrt[n]{|a_n|} \leq q \forall n \geq N $.
			Dann konvergiert
			\[ \sum_{n=1}^{\infty} a_n \]
			absolut.
		\item Es gebe $ 1 \leq q < \infty \wedge N \in \N $ mit $ \sqrt[n]{|a_n|} > q \forall n \geq N $.
			Dann divergergiert
			\[ \sum_{n=1}^{\infty} a_n \]
	\end{enumerate}
\end{subproposition}
\begin{subproof*}
	Analog zum Quotientenkriterium, ab $ n = N $ nutze $ | a_N | \leq q^n $ + Geom
\end{subproof*}
\begin{subexample*}
	{\color{yellow} Konvergiert $ \sum_{n=1}^{\infty} a_n $ vielleicht sogar absolut?}
	Nullfolge? -> nein reihe divergent Trivialkriterium\\
	-> ja:\\
	Alternierende Reihe -> ja Leibniz -> konvergienz/Divergenz\\
	-> nein: Quotiont -> ja Quotientenkriterium\\
	-> nein: Potent -> ja Wurzelkriterium\\
	-> nein: geeignete Maj.? -> nein: Tricky
\end{subexample*}

\begin{subcorollary}[Wurzelkriterium in Limesform]
	Ist $ a_n \subset \R $ Folge mit
	\begin{enumerate}[label=(\roman*)]
		\item \[ \lim_{n\to\infty} \sqrt[n]{|a_n|} < 1 \] , so konvergiert \[ \sum_{n=1}^{\infty} a_n \] absolut.
		\item \[ \lim_{n\to\infty} \sqrt[n]{|a_n|} > 1 \] , so divergiert \[ \sum_{n=1}^{\infty} a_n \].
	\end{enumerate}
\end{subcorollary}

\begin{subexample}
	\[\sum_{n=1}^{\infty} \underbrace{ \left( \frac{2n + 1}{3n + 2} \right)^n}_{a_n}, \]
	\begin{align*}
		\lim_{n\to\infty} \sqrt[n]{|a_n} &= \lim_{n\to\infty} \frac{2n + 1}{3n+2}\\
		~&= \frac{2}{3} < 1 \implies \text{ absolute Konvergenz nach Wurzelkriterium}\\
	\end{align*}
\end{subexample}
\[\lim_{n\to\infty} \frac{a_{n+1}}{a_n} = a \implies \lim_{\sqrt[n]{a_n}} = a,\]
falls $a, a_1, \dotsc \in \R_{>0} $\\
Das bedeuteet: Liefert das Quotientenkriterium eine Entscheidung, so auch das Wurzelkriterium. Aber \textbf{Vorsicht}, das bedeutet nicht, dass das Wurzelkriterium ``leichter'' anzuwenden ist.

\begin{subproposition}[Reihenverdichtungskriterium]
	Sei $ (a_n) \subset \R_{>0} $ eine monoton fallende Nullfolge. Dann konvergiert \[ \sum_{n=1}^{\infty} a_n \] genau dann, wenn \[ \sum_{n=1}^{\infty} 2^na_{2^n} \] konvergiert. (verdichtete Reihe)
\end{subproposition}

\begin{subproof*}[Satz \ref{5.2.10}]
	$ \forall N \in \N $
	\begin{align*}
		\sum_{n=1}^{2^N} a_n &= a_1 + \sum_{k=1}^{N} \underbrace{\sum_{n=2^{k-1} + 1}^{2^k} a_n}_{2^{k-1}\text{-Summanden}}\\
		~&\geq a_1 + \sum_{k=1}^{N} 2^{k-1}a_{2^k}\\
		~&= a_1 + \frac{1}{2} \sum_{k=1}^{N} 2^k a_{2^k}.\\
	\end{align*}
	$\implies$ Konvergiert
	\[ \sum_{n=1}^{\infty} a_n, \]
	so auch
	\[ \sum_{n=1}^{\infty} 2^n a_{2^n}\]
	\begin{align*}
		\sum_{n=1}^{2^N} a_n &= a_1 + \sum_{k=1}^{N}\sum_{n=2^{k-1}+1}^{2^k} \underbrace{a_n}_{\leq a_{2^{k-1} + 1}}\\
		~&\leq a_1 + \sum_{k=1}^{N} 2^{k-1} a_{2^{k-1} + 1}\\
		~&=a_1 + a_2 +  \sum_{k=2}^{N} 2^{k-1} a_{2^{k-1} + 1}\\
		~&=a_1 + a_2 +  \sum_{j=1}^{N-1} 2^{j} \underbrace{a_{2^j + 1}}_{\leq a_{2^j}}\\
		~&\leq a_1 + a_2 +  \sum_{j=1}^{N-1} 2^{j} a_{2^j + 1} \\
	\end{align*}
	$\implies$ Konvergiert
	\[ \sum_{n=1}^{\infty} 2^n a_{2^n}, \]
	so auch
	\[ \sum_{n=1}^{\infty} a_n.\qed\]
\end{subproof*}

\begin{subexample}
	Für welche $ s > 0 $ konvergiert \[ \sum_{n?1}^{\infty} \underbrace{ \frac{1}{n^s} }_{ a_n }? \]
	\begin{itemize}
		\item Quotientenkriterium: $\to 1$ \textsc{Fail}
		\item \textbf{Reihenverdichtung:} Verdichtete Reihe:
			\[ \sum_{n?1}^{\infty} \frac{2^n}{2^{ns}} = \sum_{n=1}^{\infty} \underbrace{ \left( 2^{1-s} \right)^n}{q}. \]
			Das ist eine \textbf{geometrische Reihe}, die genau für $ |q| < 1 $. aber $ q = 2^{1-s} < 1 $ genau dann wenn $ s > 1 $
	\end{itemize}
\end{subexample}

\subsection{Umordnung von Reihen}
\begin{subdefinition}
	Wir nennen eine bijektive Abbildung $\sigma : \N \to \N $ eine \textbf{Umordnng}. Ist $ (a_n) \subset \R $, so heißt
	\[ \sum_{n=1}^{\infty} a_{sigma(n)} \]
	die (zu $\sigma$ gehörige) Umordnung der Reihe.\\
	Wir nennen die Reihe
	\[ \sum_{n=1}^{\infty} a_n \]
	\textbf{unbedingt konvergent}, falls \textbf{jede} Umordnung der Reihe gegen denselben Wert konvergiert. Konvergiert
	\[ \sum_{n=1}^{\infty} a_n, \]
	aber nicht unbedingt, so heißt die Reihe bedingt konvergent.
\end{subdefinition}

\begin{subproposition}[Dirichletscher Umordnungssatz]
	Eine absolut konvergierende Reihe ist unbedingt konvergent.
\end{subproposition}
\begin{subproof*}[Satz \ref{5.3.2}]
	Sei
	\[ \sum_{n=1}^{\infty} a_n \]
	absolut konvergent und $ \sigma: \N \to \N $ bijektiv. Sei $ \varepsilon > 0 $
	\begin{enumerate}[label=\arabic*]
		\item 
			\[ \sum_{n=1}^{\infty} |a_n| \]
			konvergiert
			\[ \implies \exists n_0 \in \N : \forall n \geq m \geq n_0 : \sum_{k=m}^{n} |a_k| < \frac{\varepsilon}{2} \]
			(Parialsummen Cauchy)
		\item Ist 
			\[ s = \sum_{n=1}^{\infty} |a_n|, \]
			so
			\[ \exists n_1 \geq n_0 : \left | \sum_{k=1}^{n_1} a_k - s \right | < \frac{\varepsilon}{2}. \]
		\item Wähle $m_1 \in \N : \{ 1, \dotsc, n_1 \} \subset \{ \sigma(1), \dotsc, \sigma(m_1) \} ( \rightsquigarrow \sigma $ bijektiv)
	\end{enumerate}
	Sei $ n \geq m_1 $ beliebig. Dann $ \exists n_2, \dotsc, n_l \in \N : n_1 < n_2 < \dotsb < n_l $
	\[ \{ 1, \dotsc, n_1, n_2, \dotsc, n_l \} = \{ \sigma(1), \dotsc, \sigma(n) \}. \]
	\begin{align*}
		\implies \left | \sum_{k=1}^{n} a_{\sigma(k)} - \sum_{k=1}^{n_1} a_k \right | &= \left | \sum_{j=2}^{l} a_{n_j} \right |\\
		~& \leq \sum_{k=n_2}^{n_l}|a_k|\\
		~&\overset{1}{<} \frac{\varepsilon}{2}
	\end{align*}
	Nun
	\[ \forall n \geq m : \left | s - \sum_{k=1}^{n} a_{\sigma(k)} \right | \]
	\[ \leq \underbrace{ \left | s - \sum_{k=1}^{n} a_k \right| }_{\overset{2}{\leq} \frac{\varepsilon}{2}} + \underbrace{ \left | \sum_{k=1}^{n_1} a_k - \sum_{k=1}^{n} a_{\sigma(k)} \right| }_{< \frac{\varepsilon}{2}} < \varepsilon\qed \]
\end{subproof*}

\begin{subproposition}[Riemannscher Umordnungssatz]
	Sei 
	\[ \sum_{n=1}^{\infty} a_n \]
	konvergent nicht absolut konvergent. Dann gibt es zu jedem $ s \in \R $ eine Umordung $ \sigma $ mit 
	\[ \sum_{n=1}^{\infty} a_{\sigma(n)} = s.\]
\end{subproposition}

