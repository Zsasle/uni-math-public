\section*{Organisation, Tipps \& Tricks und Literaturhinweise}

Mathe...
\begin{itemize}
	\item ist intellektuell extrem herausfordernd
	\item kommt mit einem hohen Arbeitsaufwandr
	\item oft falschen Erwartungen und
	\item ist wie Ausdauersport
\end{itemize}

aber dafür ist Mathe eines der  schönsten Studien c:

Generelles Zeitmanagement:
\begin{itemize}
	\item Vor- und Nachbereitung wahrscheinlich mehr als die gesetzten $14 \times \qty{3}{\hour} = \qty{42}{\hour}$
	\item Klausurvorbereitung auch mehr als $\qty{39}{\hour}$
	\item Pro Woche $ 2 \times \qty{1.5}{\hour}$, $2 \times \qty{2}{\hour} $, $ \qty{1.5}{\hour} $, $ \qty{10}{\hour} $
	\item Es gibt immer eine Aufgabe die man nicht lösen kann
	\item In die Vorlesungen kommen
\end{itemize}

Vorlesung:
\begin{itemize}
	\item normal nicht alles zu verstehen
	\item Notizen was man nicht versteht
	\item Punkte konzise angehen
	\item \textbf{Mathe muss sich gedanklich setzen} - genügend Zeit zu verarbeiten
\end{itemize}

Übungen:
\begin{itemize}
	\item zeitintensiv
	\item Ergebnisse vernünftig aufschreiben
	\item Weg zu einer korrekter Lösung ist sehr langwierig
	\item \textbf{nicht 10 Blätter Papier ab, von denen 9.5 inkonklusiv sind}
	\item also schön Aufschreiben
\end{itemize}

Wenn wir einen Satz gezeigt bekommen, dann bekommen wir nicht die gescheiterten Jahrelangen Versuche zur Schau, sondern nur die Ausgearbeitete Lösung $\rightarrow$ also bei uns auch langer weg, aber Aufschreiben nur klein

Übungszettel:
\begin{itemize}
	\item $ 50\% $ muss richtig sein
	\item bis Freitag 10:00 Uhr
	\item in F4
	\item diese Woche nicht so umfangreich, weil weniger Zeit
	\item auf ILIAS Terminfindung Abstimmung
	\item Donnerstag Einteilung in Tutorien
	\item Blätter tackern :c
	\item alle zwei Wochen Beweismechanik Aufgaben, nur digital nicht in Papier (ist dann die letzte Aufgabe)
\end{itemize}

Literaturempfehlung:
\begin{itemize}
	\item Otto Forster: Analysis 1
		\begin{itemize}
			\item kurz und knapp - aber konzise, udn das hilft
			\item ähnliche Struktur wie Vorlesung
			\item weig motivation und wenige Querverbindungen
		\end{itemize}
	\item Königsberger: Analysis 1
		\begin{itemize}
			\item kurz - aber konzise
			\item alle themen der Vorlesung, andere Struktur
			\item mehr motivation und Querverbindungen
		\end{itemize}
	\item Klaus Fritsche: Grundkurs Analysis 1
		\begin{itemize}
			\item ausführlich
		\end{itemize}
	\item Daniel Grieser: Analysis I
		\begin{itemize}
			\item Ausfühlich, aber mit Fokus auf das Wesentliche
			\item alle Themen der Volesung enthalten, ähnliche Struktur
			\item bunt??
		\end{itemize}
	\item Harro Huser: Lehrbuch der Analysis Teil 1
		\begin{itemize}
			\item extrem ausfühlich,dick, an einigen stellen sehr extensiv
			\item alle und mehr Themen als Vorlesung
			\item Querverbindungen
		\end{itemize}
	\item Walter Rudin: Analysis
		\begin{itemize}
			\item sehr knapp und elegant
			\item klassiker
			\item alle themen der Volesung, leicht andere Struktur
			\item empfehlenswertes Buch fortgeschrittene Leser*innen
			\item nicht für Anfänger*innen
		\end{itemize}
	\item Herber amann, Joachim Escher: Analysis I
		\begin{itemize}
			\item strkt logischer Aufbau, damit teils länglich. Großes Bild
			\item alle Themen, andere Struktur
			\item auch nicht für anfänger*innen
		\end{itemize}
	\item Terence Tao: Analysis (englisch, aber gut)
	\item Rober Denk, Reinhard Racke: Kompendium der ANalysis
		\begin{itemize}
			\item kurz und knapp, teils wie Nachschlagewerk
			\item alle themen
		\end{itemize}
	\item Florian Modler, Martin Kreh: Tutorium Analysis 1 und Lineare Algebra 1
		\begin{itemize}
			\item kurz und knapp, teils wie nachschalgewerk
			\item von studierende für studierende
			\item aber enthält ein paar Fehler
		\end{itemize}
\end{itemize}

\newpage
