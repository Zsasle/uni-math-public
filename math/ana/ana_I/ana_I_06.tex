\section{Elementare topologische Konzepte in \mathsec{\R}{R}}
\subsection{Offene und abgeschlossene Mengen}
\begin{subdefinition}
	Eine Menge $ A \subset \R $ heißt \textbf{abgeschlossen} falls der Grenzwert jeder konvergenten Folge $ (x_n) \subset \R $ auch zu $ A $ gehört: $ x_1, x_2, \dotsc \in A $ und $ x_n \to x \in \R \implies x \in A $. Ist hingegen $ \R\setminus A $ abgeschlossen, so heißt $ A $ \textbf{offen}.
\end{subdefinition}

\begin{subexample}
	$ a < b $, $ A \coloneqq [a, b] \coloneqq \{ x \in \R : a \leq x \leq b \} $
	Sei $ (x_n) \subset A $, d.h., $ x_1, x_2, \dotsc \in A $ mit $ x_n \to x ( x = \lim_{n\to\infty} x_n ) $
\end{subexample}
Stabilität von `$\leq$', `$\geq$' unter Limesbildung $ \underset{a\leq x_n \leq b }{\implies} a \leq x, x \leq b \implies x \in A \implies A \text{ abgeschlossen} $.
Setzte für $ \varepsilon > 0, x \in \R: B_{\varepsilon}(x) \coloneqq \{ y \in \R : | x - y | < \varepsilon \} $ ``offener $\varepsilon$-Ball um $x$''

\begin{sublemma}
	Eine Menge $ A \subset \R $ ist offen genau dann, wenn
	\[ \forall x \in A \exists \varepsilon > 0 : B_{\varepsilon} (x) \subset A \]
\end{sublemma}

\begin{subproof*}[Lemma \ref{6.1.3}]
	\begin{description}
		\item[``$\implies$''] Angenommen $ \exists x \in A \forall \varepsilon > 0 : B_{\varepsilon}(x) \not\subset A $\\
			$\implies \exists x \in A \forall n \in \N : \exists x_n \in B_{\frac{1}{n}}(x) \cap ( \R \setminus A ).$
			$\implies (x_n) \subset \R \setminus A \wedge x_n \to x \in A $. Also kann $ \R \setminus A $ nicht abgeschlossen sein, also $A $ nicht offen. $ \implies ( A \text{ offen } \implies \text{Bedinung gilt} ) $\qed
		\item[``$ \impliedby $''] zu zeigen ... gilt $ \implies $ $ A $ offen. ( $ \iff \R\setminus A $ enthalten )\\
			Sei $ (x_{n}) \subset \R \setminus A $ konvergent gegen $ x \in  \R $. Angenommen, $ x \in  A $. Nach stern $ \exists \varepsilon > 0 : B_{\varepsilon}(x) \subset A $. Da $  x_{n} \to x: \exists N \in  \N \forall n\geq N: x_{n} \in B_{\varepsilon}(x) \subset A $. Damit $ \forall n \geq N: x_{n} \in A $ Widerspruch zu $ x_{n} \in \R \setminus A $\qed
	\end{description}
\end{subproof*}

\begin{subtheorem}[(offene Teilmenge als Topologie)]
	Das System $ T $ aller offenen Teilmengen von $  \R $ hat folgende Eigenschafften
	\begin{enumerate}[label=T\arabic*)]
		\item $ \emptyset, \R \in T $
		\item $ A, B \in T \implies A \cap B \in T $
		\item Ist $ I $ Eine Indexmenge und $ (A_{i})_{i \in I} \subset T $, so ist $ \bigcup_{i \in  I} A_{i} \in T $
	\end{enumerate}
	Das bedeutet, dass $ T $ eine \textbf{Topologie} auf $ \R $ ist.
	\begin{itemize}
		\item Allgemein: Topologien $ \hat{=} $ Systeme offener Mengen
	\end{itemize}
\end{subtheorem}

\begin{subproof*}
	Alles nach Lem. \ref{6.1.3}
	\begin{enumerate}[label=(T\arabic*)]
		\item $ \OO , \R \in  \mathcal{T}  $
			\[
			A \text{ offen } \iff \forall x \in A: \exists \varepsilon > 0 : B_{\varepsilon}(x) \subset A
			\]
			$ A = \OO $, so trivialer weiße erfüllt. für $ A = \R $:
			\[
			\forall x \in  \R ; \forall  \varepsilon > 0: B_{\varepsilon}(x) \subset \R \implies \text{ (T1)}
			\]
		\item zz.: $ A, B $ offen $ \implies A \cap B $ offen
			\begin{enumerate}[label=(\roman*)]
				\item $ A \cap B = \OO \overset{\text{(T1)}}{\implies} a \cap  B \in T $
				\item $ A \cap B \neq \OO\implies\exists x: x \in  A \wedge x \in B $.
					\[
						A \wedge B \text{ offen } \implies \exists \varepsilon_1, \varepsilon_2 > 0: B_{\varepsilon_1}(x) \subset A \wedge B_{\varepsilon_2}(x) \subset B.
					\]
					Dann $ B_{\varepsilon}(x) \subset B_{\varepsilon_1}(x) \subset A \wedge B_{\varepsilon}(x) \subset B_{\varepsilon_2}(x) \subset B $\\
					$ \implies B_{\varepsilon}(x) \subset A \cap B \implies A \cap B \in  \mathcal{T} $
			\end{enumerate}
		\item $ (A_{i} )_{i \in I} \in \mathcal{T}  $ (d.h. $ \forall i \in  I : A_i $ ist offen)\\
			zu zeigen $ \bigcup_{i \in  I} A_i $ offen. Sei $  x \in  \bigcup_{i \in  I} A_i $. Dann $ \exists i_0 \in  I : x \in  A_{i_0}  $. Aber $  A_{i_0}  $ offen, also $  \exists \varepsilon > 0: B_{\varepsilon}(x) \subset A_{i_0}  $. Damit $ B_{\varepsilon}(x) \subset A_{i_0} \subset \bigcup_{i \in  I} A_{i}  $. Also $ \bigcup_{i \in  I} A_{i}  $ offen.\qed
	\end{enumerate}
\end{subproof*}

\begin{sublemma}
	Seien $ -\infty \leq a \leq b \leq \infty $. Dann ist $ (a,b) $ offen und für $ -\infty < a \leq b < \infty $ das Intervall $ [a,b] $ abgeschlossen.
\end{sublemma}

\begin{subproof*}[Lem \ref{6.1.5}]
	$ (a,b) $ offen: $ x \in  (a, b) \; \varepsilon \coloneqq \min \{d_1, d_2\} \implies B_{\varepsilon}(x) \subset (a, b) $ enthalten.\\
	$ \implies (a, b) $ offen. $ [a, b] $ abgeschlossen.: Hierzu $ \R \setminus [a, b] $ offen. Aber $  \R \setminus [a, b] = (-\infty, a) \cup (b, \infty) $. Das ist nach Thm \ref{6.1.4} offen $ \implies [a, b] $ abgeschlossen\qed
\end{subproof*}

\begin{subdefinition}[Häufungs-, Berührpunkte]
	Sei $ A \in  \R \wedge a \in \R $. Wir nennen $ a $ einen
	\begin{enumerate}[label=\roman*)]
		\item \textbf{Berührungspunkt} von $ A $, falls in jedem $ B_{\varepsilon}(a) \varepsilon > 0 $, mindestens ein Element aus $ A $ liegt.
		\item \textbf{Häufungspunkt} von $ A $, falls in jedem $ B_{\varepsilon}(a), \varepsilon > 0 $, unendlich viele Elemente aus $ A $ liegen
	\end{enumerate}
\end{subdefinition}

\begin{subexample}
	$ A = (0, 1] \cup \{2\} $
	\begin{itemize}
		\item Dann ist $ a=0 $ Berührpunkt und Häufungspunkt
	\end{itemize}
	Formal: Sei $ \varepsilon > 0 $ und sei $ (x_n) = (\frac{1}{n}) $. Dann $ (x_n) \subset (0, 1] $ und $ x_n \to 0 $. D.h. $ \exists N \in  \N \forall n \geq N: |x_n| < \varepsilon $, also $ 0 < x_n < \varepsilon $. Damit ist 0 Häufungpunkt von $ A $.
	\begin{itemize}
		\item Dann ist $ a=2 $ Berührpunkt, aber kein Häufungspkt.\\
			Ist $ 0< \varepsilon < 1 $, so $ 'B_{\varepsilon}(2) \cap A = \{2\} $. Also ist $ a = 0 $ Berührungspunkt, aber kein Häufungspunkt von $ A $
	\end{itemize}
\end{subexample}

\subsection{Kompaktheit}
\begin{subdefinition}
	Wir nenn eine Menge $ A \subset \R $ \textbf{kompakt} falls jede Folge in $ A $ eine Teilfolge hat, die gegen ein Element aus $ A $ konvergiert. D.h.: Ist $ (x_n) \subset A $, so $ \exists (x_{n_k}) \subset  (x_n) \exists x \in A: x \lim_{k \to \infty} x_{n_k}$.
\end{subdefinition}

\begin{subexample}
	Wir betrachten
	\[
		A \coloneqq \{\frac{1}{n}: n \in \N\} ,
	\]
	\[
		B \coloneqq [0, 1], 
	\]
	\[
		C \coloneqq [0, \infty) ( = \{x \in \R: x \geq 0\} )
	\]
	\begin{itemize}
		\item $ A $ ist \textbf{nicht kompakt:} Betrachte $ (x_n) = (\frac{1}{n}) $.\\
			Per Def: $ (x_n) \subset A $. Aber $ x \to 0 $. Jede Teilfolge von $ (x_n) $ konvergiert auch gegen $ x = 0 $.Aber $ 0 \not\in A $. Damit konvergiert jede Teilfolge von $ (x_n) $ gegen $ 0 \not\in $, also $ A $ nicht kompakt
		\item B ist \textbf{kompakt}: Sei $ x_n \subset [0, 1] $. Also $ (x_n) $ beschränkt. Nach Bolzano-Weierstraß hat $ (x_n) $ eine konvergente Teilfolge $ (x_{n_k}) : x_{n_k} \to x \in  \R $. Aber $  [0, 1] $ ist abgeschlossen, also $  x \in [0, 1] $. Da $ (x_n) $ beliebig, $ B $ kompakt
		\item C ist \textbf{nicht kompakt:} $ \left( C = [0, \infty) \right) $. Betrachte $  (x_n) = (n) $. Dann $ (x_n) \subset C $ und jede Teilfolge divergiert gegen $ +\infty $. Also konvergiert keine Teilfolge und damit $ C $ nicht kompakt
	\end{itemize}
\end{subexample}
\begin{subtheorem}[Heine-Borel]
	Eine Menge $ A \subset \R $ ist genau dann kompakt, wenn sie abgeschlossen und beschränkt ist.
\end{subtheorem}

\begin{subproof*}[Theorem \ref{6.2.3}]
	\begin{description}
		\item[`` $ \implies $ ''] zu zeigen: $ A $ kompakt $ \implies A $ abgeschlossen und beschränkt\\
			\textbf{Abgeschlossen:} zu zeigen $ (x_n) \subset A $ konvergiert mit $ x_n \to x \in \R $, so $ x \in  A $.\\
			Sei $ (x_n) $ eine solche Folge. DAnn konvergiert $ x_n $ gegen $ x \in  \R $. Nach Kompaktheit $ \exists (x_{n_k}) \subset (x_n) \exists < \in  A: x_{n_k} \to y $. Aber \textbf{jede} Teilfolge einer konvergenten Folge konvergiert gegen denselben Limes. Also $ x_{n_k} \to x, k \to \infty $. Aber Limiten sind eindeutig, also $ x = y \in  A $. Also $ x \in  A $, also ist $ A $ abgeschlossen\\
			\textbf{Beschränkung:} Angenommen, $ A $ ist nicht beschränkt, \OE nach oben unbeschränkt. Dann gibt es eine Folge $ x_n \subset A $, die monoton gegen $ +\infty $ divergiert. Dann aber auch jede Teilfolge, und damit kann keine Teilfolge konvergieren $ \implies $ Widerspruch zu Kompaktheit. $ \implies A $ ist beschrnänkt
			\item[`` $ \impliedby $ ''] Sei $  A \subset \R $ abgeschlossen und beschrännkkt. Sei $ (x_n) \subset A $. Da $  A $ beschränkt, ist $ (x_n) $ beschränkt. Nach Bolzano-Weierstraß $ \exists (x_{n_k}) \subset (x_n) \exists x \in \R : x = \lim_{k \to \infty} x_{n_k} $. Aber $ A $ ist abgeschlossen und damit $ x \in A $. Also ist $ A $ kompakt\qed
	\end{description}
\end{subproof*}
\begingroup
\color{blue}
\begin{itemize}
	\item Maximum ist Supremum, das in der Menge enhalten ist, und Minimum ist Infimum, das in der Menge enthalten ist
\end{itemize}
\endgroup

\begin{subtheorem}
	Ist $ K \subset \R $ kompakt, so existieren $  \max(K) \wedge \min(K) $.
\end{subtheorem}

\begin{subproof*}[Thm. \ref{6.2.4}]
	\OE für Maximum. Sei $ m \coloneqq \sup(K) $. Nach Heine-Borel: $ K $ beschränkt, also $ m < \infty $. Da $ m = \sup(K), \exists (x_n) \subset K: x_n \to m $. Nach Heine-Borel: $ K $ abgeschlossen $ \implies m = \lim_{n \to \infty} x_n \in K $. Damit ist $ m $ Maximum.\qed
\end{subproof*}

\begingroup
\color{red}
\begin{itemize}
	\item $ [0, 1] $ kompakt, $ \min[0, 1] = 0, \max[0, 1] = 1 $.
	\item $ (0, 1) $ nicht kompakt, $ \inf(0, 1) = 0 \not\in (0, 1) $ 
	\item $ [0, \infty) $ nicht kompakt, $ \sup[0, \infty) = \infty \not\in \R $.
\end{itemize}
\endgroup

\subsection{Dichtheit, \mathsec{\Q}{Q} und \mathsec{\R}{R} }
Dichtheit bezieht sich auf Approximierbarkeit.\\
\textbf{informell:} $ B \in \R $ liegt \textbf{dicht} in $ \R $, falls jedes Element aus $ \R $ durch Eemente aus $ B $ approximiert werden kann.
\begin{subdefinition}
	Sei $ A \subset \R $. Eine Teilmenge $ B \subset A $ heißt \textbf{dicht in $ A $}, falls
	\[
	\forall x \in  A \forall  \varepsilon > 0 \exists  y \in B_{\varepsilon}(x): y \in B.
	\]
\end{subdefinition}

\begin{subtheorem}[$ \Q $ ist dicht in $ \R $ ]
	\begin{itemize}
		\item Bereits in Kapitel 3 gesehen: $ \sqrt{2} $ kann durch rationale Zahlen approximiert werden. D.h. $ \exists (x_n) \subset \Q : x_n \to \sqrt{2} \not\in \Q $. (speziell $ \Q $ nicht abgeschlossen) Thm \ref{6.3.2}: Das geht für alle $ x \in \R $
	\end{itemize}
\end{subtheorem}

\begin{subtheorem}[Dezimaldarstellungen]
	\begin{equation}
		\forall x \in [0, 1] \forall k \in \N  \exists ! a_k \in \left\{ 0, \dotsc, 9 \right\} : x = \sum_{k=1}^{\infty} a_k 10^{-k}.
		\label{eq:thm.6.3.3.1}
	\end{equation}
\end{subtheorem}

\begin{subproof*}[Satz 6.3.3]
	\ref{eq:thm.6.3.3.1} entspricht $ 0, a_1 a_2 a_3 a_4 \dotsc $\\
	(Übungsblatt 6 Aufgabe 3)\qed
\end{subproof*}

\begin{subproof*}[Satz 6.3.2]
	Sei $  x \in  [0, 1] \wedge x = \sum_{k=1}^{\infty} a_k 10^{-k} $ die Dezimaldarstellung aus Satz \ref{6.3.3}. Definiere $ (x_n) $ via $ x_n \coloneqq \sum_{k=1}^{\infty} a_k 10^{-k} \in \Q  $. Aber $ x = \lim_{n \to \infty} x_n $, also ist $ \Q \cap [0, 1] $ dicht in $ [0, 1] $. Ist $ x \in  \R  $, so sei $ m \in  \Z  $ die größte ganze Zahl mit $ m \leq x $. Dann $ x = m + \Theta, \Theta \in  (0, 1] $. Nach erstem Teil $ \exists (\Theta_n) \subset \Q  \cap  [0, 1]: \lim_{n \to \infty}  \Theta_n $. Dann $ (m + \Theta_n) \subset  \Q  \wedge x = \lim_{n \to \infty} \underbrace{m + \Theta_n}_{\in \Q } $.
	Also $ \Q  $ dicht in $ \R  $.\qed
	\textbf{Erinnerung:} $ \Q  $ abzählbar (Cantorsches Diagonalschema, Kapitel 1)
\end{subproof*}

\begin{subtheorem}
	$ \R  $ ist überabzählbar
\end{subtheorem}
\begin{subproof}[Satz \ref{6.3.4}]
	Angenommen $ \R  $ abzählbar, so auch $ (0, 1) $. Dann $ \exists x_n \subset (0, 1): \forall x \in  (0, 1) \exists n \in  \N : x = x_n $. Nach \ref{6.3.3} knnen wir schreiben:\\
	$ \forall a_{ij} \in \left\{ 0, \dotsc, 9 \right\}  $
	\begin{align*}
		x_1 &= 0, a_{11} a_{12} a_{13} \dotso \\
		x_2 &= 0, a_{21} a_{22} a_{23} \dotso \\
		x_3 &= 0, a_{31} a_{32} a_{33} \dotso \\
		\vdots &= ~ \\
	\end{align*}
	
	\[
		b_{jj} \coloneqq \begin{cases}
			a_{jj} + 2 &= \text{falls} a_{jj} \leq 5 \\
			a_{jj} - 2 &= \text{falls} a_{jj} > 5 \\
		\end{cases}
	\]
	Betrachte $ z \coloneqq 0, b_{11} b_{22} b_{33} b_{44} \dotso $.
	Damit $ \forall j \in \N : \left | z - x_j \right| \geq 10^{-j} $, also $ \forall j \in \N : x_j \neq z $.\\
	$ \implies \R  $ ist überabzählbar.\qed
\end{subproof}
\framebox{ $ \left| \sum_{k=1}^{\infty} b_kk 10^{-k} - \sum_{k=1}^{\infty} a_{jk} 10^{-k} \right| = something $}

