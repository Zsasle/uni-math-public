\documentclass[sectionformat = aufgabe]{gadsescript}
\settitlelh{\today\\\semester\\Gruppe 3, Briefkasten 11, Baiz}
\setsemester{Summer Semester 2024}%
\setuniversity{University of Konstanz}%
\setfaculty{Faculty of Science\\(Mathematics and Statistics)}%
\settitle{Übungsblatt 09}
\setsubtitle{Elias Gestrich}
%Salisha Baiz, salisha.baiz@uni-konstanz.de

\begin{document}
\maketitle
\setcounter{section}{9}
\subsection{}
\textbf{Beh.:} Sei $ V_n $ die $ n \times n $ Vandermonde-Matrix. Dann gilt $ \det V_n = \prod_{1 \leq i < j \leq n}^{} (x_j - x_i)  $.
\textbf{Bew.:}
\begin{description}
	\item[I.A.] $ n = 2 $. Dann gilt $ \det V_n = x_2 - x_1 = \prod_{1 \leq i < j \leq n}^{} (x_j - x_i)  $
	\item[I.V.] $ \det V_{n} = \prod_{1 \leq i < j \leq n}^{} (x_j - x_i)  $.
	\item[I.S.] $ n \curvearrowright n + 1 $. Zu zeigen $ \det V_{n + 1} = \prod_{1 \leq i < j \leq n + 1}^{} (x_j - x_i)  $.
		\[
			\det V_{n + 1} = \sum_{k=1}^{n + 1} (-1)^{k + n + 1} x_k^{n} \det V_{n, k} = \sum_{k=1}^{n + 1} (-1)^{k + n + 1} x_k^n \prod_{1 \leq i < j \leq n + 1, i, j \neq k}^{}  (x_j - x_i)
		\]
\end{description}

\subsection{}
\begin{enumerate}[label=(\alph*)]
	\item Entwicklung nach der $ (m + n) $-ten Zeile, dann $ (m + n - 1) $-ten Zeile, usw.
		\[
			\det \begin{pmatrix} A & \mathcal{O} \\ \mathcal{O} & I_n \end{pmatrix} = 1 \cdot \det \begin{pmatrix} A & \mathcal{O} \\ \mathcal{O} & I_{n - 1}  \end{pmatrix} = \dotsb = \det(A)
		\]
		bzw. der 1-ten, 2-ten, usw. Zeile:
		\[
			\det \begin{pmatrix} I_n & \mathcal{O} \\ \mathcal{O} & A \end{pmatrix} = 1 \cdot \det \begin{pmatrix} I_{n - 1}  & \mathcal{O} \\ \mathcal{O} & A  \end{pmatrix} = \dotsb = \det(A)
		\]
	\item Wenn $ C $ invertierbar, dann $ \det(C) \neq 0 $ und $ \exists e_1, \dotsc, e_k \in Mat_{n \times n}  $ so, dass $ C = e_k \dotsb e_1 \cdot I_n $,
		Sei dann $ E_i \coloneqq \begin{pmatrix} I_n  & \mathcal{O} \\ \mathcal{O} & e_i \end{pmatrix}  $ so, dass
		\[
			\begin{pmatrix} A & B \\ \mathcal{O} & C \end{pmatrix} = E_k \dotsb E_1 \cdot \begin{pmatrix} A & B \\ \mathcal{O} & I_n \end{pmatrix} 
		\]
		(Ich weiß, ich weiß muss man noch beweisen, aber jaaa)\\
		Also
		\[
			\det 
			\begin{pmatrix} A & B \\ \mathcal{O} & C \end{pmatrix}
			= \det \left( E_k \dotsb E_1 \cdot \begin{pmatrix} A & B \\ \mathcal{O} & I_n \end{pmatrix} \right)
			= \det \left( E_k \dotsb E_1 \right) \left( \begin{pmatrix} A & B \\ \mathcal{O} & I_n \end{pmatrix} \right)
			= \det \left( C \right) \left( \begin{pmatrix} A & B \\ \mathcal{O} & I_n \end{pmatrix} \right)
		\]
		Wenn $ C $ nicht invertierbar, dann gibt es Elementarumformungen, so, dass eine Nullzeile bei Zeile $ n $ entsteht, sodass bei der Entwicklung nach der $ n $-ten Zeile 0 raus kommt.
		Also $ \det(C) = 0  = \det (C) \begin{pmatrix} A & B \\ \mathcal{O} & I_n \end{pmatrix} = \det \begin{pmatrix} A & B \\ \mathcal{O} & C \end{pmatrix} $ 
	\item Analog zur (a) Sei $ B_i $ Die $ n \times i $-Matrix, die man erhält, wenn man die $ i+1 $ bis $ n $-te Zeilen streicht.
		Betrachte
		\[
			\det \begin{pmatrix} A & B_n \\ \mathcal{O} & I_n \end{pmatrix} = 1 \cdot \det \begin{pmatrix} A & B_{n - 1} \\ \mathcal{O} & I_{n - 1}  \end{pmatrix} = \dotsb = \det(A)
		\]
	\item Betrachte:
		\[
			\det \begin{pmatrix} A & B \\ \mathcal{O} & C \end{pmatrix} \overset{\text{(b)} }{=} \det \begin{pmatrix} A & \mathcal{O} \\ \mathcal{O} & C \end{pmatrix} \overset{\text{(c)} }{=} \det(C) \cdot \det \begin{pmatrix} A & \mathcal{O} \\ \mathcal{O} & I_n \end{pmatrix} \left( = \det (A) \det (C) \right) 
		\]
\end{enumerate}

\subsection{}
\begin{enumerate}[label=(\alph*)]
	\item Betrachte
		\begin{align*}
			\adj\left( AB \right) &= B^{-1} A^{-1} AB \adj\left( AB \right)  \\
			~ &\overset{\text{Kor. 16.7} }{=} B^{-1} A^{-1} \det\left( AB \right) I_n \\
			~ &= B^{-1} A^{-1} \det(A) \det(B) I_n \\
			~ &= B^{-1} \det(B) I_n A^{-1} \det(A) I_n \\
			~ &= \adj(B) \adj(A)\qed
		\end{align*}
	\item Betrachte
		\begin{align*}
			\det\left( \adj(A) \right) &= \det\left( A^{-1} A \adj(A) \right)  \\
			~ &\overset{\text{Kor. 16.7} }{=} \det (A^{-1} ) \det\left( \det(A) I_n \right)  \\
			~ &= \det (A)^{-1} \det(A)^n  \\
			~ &= \det(A)^{n - 1}  \qed
		\end{align*}
	\item Betrachte
		\begin{align*}
			\adj\left( \adj \left( A \right)  \right)
			&\overset{\text{Kor. 16.7} }{=} \adj\left( A \right) ^{-1} \adj\left( A \right) \adj\left( \adj\left( A \right)  \right)  \\
			&= \adj\left( A \right) ^{-1} \det \left( \adj\left( A \right)  \right)  \\
			&\overset{\text{(b)} }{=} \left( A^{-1} A \adj\left( A \right)  \right) ^{-1} \det \left( A \right)^{n - 1} \\
			&\overset{\text{Kor. 16.7} }{=} \left( A^{-1} \det\left( A \right)  \right) ^{-1} \det \left( A \right)^{n - 1}   \\
			&= A \det\left( A \right) ^{-1} \det \left( A \right)^{n - 1}   \\
			&= \det \left( A \right)^{n - 2} A   \qed
		\end{align*}
\end{enumerate}

\subsection{}
Löse das Gleichungssystem
\[
	\begin{pmatrix}
		1 & 2 & 4 \\
		2 & 2 & 1 \\
		5 & 6 & 3 \\
	\end{pmatrix} 
	\begin{pmatrix} x_1 \\ x_2 \\ x_3 \end{pmatrix} 
	= 
	\begin{pmatrix} 1 \\ 1 \\ 2 \end{pmatrix} 
\]

Nach Cramers Regel:
\[
	\begin{pmatrix}
		1 & 2 & 4 \\
		2 & 2 & 1 \\
		5 & 6 & 3 \\
	\end{pmatrix} ^{-1} 
	\begin{pmatrix} 1 \\ 1 \\ 2 \end{pmatrix} 
	= 
	\begin{pmatrix} x_1 \\ x_2 \\ x_3 \end{pmatrix} 
\]
Finde 
\[
	\begin{pmatrix}
		1 & 2 & 4 \\
		2 & 2 & 1 \\
		5 & 6 & 3 \\
	\end{pmatrix} ^{-1} 
\]
(Entschuldigung, ich schreibe nicht auf, was ich gemacht habe, das war mir in \LaTeX{} zu aufwendig)
\[
	\left(
		\begin{array}{c c c | c c c}
			1 & 2 & 4 & 1 & 0 & 0 \\
			2 & 2 & 1 & 0 & 1 & 0 \\
			5 & 6 & 3 & 0 & 0 & 1 \\
		\end{array}
	\right)
	\rightsquigarrow
	\left(
		\begin{array}{c c c | c c c}
			1 &  2 &  4  &  1 & 0 & 0 \\
			0 & -2 & -7  & -2 & 1 & 0 \\
			0 & -4 & -17 & -5 & 0 & 1 \\
		\end{array}
	\right)
\]
\[
	\rightsquigarrow
	\left(
		\begin{array}{c c c | c c c}
			1 &  0 & -3 & -1 &  1 & 0 \\
			0 & -2 & -7 & -2 &  1 & 0 \\
			0 &  0 & -3 & -1 & -2 & 1 \\
		\end{array}
	\right)
	\rightsquigarrow
	\left(
		\begin{array}{c c c | c c c}
			1 &  0 &  0 &  0			&  3				& -1 \\
			0 & -2 &  0 & -2 + 7 \cdot 3^{-1}	&  1 + 7 \cdot 2 \cdot 3^{-1}	& -7 \cdot 3^{-1}  \\
			0 &  0 & -3 & -1			& -2				& 1 \\
		\end{array}
	\right)
\]
\[
	\rightsquigarrow
	\left(
		\begin{array}{c c c | c c c}
			1 & 0 & 0 & 0			&  3				& -1 \\
			0 & 1 & 0 & 1 - 7 \cdot 6^{-1}	&  -3 \cdot 6^{-1} - 14 \cdot 6^{-1}	& 7 \cdot 6^{-1}  \\
			0 & 0 & 1 & 3^{-1} 		& 2 \cdot 3^{-1}		& -3^{-1}  \\
		\end{array}
	\right)
\]
\[
	\rightsquigarrow
	\left(
		\begin{array}{c c c | c c c}
			1 & 0 & 0 &  0			&  3			& -1 \\
			0 & 1 & 0 & -1 \cdot 6^{-1}	&  - 17 \cdot 6^{-1}	& 7 \cdot 6^{-1}  \\
			0 & 0 & 1 &  3^{-1} 		& 2 \cdot 3^{-1}	& -3^{-1}  \\
		\end{array}
	\right)
\]
Sodass
\begin{align*}
	\begin{pmatrix}
		1 & 2 & 4 \\
		2 & 2 & 1 \\
		5 & 6 & 3 \\
	\end{pmatrix} ^{-1} 
	\begin{pmatrix} 1 \\ 1 \\ 2 \end{pmatrix} 
	&= 
	\left(
		\begin{array}{c c c}
			 0		&  3			& -1 \\
			-1 \cdot 6^{-1}	& - 17 \cdot 6^{-1}	& 7 \cdot 6^{-1}  \\
			 3^{-1} 	& 2 \cdot 3^{-1}	& -3^{-1}  \\
		\end{array}
	\right)
	\begin{pmatrix} 1 \\ 1 \\ 2 \end{pmatrix} \\
	~ &=
	\begin{pmatrix} 
		 0		+  3			  -2 \\
		-1 \cdot 6^{-1}	  - 17 \cdot 6^{-1}	+ 14 \cdot 6^{-1}  \\
		 3^{-1} 	+ 2 \cdot 3^{-1}	  -2 \cdot 3^{-1}  \\
	\end{pmatrix} \\
	~ &=
	\begin{pmatrix} 
		1 \\
		-4 \cdot 6^{-1} \\
		3^{-1} \\
	\end{pmatrix} \\
	~ &=
	\begin{pmatrix} 
		1 \\
		-2 \cdot 3^{-1} \\
		3^{-1} \\
	\end{pmatrix} \\
\end{align*}

\end{document}

