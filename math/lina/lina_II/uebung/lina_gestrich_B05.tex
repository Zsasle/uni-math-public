\documentclass[sectionformat = aufgabe]{gadsescript}
\settitlelh{\today\\\semester\\Gruppe 3, Briefkasten 11, Baiz}
\setsemester{Summer Semester 2024}%
\setuniversity{University of Konstanz}%
\setfaculty{Faculty of Science\\(Mathematics and Statistics)}%
\settitle{Übungsblatt 05}
\setsubtitle{Elias Gestrich}
%Salisha Baiz, salisha.baiz@uni-konstanz.de

\begin{document}
\maketitle
\setcounter{section}{5}
\subsection{}
\begin{enumerate}[label=(\alph*)]
	\item 
		\begin{description}
			\item[Beh.:] $ \Kern (D) = \left\{ 0 \right\} \cap K[x]_{= 0}   $ und $ R_{D} = K[x] $ für $ \Char (K) = 0 $
			\item[Bew.:] ~
				\begin{description}
					\item[$ \Kern (D) = \left\{ 0 \right\} \cap K{[x]}_{= 0}  $:] ~
						\begin{description}
							\item[``$ \subset $'':] 
								Sei $ f \in K[x] $, mit $ \deg f = n \geq 1 $, sodass
								\[
									f = \sum_{i=0}^{n} f_i x^i
								\]
								mit insbesondere $ f_i \neq 0 $ gilt.
								Daraus folgt:
								\[
									D(f) = \sum_{i=0}^{n} i f_i x^{i - 1} = \left( \sum_{i=1}^{n - 1} i f_i x^{i - 1} \right) + \underbrace{n f_n}_{ \neq 0} x^{n - 1} 
								\]
								Also $ \deg D(f) = n - 1 \geq 0 \implies D(f) \neq 0  $. Was zu zeigen war.
							\item[``$ \supset $'':] 
								Sei $ f = 0 $, dann folgt aus Linearität von $ D $, dass $ D(f) = D(0) = 0 $.\\
								Sei $ f \in K[x]_{=0}  $, sodass $ f(x) = c $, dann gilt $ D(f) = 0 $.\\
						\end{description}
					\item[$ R_D = K{[x]} $:] ~
						\begin{description}
							\item[``$ \subset $'':] trivial
							\item[``$ \supset $'':] 
								Sei $ f \in K[x] $, zu zeigen $ \exists g \in K[x] $ mit $ D(g) = f $.
								Da $ f \in K[x] : \exists f_0, \dotsc, f_n \in K : f = \sum_{i=0}^{n} f_i x^i $.\\
								Setze
								\[
									g \coloneqq \sum_{i=0}^{n} \frac{ f_{i}  }{ i + 1 } x^{i + 1} .
								\]
								So, dass gilt:
								\[
									D(g) = \sum_{i=0}^{n} ( i + 1 ) \cdot \frac{ f_i }{ i + 1 } \cdot x^{i + 1 - 1} = \sum_{i=0}^{n} f_i x^i = f \qed
								\]
						\end{description}
				\end{description}
		\end{description}
	\item ~
		\begin{description}
			\item[Beh.:] $ \Kern(D) = \Span \left\{ 1, x^p, x^{2p} , \dotsc \right\}  $ und $ R_D = \Span \left( \left\{ 1, x , x^2, \dotsc \right\} \setminus \left\{ x^{p - 1} , x^{2p - 1} , \dotsc \right\}  \right) \eqcolon X $
				\begin{description}
					\item[$ \Kern (D) = \Span \left\{ 1, x^p, x^{2p} , \dotsc \right\}  $:] ~
						\begin{description}
							\item[``$ \subset $'':] 
								Sei $ f \in K[x] $, so dass $ f = \sum_{i=0}^{n} f_i x^{i}  $, wobei ein $ i_0 \leq n $ existiert mit $ f_{i_0} \neq 0 $ und $ \overline{i_0} \neq 0 $.
								Daraus folgt:
								\[
									D(f) = \sum_{i=1}^{n} i f_{i} x^{i - 1} = \underbrace{i_0}_{\neq 0} \underbrace{f_{i_0}}_{\neq 0} x^{i_0 - 1} + \sum_{i=1}^{i_0 - 1} i f_i x^{i - 1} + \sum_{i=i_0 + 1}^{n} i f_i x^{i - 1} \neq 0
								\]
							\item[``$ \supset $'':] 
								Sei $ f \in K[x] $, sodass $ f_0, \dotsc, f_{n}  \in K $ existieren mit
								\[
									f = \sum_{i=0}^{n} f_{i} x^{ip} 
								\]
								so, dass
								\[
									D(f) = \sum_{i=1}^{n} i \underbrace{p}_{= 0?} f_i x^{ip - 1} = 0
								\]
								was zu zeigen war
						\end{description}
					\item[$ R_D = X $:] ~
						\begin{description}
							\item[``$ \subset $'':]
								Sei $ f \in K[x] $ gegeben, mit $ \exists f_0, \dotsc, f_n \in K : f = \sum_{i=0}^{n} f_i x^i $. Sei $ g \coloneqq D(f) $ mit $ g = \sum_{i=0}^{n - 1} g_i x^i $ zu zeigen $ D(f) = g \in X $, also $ g_{ap - 1} = 0 $ für $ a \in \N $
								\[
									D(f) = \sum_{i=1}^{n} i f_i x^{i - 1} = \sum_{i=0}^{n - 1} ( i + 1 ) f_{i + 1} x^i = \sum_{i=0}^{n - 1} g_i x^i
								\]
								Also $ g_i = ( i + 1 ) f_{i + 1}  $.
								Sei $ a \in \N $, zu zeigen $ g_{ap - 1} = 0 $:
								\[
									g_{ap - 1} = ( (ap - 1) + 1 ) f_{(ap - 1) + 1} = \underbrace{ap}_{= 0} f_{ap} = 0
								\]
								was zu zeigen war.
							\item[``$ \supset $'':] 
								Sei $ f \in X $, zu zeigen $ \exists g \in K[x] : D(g) = f $.
								Da $ f \in X $ gibt es $ f_0, \dotsc, f_n \in K $ so, dass
								\[
									f = \sum_{i=0}^{n} f_i x^i
								\]
								wobei $ f_{ap - 1} = 0 $ für $ a \in \N  $.
								Setze für $ p \nmid i $, so dass $ i \not\equiv 0 $
								\[
									g_i = \frac{f_{i - 1}}{ i } 
								\]
								und $ g_i = 0 $ sonst, sodass $ i \cdot g_i = f_{i - 1}  $ für alle $ i = 1, \dotsc, n + 1 $
								so, dass
								\[
									D(g) = \sum_{i=1}^{n + 1} i g_i x^{i - 1} = \sum_{i=1}^{n + 1} f_{i - 1}  x^{i - 1} = \sum_{i=0}^{n} f_i x^i = f\qed
								\]
						\end{description}
				\end{description}
		\end{description}
\end{enumerate}

\subsection{}
\begin{enumerate}[label=(\alph*)]
	\item es gilt für $ f = x^3 - 2x^2 + x - 4 $:
		\begin{align*}
			f^{(1)} &= 3x^2 - 4x + 1\\
			f^{(2)} &= 6x - 4 \\
			f^{(3)} &= 6 \\
			f^{(4)} &= 0
		\end{align*}
		Also $ f(2) = 2^3 - 2 \cdot 2^2 + 2 - 4 = 8 - 8 + 2 - 4 = -2 $, $ f^{(1)} (2) = 3 \cdot 2^2 - 4 \cdot 2 + 1 = 12 - 8 + 1 = 5 $, $ f^{(2)} (2) = 6 \cdot 2 - 4 = 12 - 4 = 8 $ und $ f^{(3)} (2) = 6 $
		so, dass die Taylorentwicklung von $ f $ in 2 wie folgt aussieht:
		\[
			f = \sum_{i=0}^{3} \frac{ f^{(i)} (2) }{ i! } (x - 2)^i =  -2 + 5(x - 2) + \frac{ 8 }{ 2 } (x - 2)^2 + \frac{ 6 }{ 2 \cdot 3 } (x - 2)^3 = - 2 + 5(x - 2) + 4(x - 2)^2 + (x - 2)^3
		\]
	\item ~
		\begin{description}
			\item[Vor.:] Seien $ f_0, \dotsc, f_n \in K[x]_{\leq n}  $ mit $ \deg(f) = i $ für alle $ i = 0, \dotsc, n $.
			\item[Beh.:] $ \left\{ f_0, \dotsc, f_n \right\}  $ ist eine Basis von $ K[x]_{\leq n}  $,
				also
				\begin{itemize}
					\item $ f_0, \dotsc, f_n $ l.u,
					\item $ f_0, \dotsc, f_n \in K[x]_{\leq n}  $ und
					\item erzeugend, bzw. es reicht zu zeigen, dass $ \left| \left\{ f_0, \dotsc, f_n \right\}  \right| = \dim K[x]_{\leq n} $
				\end{itemize}
			\item[Bew.:] ~
				\begin{description}
					\item[``Lineare Unabhängigketi'':] Zum Widerspruch, sei
						\[
							\sum_{i=0}^{n} a_i f_i = 0
						\]
						mit $ a_i \in K $ für $ i = 0, \dotsc, n $ und es existerit ein $ i_0 \in \left\{ 0, \dotsc, n \right\} $ mit $ a_{i_0} \neq 0 $.
						Da $ \left\{ 0, \dotsc, n \right\}  $ endlich, existiert ein größtes $ i_0 $ so, dass $ a_{i_0} \neq 0 $
						Das bedeutet aber
						\[
							\deg \left(\sum_{i=0}^{n} a_i f_i \right)= \underbrace{\deg\left(\sum_{i=0}^{i_0 - 1} a_i f_i\right) }_{< i_0} + \underbrace{\deg(a_{i_0} f_{i_0}) }_{= i_0} \overset{Aufgabe 3.4 (d)}{=} i_0
						\]
					\item[``$ f_0, \dotsc, f_n \in K{[x]}_{\leq n}  $'':] gilt nach Voraussetzung
					\item[``$ \left| \left\{ f_0, \dotsc, f_n \right\}  \right| = \dim K{[x]}_{\leq n} $'':]
					$ \left| \left\{ f_0, \dotsc, f_n \right\}  \right| = n + 1 = \dim K[x]_{\leq n} $\qed
				\end{description}
		\end{description}
	\item ~
		Für die Basiswechselmatrix $ P $ von $ \mathcal{B}^\prime   $ nach $ \mathcal{B}  $ gilt:
		\[
			\begin{pmatrix} 
				[1]_{\mathcal{B} } & [x - 1]_{\mathcal{B} } & [(x - 1)^2]_{\mathcal{B} } & \hdots & [(x - 1)^n]_{\mathcal{B} } \\
			\end{pmatrix} 
		\]
		Es gilt $ (x - 1)^i = \sum_{j=0}^{i} \binom{i}{j} x^{j}  \cdot (-1)^{i - j} $ nach Ana I.
		Also
		\[
			[(x - 1)^i]_{\mathcal{B} } =
			\begin{pmatrix} 
				(-1)^{i} \\
				(-1)^{i - 1}i \\
				\vdots \\
				(-1)^{i - j} \binom{i}{j}\\
				\vdots \\
				-i \\
				1 \\
				0 \\
				\vdots \\
				0\\
			\end{pmatrix} 
		\]
		\[
			P_{ij} =
			\begin{cases}
				(-1)^{j - i} \binom{j}{i} & \quad \text{für $ i \leq j $} \\
				0 & \quad \text{für $ i > j $} 
			\end{cases}
		\]
		
		Also ist die Basiswechselmatrix $ P $
		\[
			\begin{pmatrix} 
				1 & -1 &  1 & & \hdots & & (-1)^{j}			& \hdots & (-1)^{n - 1}         & (-1)^{n} \\
				0 &  1 & -2 & & \hdots & & (-1)^{j - 1} j		& \hdots & (-1)^{n - 1} (n - 1) & (-1)^{n - 1} n\\
				0 &  0 &  1 & & \hdots & & (-1)^{j - 2} \binom{j}{2}	& \hdots & (-1)^{n - 2} \binom{n-1}{2}& (-1)^{n - 2} \binom{n}{2}\\
				0 &  0 &  0 & \ddots& & & \vdots			&        & \vdots & \vdots \\
				\vdots & \vdots & \vdots & & & & & & & \\
				0 &  0 &  0 & & \hdots & & (-1)^{j - i} \binom{j}{i}	& \hdots & (-1)^{n - 1 - i} \binom{n-1}{i} & (-1)^{n - i} \binom{n}{i}\\
				\vdots & \vdots & \vdots & & & & \vdots & & \vdots & \vdots \\
				0 &  0 &  0 & & \hdots & & 0				& \hdots & 0 &1\\
			\end{pmatrix}  
		\]
		
	\item ~
		\begin{description}
			\item[Vor.:] Sei $ a \in K $, so ist $ l_i(p) = p^{(i)} (a) $ und $ p_j \coloneqq \frac{ 1 }{ j! } (x - a)^{j}  $ für alle $ i, j = 0, \dotsc, n $
			\item[Beh.:] $ l_i(p_j) = \delta_{ij}  $ für alle $ i, j = 0, \dotsc, n $
			\item[Bew.:] ~
				\begin{description}
					\item[``$ i < j $'':] Es gilt:
						\begin{align*}
							l_i(p_j) &= p_j^{(i)} (a) \\
							&= \left(\frac{ j \cdot (j - 1) \dotsc (i + 1) }{ j! } (x - a)^{j - i}\right) (a)  \\
							&= \frac{ 1 }{ i! } (a - a)^{j - i}  \\
							&= 0
						\end{align*}
					\item[``$ i = j $'':] Es gilt:
						\begin{align*}
							l_i(p_j) &= p_j^{(i)} (a) \\
							&= \left(\frac{ j! }{ j! } (x - a)^{j - j}\right) (a)  \\
							&= 1 (a - a)^{0}  \\
							&= 1
						\end{align*}
					\item[``$ i > j $'':] Es gilt:
						\begin{align*}
							l_i(p_j) &= p_j^{(i)} (a) \\
							&= \left(\frac{ j!  }{ j! } (x - a)^{j - j}\right)^{(i - j)}  (a)  \\
							&= (1)^{(i - j)} (a) \\
							&= 0(a)\\
							&= 0
						\end{align*}
				\end{description}
		\end{description}
\end{enumerate}

\subsection{}
Die Basiswechselmatrix von $ \mathcal{B}  $ nach $ \mathcal{B} ^\prime  $ ist
\[
	\begin{pmatrix} 
		[1]_{\mathcal{B} ^\prime } & [x]_{\mathcal{B} ^\prime } & \hdots & [x^n]_{\mathcal{B} ^\prime } 
	\end{pmatrix} 
\]
Dabei ist nach Skript 22 LA I
\[
	[x^j]_{\mathcal{B} ^\prime } =
	\begin{pmatrix} 
		L_0(x^j) \\
		L_1(x^j) \\
		\vdots \\
		L_i(x^j) \\
		\vdots \\
		L_n(x^j) \\
	\end{pmatrix} 
	=
	\begin{pmatrix} 
		t_0^j \\
		t_1^j \\
		\vdots \\
		t_i^j \\
		\vdots \\
		t_n^j \\
	\end{pmatrix} 
\]
Also ist die Basiswechselmatrix
\[
	\begin{pmatrix} 
		t_0^0 & t_0^1 & t_0^2 & \hdots & t_0^n \\
		t_1^0 & t_1^1 & t_1^2 & \hdots & t_1^n \\
		t_2^0 & t_2^1 & t_2^2 & \hdots & t_2^n \\
		\vdots & \vdots & \vdots & \ddots & \vdots\\
		t_n^0 & t_n^1 & t_n^2 & \hdots & t_n^n \\
	\end{pmatrix} 
	=
	\begin{pmatrix} 
		1 & t_0^1 & t_0^2 & \hdots & t_0^n \\
		1 & t_1^1 & t_1^2 & \hdots & t_1^n \\
		1 & t_2^1 & t_2^2 & \hdots & t_2^n \\
		\vdots & \vdots & \vdots & \ddots & \vdots\\
		1 & t_n^1 & t_n^2 & \hdots & t_n^n \\
	\end{pmatrix} \qed
\]

\subsection{}
\begin{enumerate}[label=(\alph*)]
	\item Nach Korollar 10.10 reicht es zu zeigen, dass $ 2x^2 + 4x - 10 $ und $ x + 2 $ relativprim zueinander sind.
		Da $ \deg(x + 2) = 1 $ ist $ x + 2 $ ein Primpolynom. Also reicht es zu zeigen, dass $ \forall f \in K[x] : (x + 2) \cdot f \neq 2x^2 + 4x - 10 $.
		Betrachte dafür:
		\[
			( (x + 2) \cdot f )(-2) \overset{Thm. 7.5}{=} (x + 2)(-2) \cdot f(-2) = 0 \cdot f(-2) = 0 \neq -10 = 8 - 8 - 10 = 2 \cdot (-2)^2 + 4 \cdot (-4) - 10
		\]
		Also $ x + 2 $ und $ 2x^2 + 4x - 10 $ relativprim, was zu zeigen war.
	\item Durch Ausprobieren:
		$ f(-1) = 0 $, aber $ f^{(1)} (-1) = 18 - 24 \cdot (-1) - 6 \cdot (-1)^2 + 4 \cdot (-1)^3 = 18 + 24 - 6 - 4 = 32 \neq 0 $.
		Also $ (-1) $ einfache Nullstelle von $ f $.
		Es existiert also ein $ g $ so dass $ f = (x + 1)g $.
		Setze $ g \coloneqq x^3 - 3x^2 - 9x + 27 $ so dass $ (x + 1) \cdot g = f $ gilt.
		Es gilt $ g(-3) = (-3)^3 - 3 \cdot (-3)^2 - 9 \cdot (-3) + 27 = -27 - 27 + 27 + 27 = 0 $, also ist $ -3 $ Nullstelle von $ g $.
		$ g^{(1)} (-3) = 3 \cdot (-3)^2 - 6 \cdot (-3) - 9 = 27 + 18 - 9 = 36 \neq 0 $. Also ist $ -3 $ eine einfache Nullstelle von $ g $ und es existiert ein $ h $ so, dass $ g = (x + 3) h $.
		Sezte hierfür $ h = x^2 - 6x + 9 $ so, dass $ (x + 3) h = g $.
		Betrachte $ h = x^2 - 6x + 9 = x^2 - 2\cdot 3x + 9 = (x - 3)^2 $.
		Also ist $ f = (x + 3)(x +1)(x - 3)(x - 3) $.

		Trivialer Weise ist $ -3 $ eine Nullstelle von $ f $.
		Betrachte
		\[
			f^{(1)} (-3) = 18 - 24 \cdot (-3) - 6 \cdot (-3)^2 + 4 \cdot (-3)^3 = 18 + 72 - 3 \cdot 27 - 4 \cdot 27 \]
		\[
			= 90 - 7 \cdot 27 = 10 \cdot 9 - 21 \cdot 9 = -11 \cdot 9 \neq 0
		\]
		Also ist $ -3 $ einfache Nullstelle nach Satz 9.8

		Trivialer Weise ist $ 3 $ eine Nullstelle von $ f $.
		Betrachte
		\[
			f^{(1)} (3) = 18 - 24 \cdot (3) - 6 \cdot 3^2 + 4 \cdot 3^3 = 2 \cdot 9 - 8 \cdot 9 - 6 \cdot 9 + 12 \cdot 9 = (2 - 8 - 6 + 12) \cdot 9 = 0
		\]
		und
		\[
			f^{(2)} (3) = -24 - 12 \cdot (3) + 12 \cdot (3) = -24 \neq 0
		\]
		also ist $ 3 $ eine doppelte Nullstelle nach Satz 9.8
	\item 
		Für $ a = 0 $ ist $ 0 $ eine Nullstelle mit Vielfachheit $ \geq 2 $, da $ 0^p = 0 $ und $ p \cdot 0^{p - 1} = 0 $.\\
		Für $ a \geq 1 $
		Sei $ b = a^{-p} $ so, dass $ b^p - a = 0 $. Zu zeigen $ b $ ist auch Nullstelle von der ersten Ableitung.
		Dafür gilt $ p \cdot b^{p - 1} = 0 $. \qed
\end{enumerate}

\subsection*{\textbf{Zusatzafugabe für Interessierte:}}
\begin{enumerate}[label=(\alph*)]
	\item 
		\begin{enumerate}[label=(\roman*)]
			\item Sei $ f, g \in \left\{ h \in \Q [x] : h(0) = 0 \right\} \eqcolon M $, zu zeigen $ fg \in M $, also zu zeigen $ fg \in \Q [x] $ und $ (fg)(0) = 0 $.\\
				$ fg \in \Q [x] $ trivial, und $ (fg)(0) = f(0) g(0) = 0 $ 
			\item $ x^3 \in \left\{ f \in \Q [x] : f = 0 \text{ oder } \deg(f) \leq 4 \right\} \eqcolon M $, aber $ x^3 \cdot x^3 = x^6 \not\in M $, da $ \deg x^6 = 6 > 4 $
			\item Sei $ f, g \in \left\{ h \in \Q [x] : D(h)(2) = 0 \right\}  $.
				Also
				\[
					f = \sum_{i=0}^{n} \frac{f^{(i)} (2)}{ i! } (x - 2)^{i} 
				\]
				und
				\[
					g = \sum_{i=0}^{m} \frac{g^{(i)} (2)}{ i! } (x - 2)^{i}  
				\]
				Also ist
				\begin{align*}
					D(fg)(2) &= D\left(\left( \sum_{i=0}^{n} \frac{f^{(i)} (2)}{ i! } (x - 2)^i  \right) \left( \sum_{j=0}^{m} \frac{g^{(j)} (2)}{ j!} (x - 2)^j  \right)\right)(2)  \\
					&= D \left(\sum_{i=0}^{n} \sum_{j=0}^{m} \frac{f^{(i)}(2) g^{(j)} (2) }{ i!j! } (x - 2)^{i + j}\right)(2) \\
					&= \left(\sum_{i=0}^{n} \sum_{j=0}^{m} \frac{f^{(i)}(2) g^{(j)} (2) ( i + j ) }{ i!j! } (x - 2)^{i + j - 1}\right)(2) \\
					&= \sum_{i=0}^{n} \sum_{j=0}^{m} \frac{f^{(i)}(2) g^{(j)} (2) ( i + j ) }{ i!j! } \underbrace{0^{i + j - 1}}_{= 0 \text{ für $ i + j > 1 $}}  \\
					&= \frac{f^{(0)}(2) g^{(1)} (2) ( 0 + 1 ) }{ 0!1! } 0^{0 + 1 - 1} + \frac{f^{(1)}(2) g^{(0)} (2) ( 1 + 0 ) }{ 1!0! } 0^{1 + 0 - 1} \\
					&= f^{(0)}(2) \cdot 0 + 0 \cdot g^{(0)} (2) \\
					&= 0 \\
				\end{align*}
		\end{enumerate}
\end{enumerate}


\end{document}
