\documentclass[sectionformat = aufgabe]{gadsescript}
\settitlelh{\today\\\semester\\Gruppe 3, Briefkasten 11, Baiz}
\setsemester{Summer Semester 2024}%
\setuniversity{University of Konstanz}%
\setfaculty{Faculty of Science\\(Mathematics and Statistics)}%
\settitle{Übungsblatt 11}
\setsubtitle{Elias Gestrich}
%Salisha Baiz, salisha.baiz@uni-konstanz.de

\begin{document}
\maketitle
\setcounter{section}{11}
\subsection{}
\begin{enumerate}[label=(\alph*)]
	\setcounter{enumi}{1}
	\item Sei $ A_i $ $ A $ ohne die $ i+1, \dotsc, n $-ten Zeilen und Spalten.
		Entwickle die $ n $-te Zeile $ \det(A) $, sodass
		\[
			\det(A) = (-1)^{n + n} a_{nn}  \det(A_{n - 1} ) = (-1)^{2n} a_{nn} \det(A_{n - 1} ) = a_{nn} \det(A_{n - 1} )
		\]
		Folge dieses Scheme:
		\begin{align*}
			\det(A) &= a_{nn}  \det(A_{n - 1} ) \\
			~ &= \underbrace{(-1)^{2(n - 1)}}_{= 1} a_{nn} a_{(n-1),(n-1)}   \det(A_{n - 2} ) \\
			~ &= \dotsb = a_{nn} a_{(n-1),(n-1)} \dotsb a_{22} \det(a_{11}) \\
			~ &= \prod_{i=1}^{n} a_{ii}  
		\end{align*}
\end{enumerate}

\subsection{}
\begin{enumerate}[label=(\alph*)]
	\item 
		\textbf{Vor.:} $ \mathcal{B}  $ Basis zu $ W $ und $ \left( \overline{\alpha_1} , \dotsc, \overline{\alpha_n}  \right)   $ Basis zu $ V / W $, also insbesondere $ \overline{\alpha_1} , \dotsc, \overline{\alpha_n}  $ linear unabhängig\\
		\textbf{Bew.:}
		Zu zeigen $ \mathcal{B} ^\prime \coloneqq \left\{ \alpha_1, \dotsc, \alpha_n \right\}  $ l.u. und $ \mathcal{B} \cap \mathcal{B} ^\prime  $ l.u., also $ \Span \mathcal{B} ^\prime \cap W = \left\{ 0 \right\}  $, also für $ c_1, \dotsc, c_n \in K $, die nicht alle Null sind gilt:
		\[
			\sum_{i=1}^{n} c_i \alpha_i \not\in W
		\]
		Zum Widerspruch, nehme an es gäbe soche $ c_1, \dotsc, c_n $, dann würde gelten
		\[
			\sum_{i=1}^{n} c_i \alpha_i \in W \implies \overline{\sum_{i=1}^{n} c_i \alpha_i} = \overline{0} = \sum_{i=1}^{n} c_i \overline{\alpha_i} 
		\]
		also $ \overline{\alpha_1}, \dotsc, \overline{\alpha_n}   $ linear abhängig was im Widerspruch zur Vor. steht.
	\item 
		\textbf{Vor.:} Für alle $ \alpha \in W $ gilt $ T(\alpha) \in W $\\
		\textbf{Beh.:} Für alle $ f \in K[x] $ gilt für alle $ \alpha \in W $, dass $ f(T)(\alpha) \in W $.\\
		\textbf{Bew.:} Beh. $ \forall i \in N_{0}, \alpha \in W  : T^{i} (\alpha) \in W  $.
		Bew. der Beh. durch vollständige Induktion:
		Sei $ \alpha \in W $ beliebig:
		\begin{description}
			\item[I.A.] $ i = 0 $: $ T^{0} (\alpha) = \Id(\alpha) = \alpha \in W $
			\item[I.V.] $ T^{i} (\alpha) \in W $ 
			\item[I.S.] $ i \rightsquigarrow i + 1 $:
				\[
					T^{i + 1} (\alpha) = T \left( \underbrace{T^{i} (\alpha)}_{\in W \text{ nach I.V.} } \right) \in W
				\]
		\end{description}
\end{enumerate}

\subsection{}
\begin{enumerate}[label=(\alph*)]
	\item Sei $ \mathcal{B}_1  $ Basis von $ U_1 $ und $ \mathcal{B} _2 $ Basis von $ U_2 $, sodass $ \mathcal{B} = \mathcal{B} _1 \cap \mathcal{B} _2 $ eine geordnete Basis von $ U $, wobei die ersten $ m $ Elementen aus $ U_1 $ und die $ m+1 $ bis $ n $-ten Elemente aus $ U_2 $ kommen.
		Dann ist nach Vorlesung
		\[
			[T]_{\mathcal{B} } = \begin{pmatrix} [T_{U_1} ]_{\mathcal{B} _1} & C \\ 0 & D  \end{pmatrix}
		\]
		so dass
		Das $ \MinPol T $
\end{enumerate}



\end{document}

