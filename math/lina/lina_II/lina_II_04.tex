\section{Euklidische und Unitäre Räume}
\setcounter{subsection}{24}
\subsection{Skript 25}
\setcounter{subsubsection}{14}
\subsubsection{Innere Produkte:}
\setcounter{subenvironmentnumber}{-1}
\begin{subdefinition}
	Eine \textbf{inneres Produkt} (auch \textbf{Skalar}produkt) auf $ V $ ist eine Abbildung
	\[
		V \times V \to K, (x, y) \mapsto (x | y)
	\]
	so, dass
	\begin{enumerate}[label=(\arabic*)]
		\item $ (x|y) = \overline{(y|y)}  $ {\color{gadse-orange} $ \leftarrow $ Da $ (x|x) = \overline{(x|x)}  $, also $ (x|x) \in \R  $.}
		\item $ (c_1x_1 + c_2x_2|y) = c_1(x_1|y) + c_2(x_2|y) $ 
		\item $ (x|x) \geq 0 $ und $ (x|x) = 0 \iff x = 0 $
	\end{enumerate}
\end{subdefinition}

\begin{subnote}[Bemerkung]
	Wir folgern: $ (x|c_1y_1 + c_2y_2) = \overline{(c_1y_1 + c_2y_2| x)} = \overline{c_1(y_1|x) + c_2(y_2|x)} = \overline{c_1} \overline{(y_1|x)} + \overline{c_2} \overline{(y_2|y)} = \overline{c_1} (x|y_1) + \overline{c_2} (x|y_2) $.
\end{subnote}

Wir setzen $ (x|x) = \left\| x \right\| ^2 $ und $ \left\| x \right\| \coloneqq \sqrt{(x|x)}  $ nennen wir die \textbf{Norm von $ x $}.

\begin{subnote}[Bemerkung]
	Es gilt $ \left\| cx \right\| = \left| c \right| \left\| x \right\|  $.\qed
\end{subnote}

\begin{subdefinition*}[Terminologie]
	\begin{itemize}
		\item $ K = \R , \left( V, ( \cdot | \cdot ) \right)  $ ist \textbf{euklidischer Raum} und $ ( \cdot | \cdot ) $ heißt \textbf{symmetrisch} \textbf{bilineare} \textbf{positiv definite} Form.
		\item $ K = \C, \left( V, ( \cdot | \cdot ) \right)  $ heißt \textbf{hermitescher} oder \textbf{unitärer} Raum und $ ( \cdot | \cdot ) $ ist \textbf{hermitesch} \textbf{symmetrisch}, oder \textbf{konjugiert bilieare} \textbf{positiv definite} Form
	\end{itemize}
\end{subdefinition*}

\begin{subexample}
	Auf $ V = K^n $ so definiert: \textbf{standard Skalarprodukt}
	\[
		(x|y) = \sum_{i=1}^{n} \varepsilon _i \overline{\eta_i} 
	\]
	wobei $ x = \left( \varepsilon _1, \dotsc, \varepsilon _n \right) \in V, y = (\eta_1, \dotsc, \eta_n) \in V $.
\end{subexample}

\begin{subdefinition}
	Ansatz $ \left( V, ( \cdot | \cdot ) \right)  $ 
	\begin{enumerate}[label=(\roman*)]
		\item $ x, y \in V $ sind \textbf{orthogonal} falls $ (x|y) = 0 $ (oder $ (y|x) = 0 $)
		\item $ W_1, W_2 \subseteq V $ Unterräume von $ V $\\
			$ W_1, W_2 $ sind \textbf{orthogonal} wenn $ (x|y) = 0 $ $ \forall x \in W_1 $ und $ y \in W_2 $ 
		\item $ S \subseteq V $ ist \textbf{orthonormal} falls $ (x|y) = 0 $ für $ x \neq y $ und $ (x|y) = 1 $ für $ x = y \neq 0 $.
			Also $ S = \left\{ x_1, \dotsc, x_n \right\}  $ ist orthonormal falls $ (x_i|x_j) = \delta_{ij} $ $ \forall i, j = 1, \dotsc, n $
	\end{enumerate}
\end{subdefinition}

\begin{subnote}
	\begin{enumerate}[label=(\roman*)]
		\item $ S  $ ist orthonormal $ \implies  $ $ S $ ist linear unabhängig
			\begin{subproof*}
				Sei
				\[
					\sum c_i x_i = 0 \implies 0 = \left( \sum c_ix_i|x_j \right) = \sum c_i (x_i|x_j) = c_j \quad \forall j
				\]
				
			\end{subproof*}
		\item $ \dim V = n \implies  \left| S \right| \leq n $ wenn $ S $ orthonormal
	\end{enumerate}
\end{subnote}

\begin{subnote}
	orthogonale $ \dim V \coloneqq \max \left\{ \left| S \right| ; S \text{orthonormal}  \right\}  $
\end{subnote}

\begin{subnote}
	orthonormale $ \dim V \leq \dim V $
\end{subnote}

\begin{subdefinition*}[Notation]
	Für $ S \subseteq V $ setze $ S^{\perp} \coloneqq \left\{ x \in V | (x|s) = 0 \right\} \forall s \in S $
\end{subdefinition*}

\begin{subnote}
	\begin{enumerate}[label=(\roman*)]
		\item $ S^{\perp}  $ ist ein Unterraum
		\item $ S \subseteq \left( S^{\perp}  \right) ^{\perp} = S^{\perp\perp}  $ 
		\item $ \Span  (S) \subseteq S^{\bot\bot} $
	\end{enumerate}
\end{subnote}
\begin{subproof*}[Bemerkung \ref{4.25.8}]
	Wir beweisen
	\begin{enumerate}[label=(\roman*)]
		\item $ 0 = (0|y) \implies \left\{ 0 \right\} \subseteq S^{\perp}  $
			Für $ x_1, x_2 \in S^{\perp} , c \in K : \left( x_1 + cx_2 | s \right) = (x_1|s) + c(x_2|s) = 0 $ $ \forall s \in S $
	\end{enumerate}
	(ii) und (iii) folgen aus (i) \qed
\end{subproof*}

\begin{subdefinition}
	Sei $ W \subseteq V $ ein Unterraum, $ W^{\perp}  $ heißt das \textbf{orthogonale Kompement von $ W $ in $ V $}
\end{subdefinition}

\begin{subtheorem}[Bessel's Ungleichung]
	Sei $ S = \left\{ x_1, \dotsc, x_n \right\}  $ orthonormal, $ x \in V $.
	Setze $ c_i \coloneqq (x|x_i) $ für $ i = 1, \dotsc, n $.
	Es gelten:
	\begin{enumerate}[label=(\roman*)]
		\item 
			\[
				\sum_{i}^{} \left| c_i \right| ^2 \leq \left\| x \right\| ^2
			\]
		\item 
			\[
				x^\prime \coloneqq x - \sum_{i}^{} c_i x_i
			\]
			ist orthogonal zu $ x_j $ $ \forall j = 1, \dotsc, n $
	\end{enumerate}
\end{subtheorem}
\begin{subproof*}[Satz \ref{4.25.10} Bessel's Ungleichung]
	\begin{enumerate}[label=(\roman*)]
		\item Wir berechnen
			\begin{align*}
				0 \leq (x^\prime | x^\prime ) &= (x - \sum c_i x_i | x - \sum c_i x_i) \\
				~ &= (x | x) - \sum_i c_i (x_i|x) - \sum_i \overline{c_i} (x|x_i) + \sum_{ij} c_i \overline{c_j} (x_i|x_j)  \\
				~ &= \left\| x \right\| - \sum_i c_i \overline{c_i} - \sum_i \overline{c_i} c_i + \sum_i c_i \overline{c_i}  \\
				~ &= \left\| x \right\| - \sum_i \left| c_i \right| ^2 \\
			\end{align*}
			damit ist die (i) bewiesen
		\item 
			\[
				(x^\prime | x_j) = (x | x_j) - \sum_i c_i(x_i|x_j) = c_j - c_i = 0 \qed
			\]
			
	\end{enumerate}
	
\end{subproof*}

\subsection{Skript 26}
\begin{subtheorem}[Ungleichung von Schwarz]
	Für alle $ x, y \in V $ gilt:
	\[
		\left| (x|y) \right| \leq \left\| x \right\| \left\| y \right\| .
	\]
\end{subtheorem}
\begin{subproof*}[Satz \ref{4.26.1} Ungleichung von Schwarz]
	wenn $ y = 0 $ passt.\\
	Sei $ y \neq 0 $ setze
	\[
		y_1 \coloneqq \frac{ y }{ \left\| y \right\|  }
	\]
	so, dass $ \left| y_1 \right|  $ ist orthonormal.
	Bessel $ \implies \left| (x|y_1) \right| \leq \left\| x \right\| ^2 $, d.h.
	\[
		\frac{ 1 }{ \left\| y \right\|  } \left| (x |y) \right| ^2 \leq \left\| x \right\| ^2 \implies \left| (x|y) \right| ^2 \leq \left\| x \right\| ^2 \left\| y^2 \right\| 
	\]
\end{subproof*}

\begin{subdefinition}
	\[
		\delta(x, y) = \left\| x - y \right\| 
	\]
	ist \textbf{Distanz} zwischen $ x $ und $ y $.
\end{subdefinition}

\begin{subproposition}
	$ \forall x, y, z \in V $ 
	\begin{enumerate}[label=(\roman*)]
		\item $ \delta(x, y) = \delta(y, x) $ 
		\item $ \delta(x, y) \geq 0 $, $ \delta(x, y) = 0 \iff x = y $ 
		\item $ \delta(x, y) \leq \delta(x, z) + \delta(z, y) $
	\end{enumerate}
\end{subproposition}
\begin{subproof*}[Proposition \ref{4.26.3}]
	\begin{description}
		\item[(iii)] 
			\begin{align*}
				\left\| x + y \right\| ^2 &= ( x + y | x + y ) \\
				~ &= \left\| x \right\| ^2 + (x | y) + (y | x) + \left\| y \right\| ^2 \\
				~ &= \left\| x \right\| ^2 + 2 \Real (x |y) + \left\| y \right\| ^2 \\
				~ &\leq \left\| x \right\| ^2 + 2 \left| (x | y) \right|  + \left\| y \right\| ^2 \\
				~ &\leq \left\| x \right\| ^2 + 2 \left\| x \right\| \left\| y \right\| + \left\| y \right\| ^2 \\
				~ &= \left( \left\| x \right\| + \left\| y \right\|  \right) ^2 \qed
			\end{align*}
	\end{description}
\end{subproof*}

\begin{subtheorem}[Gram-Schmidt Verfahren]
	Sei $ (V, ( \cdot | \cdot )) $ inneres Produkt $ \dim V = n $.
	Dann hat $ V $ eine orthonormale Basis.
\end{subtheorem}
\begin{subproof*}[Satz \ref{4.26.4} Gram-Schmidt Verfahren]
	Sei $ \mathcal{X}  = \left\{ x_1, \dotsc, x_n \right\} $ eine Basis für $ V $.
	Wir werden eine orthonormale Basis
	\[
		\mathcal{J} = \left\{ y_1, \dotsc, y_n \right\} 
	\]
	per Induktion aufbauen
	\begin{description}
		\item[I.A.:] $ x_1 \neq 0 $. Setze $ y_1 \coloneqq \frac{ x_1 }{ \left\| x_1 \right\|  }  $ 
		\item[I.Annahme:] Seien $ y_1, \dotsc, y_r $ schon definiert so, dass $ \left\{ y_1, \dotsc, y_r \right\}  $ orthonormal und $ y_j \in \Span \left\{ x_1, \dotsc, x_j \right\} \forall j = 1, \dotsc, r $.
		\item[I.S.:]
			Setze
			\[
				c_j \coloneqq (x_{r + 1} | y_j) \forall j = 1, \dotsc, r
			\]
			Betrachte $ z \coloneqq x_{r + 1} - \sum_{i}^{r} c_i y_i $ 
			Berechne
			\[
				(z|y_j) = (x_{r + 1} | y_j) - c_j = c_j - c_j = 0 \quad \forall j = 1, \dotsc, r
			\]
			\[
				z \in \Span \left\{ x_{r + 1}, y_1, \dotsc, y_r \right\} 
			\]
			
			Setze $ y_{r + 1} \coloneqq \frac{ z }{ \left\| z \right\|  }  $.\\
			\textbf{Bew.:} $ \left\{ y_1, \dotsc, y_{r + 1}  \right\}  $ orthonormal, l.u., $ \Span \left\{ x_1, \dotsc, x_n \right\} = \Span \left\{ y_1, \dotsc, y_n \right\}  $\qed
	\end{description}
\end{subproof*}

$ (V, ( \cdot | \cdot ) $ $ K $-VR, $ K = \R  $ oder $ \C  $: $ \dim V < \infty $
\setcounter{subenvironmentnumber}{6}
\begin{subtheorem}
	Sei $ W \subseteq V $ ein Unterraum.
	Es gelten
	\begin{enumerate}[label=(\arabic*)]
		\item $ V = W \oplus W^{\perp}  $ 
		\item $ W^{\perp\perp} = W $
	\end{enumerate}
\end{subtheorem}
\begin{subproof*}[Satz \ref{4.26.7}]
	\begin{enumerate}[label=(\arabic*)]
		\item Sei $ \mathcal{X}  = \left\{ x_1, \dotsc, x_n \right\}  $ eine orthonormale Basis für $ W $ (Existenz folgt aus Gram-Schmidt).
			Sei $ z \in V $.
			setze
			\[
				x \coloneqq \sum_{i=1}^{n} c_i x_i
			\]
			wobei $ c_i \coloneqq (z | x_i) $.
			Es ist $ x \in W $.
			Bessel liefert $ y \coloneqq z - x $ ist orthogonal zu $ x_i $ $ \forall i = 1, \dotsc, n $, und somit $ y \in W^{\perp}  $.
			Also $ z = x + y $, wobei $ x \in W, y \in W^{\perp}  $.
			Es gilt $ W \cap W^{\perp} = \left\{ 0 \right\}  $ (weil $ (x|x) = 0 \iff x = 0  $\qed
		\item Sei $ z \in V $, $ z = x + y $ wie in (1).
			Berechne $ (z | x) = \left\| x \right\| ^2 + (y|x) = \left\| x \right\| ^2 $.
			Analog $ (z|y) = \left\| y \right\| ^2 $.
			Sei nun $ z \in W^{\perp\perp}  $, dann ist $ (z|y) = 0 = \left\| y \right\|^2  $.
			Also $ z = x \in W $\qed
	\end{enumerate}
\end{subproof*}

\setcounter{subsubsection}{15}
\subsubsection{Beziehung zu linearen Funktionalen}
\begin{subtheorem}[Riesz Darstellung]
	Sei $ f \in V^* $, dann $ \exists ! y \in V $ so, dass
	\begin{equation}
		\label{eq:4.26.8.1}
		\tag{$ \dag $}
		\forall x \in V : {\color{gadse-red} f}(x) {\color{gadse-red}=} (x | {\color{gadse-red}})
	\end{equation}
\end{subtheorem}
\begin{subproof*}[Satz \ref{4.26.8} Riesz Darstellung]
	$ \exists z $ 
	\begin{itemize}
		\item Sei $ f = 0 $ setze $ y = 0 $, dann sind Forderungen erfüllt
		\item Sei $ f \neq 0 $, betrachte
			\[
				W \coloneqq \Kern (f) \subsetneq V \text{ oder} 
			\]
			\[
				W^{\perp} \neq \left\{ 0 \right\} 
			\]
		\item sei $ y_0 \neq 0 $, $ y_0 \in W^{\perp}  $ 
			\OE{} $ \left\| y_0 \right\| = 1 $.
			Setze $ y \coloneqq \overline{f(y_0)} y_0 $.
			Beobachte
			\[
				(y_0 | y) = (y_0 | \overline{f(y_0) y_0} = f(y_0) (y_0 | y_0) = f(y_0)
			\]
			somit gilt \eqref{eq:4.26.8.1} für $ y_0 $ 
		\item Für $ x = \lambda y_0 $ berechnen wir allgemein.
			\[
				f(x) = f(\lambda y_0) = \lambda f(y_0) = \lambda (y_0 | y) = (\lambda y_0 | y) = (x | y).
			\]
			Also \eqref{eq:4.26.8.1}
		\item Für $ x \in W $ berechne:
			\[
				(x | y) = (x | \overline{f(y_0)} y_0) = f(y_0) (x | y_0) = 0 = f(x).
			\]
			Also \eqref{eq:4.26.8.1} erfüllt.
			Sei nun $ x \in V $ beliebig und schreibe $ x = x_0 + \lambda y_0 $ wobei $ \lambda \coloneqq \frac{ f(x) }{ f(y_0) }  $ und $ x_0 \coloneqq x - \lambda y_0 $ 
			Berechne
			\[
				f(x_0) = f(x) - \frac{ f(x) }{ f(y_0) } f(y_0) = 0
			\]
			also ist $ x_0 \in W $ und
			\[
				f(x) = f(x_0) + f(\lambda y_0) \overset{\text{\eqref{eq:4.26.8.1}} }{=} (x_0 | y) + ( \lambda y_0 | y) = (x_0 + \lambda y_0 | y) = (x | y)
			\]
			Damit \eqref{eq:4.26.8.1} erfüllt.
	\end{itemize}
	\textbf{Eindeutigkeit:} Seien $ y_1, y_2 \in V $ mit $ (x | y_1) = (x|y_2) $ $ \forall x \in V $.
	Dann ist $ (x | (y - y_2) ) = 0 $ $ \forall x \in V $.
	Insbesondere gilt es auch für $ x = y_1 - y_2 $.
	Also $ y_1 - y_2 = 0 \implies y_1 = y_2 $\qed
\end{subproof*}

\begin{subtheorem}
	Die Abbildung
	\[
		\rho : V^* \to V, f \mapsto y
	\]
	(wobei $ y = \rho(f) $ eindeutig definiert ist (RDS) durch {\color{gadse-orange}$ f(x) = (x | \rho(f)) $ $ \forall x \in V $} erfüllt:
	\begin{enumerate}[label=(\roman*)]
		\item $ \rho(f_1 + f_2) = \rho(f_1) + \rho(f_2) $ 
		\item $ \rho $ surjektiv
		\item $ \rho $ injektiv und {\color{gadse-orange}Achtung}
		\item $ \rho(cf) = \overline{c} \rho(f) $ $ \forall c \in K $
	\end{enumerate}
	{\color{gadse-orange}$ \rho $ ist ein konjugierter Isomorphismus}
\end{subtheorem}
\begin{subproof*}[Satz \ref{4.26.9}]
	\begin{enumerate}[label=(\roman*)]
		\item ÜA
		\item Sei $ y \in V $, setze $ \forall x \in V : f(x) \coloneqq (x | y) $.
			Dann ist $ f \in V^* $ und $ \rho(f) = y $ 
		\item $ f(x) = (x|y) = 0 \implies f = 0 $ 
		\item setze $ z \coloneqq \rho(cf) $, $ y \coloneqq \rho(f) $. Zu zeigen: $ z = \overline{c} y $, d.h. zu zeigen:
			\[
				\forall x \in V : (cf)(x) = (x| \overline{c} y)
			\]
			Tatsächlich berechne:
			\[
				(cf)(x) = cf(x) = c(x | y) = (x | \overline{c} y) \qed
			\]
	\end{enumerate}
\end{subproof*}

\begin{subcorollary}
	{\color{gadse-orange}Übertragung}
	\begin{enumerate}[label=\Roman*.]
		\item $ \forall f_1, f_2 \in V^* $ setze
			\[
				(f_1 | f_2) \coloneqq (\rho(f_1) | \rho(f_2))
			\]
			definiert ein inneres Produkt auf $ V^* $.
		\item Sei $ \mathcal{X}  = \left\{ x_1, \dotsc, x_n \right\}  $ und eine Basis für $ V $, $ \exists  $ eine Basis $ \mathcal{Y} = \left\{ y_1, \dotsc, y_n \right\}  $ eine Basis für $ V $ so, dass
			\[
				(x_i | y_j) = \delta_{ij}  $ $ \forall i, j = 1, \dotsc, n
			\]
		\item Für $ W \subseteq V $ Unterräume gilt
			\[
				\rho(W^\circ) = W^{\perp} 
			\]
			{\color{gadse-red}
			\item Sei $ T \in \mathcal{L} (V, V) $ Definiere $ T^* $ durch $ (Tx |y) \coloneqq (x|T^*y) $ $ \forall x \in V $, also d.h.
				$ \forall y, z \in V : T^*(y) = z $ genau dann wenn $ \forall x \in V : (x|z) = (Tx|y) $
			}
			$ T^* \in \mathcal{L} (V, V) $\\
			\textbf{Def:} $ T^* $ ist die \textbf{transponierte konjugierte zu $ T $}
	\end{enumerate}
\end{subcorollary}

Eigenschaften von $ T^* $ 
\begin{enumerate}[label=(\arabic*)]
	\item $ (cT)^* = \overline{c} T^* $, $ c \in K $ 
	\item Seien $ \mathcal{X} $ und $ \mathcal{Y}  $ die $ \delta $-Basen wie in II.
		Sei $ [T]_{\mathcal{X} } \coloneqq A $ und
		Es gilt: $ [T^*]_\mathcal{Y} = \overline{A^{t} } \coloneqq A^{*}  $ d.h. die $ ij $-te Koeffizient von $ A^* $ ist $ \overline{a_{ji} }  $ 
	\item $ \det A^* = \overline{\det A}  $ 
	\item die Eigenwerte von $ A^{*}  $ sind die Konjugierten der Eigenwerte von $ A $\qed
\end{enumerate}

\subsection{Skript 27}
\setcounter{subsubsection}{16}
\subsubsection{Hermite'sche Operatoren}
\[
	T^* : (Tx |y) = (x | T^*y) \text{ oder } (x | Ty) = (T^*x | y)
\]
\begin{subdefinition}
	\begin{enumerate}[label=(\roman*)]
		\item Sei $ T \in \mathcal{L} (V, V) $.
			$ T $ ist \textbf{Hermite'sch} (oder \textbf{selbstadjungiert}) falls $ T^* = T $, d.h. $ T $ ist Hermite'sch falls gilt
		\item $ K = \R  $, $ T = T^* $. $ T $ ist \textbf{reell symmetrisch}
		\item $ K = \C  $, $ T = T^* $ $ T $ heißt komplex Hermite'sch
	\end{enumerate}
\end{subdefinition}

\begin{subtheorem}
	Sei $ T \in \mathcal{L} (V, V) $ Hermite'sch.
	Es gelten $ (Tx | x) \in \R  $ $ \forall x \in V $ und alle Eigenwerte von $ T $ sind reell.
\end{subtheorem}
\begin{subproof*}[Satz \ref{4.27.2}]
	Wir berechnen für $ x \in V $ 
	$ (Tx | x) = (x | T(x) = (Tx | x) $.
	Sei nun $ Tx = cx $ mit $ x \in V, x \neq 0 $ dann ist
	\[
		\underbrace{(Tx | x)}_{\in \R } = (c x | x) = c \underbrace{\left\| x \right\| ^2}_{\in \R ^x}
	\]
	$ \implies c \in \R  $.\qed
\end{subproof*}

\begin{subnote*}[Matrixdarstellung]
	Sei $ \mathcal{X}  $ eine orthonormale Basis (Gram-Schmidt). In diesem Fall ist $ \mathcal{Y}  $ von II gleich $ \mathcal{X}  $. (d.h. $ \mathcal{X}  $ ist selbstdual).
	Sei $ T $ Hermite'sch, $ T = T^* $, dann bekommen wir
	\[
		A = [T]_{\mathcal{X} } \overset{(2)}{=} \overline{A^t} = A^*
	\]
	Das heißt\\
	$ a_{ij} = \overline{a_{ji} }  $ (A ist komplex Hermite'sch). ($ \C  $)\\
	$ a_{ij} = a_{ji} $ ($ A $ ist symmetrisch) ($ \R  $)
\end{subnote*}

\begin{subnote}
	\begin{enumerate}[label=(\roman*)]
		\item umgekehrt sei $ A $ Hermite'sche Matrix, und $ \mathcal{X}  $ eine orthonormale Basis für $ V $, $ \mathcal{X}  = \left\{ x_1, \dotsc, x_n \right\}  $.
			Dann ist der Operator $ T_A \in \mathcal{L} (V, V) $ ist auch Hermite'sch
			(\textbf{Erinnerung:} 
			\[
				T_A \left( \sum_{i=1}^{n} \varepsilon _i x_i \right) \coloneqq A \begin{pmatrix} \varepsilon_1 \\ \vdots \\ \varepsilon _n \end{pmatrix} 
			\]
		\item $ T_1, T_2 $ Hermite'sch $ \implies T_1 + T_2 $ 
		\item Sei $ T \neq 0 $ Hermite'sch und $ \alpha \in K $, $ \alpha \neq 0 $, dann ist $ \alpha T $ Hermite'sch genau dann wenn $ \alpha \in \R  $ 
		\item $ T $ ist invertierbar und Hermite'sch $ \iff T^{-1}  $ Hermite'sch
	\end{enumerate}
\end{subnote}

\begin{subtheorem}
	Seien $ T_1, T_2 $ Hermite'sch. Es gilt: $ T_1 T_2 $ ist Hermite'sch genau dann wenn
	\[
		T_1 T_2 = T_2 T_1
	\]
\end{subtheorem}
\begin{subproof*}[Satz \ref{4.27.4}]
	Wir berechnen:
	\[
		\left( T_1 T_2 \right) ^* = T_1 T_2 \iff T_2^* T_1^* = T_1 T_2 \iff  T_2 T_1 = T_1 T_2 \qed
	\]
\end{subproof*}

\begin{subtheorem}
	\begin{enumerate}[label=(\roman*)]
		\item Sei $ T_1 $ Hermite'sch und $ T_2 \in \mathcal{L} (V, V) $, $ T_2 \neq 0 $ dann ist $ T_2^* T_1 T_2 $ Hermite'sch
		\item umgekehrt $ T_2^* T_1 T_2 $ Hermite'sch ist und $ T_2 $ invertierbar, dann ist $ T_1 $ Hermite'sch
	\end{enumerate}
\end{subtheorem}
\begin{subproof*}[Satz \ref{4.27.5}]
	\begin{enumerate}[label=(\roman*)]
		\item \[
				\left( T_2^* T_1 T_2 \right) ^* = T_2^* T_1^* T_2^{* *} = T_2^* T_1 T_2
		\]
	\item 
		\[
			T_2^* T_1 T_2 = \left( T_2^* T_1 T_2 \right) ^* = T_2^* T_1^* T_2
		\]
		Nun ist $ T_2 $ invertierbar, es folgt aus Bemerkung \ref{4.27.3}, dass $ T_2^* $ invertierbar ist:
		Die letzte Gleichung multiplizieren links mit $ \left( T_2^* \right) ^{-1}  $ und rechts mit $ T_2^{-1}  $ und bekommen $ T_1 = T_1^* $\qed
	\end{enumerate}
\end{subproof*}

\subsubsection{Cartesische Zerlegung eines Operators}
\begin{subdefinition}
	$ T \in \mathcal{L} (V, V) $ heißt schief Hermite'sch, wenn $ T^* = - T $
\end{subdefinition}

\begin{subnote}
	Sei $ T \in \mathcal{L} (V, V) $ 
	\begin{itemize}
		\item schreibe
			\begin{equation}
				\label{eq:4.27.7.1}
				\tag{$ \dag $}
				T = T_1 + T_2
			\end{equation}
			wobei
			\[
				T_1 \coloneqq \frac{ T + T^* }{ 2 } \text{ und} 
			\]
			\[
				T_2 \coloneqq \frac{ T - T^* }{ 2 } 
			\]
			bemerke, dass $ T_1^* = T_1 $ Hermite'sch und $ T_2^* = - T^* $ schief Hermite'sch
		\item $ K = C : T_2 $ ist schief Hermite'sch $ \iff $ $ T_2 = i T_3 $ wobei $ T_3 $ komplex Hermite'sch ist
		\item In diesem Fall ist \eqref{eq:4.27.7.1} äquivalent zu
			\[
				T = T_1 + iT_3
			\]
	\end{itemize}
\end{subnote}

\subsection{Skript 28}
\setcounter{subsubsection}{18}
\subsubsection{Isometrie}
\begin{subdefinition}
	Sei $ U \in \mathcal{L} (V, V) $ $ U $ ist eine \textbf{Isometrie} wenn $ U^* = U^{-1}  $.\\
	$ K = \R  $ \textbf{orthogonal}\\
	$ K = \C  $ \textbf{unitär}
\end{subdefinition}

\setcounter{subenvironmentnumber}{1}
\begin{subtheorem}
	Für $ U \in \mathcal{L} (V, V) $sind äquivalent:
	\begin{enumerate}[label=(\arabic*)]
		\item $ U^* U = U U^* = \Id $ 
		\item $ (Ux|Uy) = (x|y) $ $ \forall x, y \in V $ 
		\item $ \left\| Ux \right\| = \left\| x \right\|  $ $ \forall x \in V $.
		\item $ \left\| Ux - Uy \right\| = x - y $
	\end{enumerate}
\end{subtheorem}
\begin{subproof*}[Satz \ref{4.28.2}]
	\begin{description}
		\item[``(1)$ \implies  $(2)'']
			\textbf{Berechne:} $ (Ux | Uy) = (x | U^* U y) = (x | y) $
		\item[``(2)$ \implies  $(3)'']
			anwenden von (2) mit $ x = y $.
		\item[``(3)$ \implies  $(1)'']
			Wir haben wegen Erinnerung 27.0:
			\[
				\left( Ux | Ux \right) = (U^* U x | x) = (x | x)
			\]
			Also $ \left( [U^* U - \Id]x | x \right) = 0 $ $ \forall x \in V $\\
			setze: $ T \coloneqq U^* U  - \Id $. wegen \ref{4.27.3} (ii) ist Hermite'sch.
			Ferner gilt: $ (Tx|x) = 0 $ $ \forall x \in V $\\
			\textbf{Behauptung:} $ (Tx|y) = 0 $ $ \forall x, y \in V $\\
			\textbf{Beweis der Behauptung:} benutze folgende Gleichungen für Hermite'sche Operatoren.
			Für $ K = \R  $:
			\[
				2(Tx|y) = (T(x + y) | x + y) - (T(x - y)| x - y) \text{ \textbf{Bitte Prüfen!}} 
			\]
			Für $ K = \C  $ 
			\[
				4(Tx|y) = (T(x + y)| (x + y)) - (T(x - y) | x - y) + i(T(x + iy)|(x + iy)) - i(T(x - iy)|(x - iy))
			\]
			womit die Beh. bewiesen ist.\\
			Also Beh. gilt insbesondere für $ x = y $. Das heißt: $ (Tx | Tx) = 0 $ $ \forall x \in V $. D.h. $ \left\| Tx \right\| = 0 $ $ \forall x \in V $, also $ Tx = 0 $ $ \forall x \in V $ $ \implies T = 0 $
	\end{description}
\end{subproof*}

\begin{subnote}
	$ U $ Isometrie $ \implies $ (4) $ \left\| Ux - Uy \right\| = \left\| x - y \right\|  $ $ \forall x, y \in V $.
\end{subnote}

\begin{subtheorem}
	Eigenwerte von Isometrien haben den absoluten Betrag gleich 1.
\end{subtheorem}
\begin{subproof*}[Satz \ref{4.28.4}]
	Sei $ c \in \C  $ Eigenwert $ x \neq 0 $, $ x \in V $ Eigenvektor.
	Also $ Ux = cx $. Es ist $ \left\| Ux \right\| = \left\| x \right\|  $ also $ \left\| cx \right\| = \left| c \right| \left\| x \right\| = \left\| x \right\|  $\qed
\end{subproof*}

\subsubsection{Orthonormal-Basis wechseln}
\begin{subtheorem}
	Sei $ \mathcal{X} = \left\{ x_1, \dotsc, x_n \right\}  $ eine orthonormale Basis, $ U \in \mathcal{L} (V, V) $.
	Dann ist $ U $ eine Isometrie genau dann wenn
	\[
		U \mathcal{X} \coloneqq \left\{ Ux_1, \dotsc, Ux_n \right\} 
	\]
	eine orthonormale Basis ist.\qed
\end{subtheorem}

\begin{subdefinition}
	Sei $ A \in M_{n \times n} (K) $ 
	\begin{itemize}
		\item wenn $ K = \R $, $ A $ ist \textbf{orthogonal} wenn $ A A^t = A^t A = I_n $ 
		\item wenn $ K = \C $, $ A $ ist \textbf{unitär} wenn $ A A^* = A^* A = I_n $
	\end{itemize}
\end{subdefinition}

\begin{subnote}
	\begin{enumerate}[label=(\roman*)]
		\item Seien $ U $ eine Isometrie und $ \mathcal{X}  $ eine orthonormale Basis, dann ist $ [U]_{\mathcal{X} }  $ unitär (bzw. orthogonal)
		\item Sei $ \mathcal{X}  $ eine orthonormale Basis und $ \mathcal{B}  $ eine beliebige Basis.
			Dann ist $ \mathcal{B}  $ orthonormal genau dann, wenn die Basiswechsel Matrix unitär (bzw. orthogonal) ist.
	\end{enumerate}
\end{subnote}

\subsubsection{Spektraltheorie}
\begin{enumerate}[label=(\alph*)]
	\item Hermite'sche $ (T^* = T) $ 
	\item schief Hermite'sche $ (T^* = -T) $ 
	\item unitäre (bzw. orthogonale) $ (T^* = T^{-1})  $
\end{enumerate}
alle diese Operatoren erfüllen folgende Eigenschaft:

\begin{subdefinition}
	$ T \in \mathcal{L} (V, V) $ ist \textbf{normal}, falls $ T^* T = T T^* $
\end{subdefinition}

\begin{subtheorem}[Spektralsatz für normale Operatoren]
	(vgl. Satz \ref{3.23.2} Primzerlegung)\\
	Sei $ T \in \mathcal{L} (V, V) $ normal und setze $ p = \MinPol(T) $.
	Dann ist $ p = p_1 \dotsb p_k $ wobei $ p_i $ irreduzible und normiert ist $ \forall i \in 1, \dotsc, k $ (i.e. $ \deg p_i = \text{1 oder 2} $)
	Für jedes $ i = 1, \dotsc, k $: Sei $ W_i \coloneqq \Kern p_i (T) $ der $ T $-invariante Unterraum von $ V $.
	Dann ist $ W_i $ orthogonal zu $ W_j $ für alle $ i \neq j $ und
	\[
		V = W_1 \oplus \dotsb \oplus W_k
	\]
	i.e. $ V $ ist die orthogonale direkte Summe von $ W_1, \dotsc, W_k $.
\end{subtheorem}
Für den Beweis brauchen wir
\begin{sublemma}
	Sei $ T \in \mathcal{L} (V, V) $ und sei $ W \subseteq V $ $ T $-invariant.
	Dann ist $ W^\perp \subseteq V $ $ T^* $-invariant
\end{sublemma}
\begin{subproof*}[Hilfslemma \ref{4.28.10}]
	Sei $ u \in W^\perp $ und $ w \in W $, und berechne
	\[
		(w | T^*u) = (Tw|u) = 0 \quad \forall w \in W
	\]
	Also ist $ T^*u \in W^\perp $\qed
\end{subproof*}

\subsection{Skript 29}

\begin{subnote}
	Da $ (T^*)^* = T $ (siehe ÜA).
	Nun können wir Lemma \ref{4.28.10} auf $ T^* $ anwenden und bekommen
	\[
		W \subseteq V \quad T^*\text{-invariant} 
	\]
	dann ist
	\[
		W^\perp \subseteq V \quad T \text{-invariant} 
	\]
\end{subnote}

\begin{sublemma}
	Sei $ T \in \mathcal{L} (V, V) $ \textbf{normal} und $ g(x) \in K[x] $ und $ W \coloneqq \Kern g(T) $.
	Dann ist $ W^\perp $ $ T $-invariant
\end{sublemma}
\begin{subproof*}[Lemma \ref{4.29.2}]
	Wir zeigen, dass $ W $ $ T^* $-invariant ist.
	Nun $ T $ normal $ \implies T T^* = T^*T \implies g(T)T^* = T^* g(T) $.
	Sei $ u \in W $. Berechne
	\[
		g(T) T^* (u) = T^* (g(T)(u)) = 0.
	\]
	Also $ T^*(u) \in W $.
	Also $ W $ $ T^* $-invariant.
	Aus Bem. \ref{4.29.1} folgt: $ W^\perp $ $ T $-invariant
\end{subproof*}

\begin{subnote}
	Sei $ g $ ein Faktor vo $ p \coloneqq \MinPol (T) $, dann ist $ g(T) $ \textbf{nicht} invertierbar.
	In der Tat:
	Sei $ p = gh $ mit $ 0 < \deg h < \deg p $.
	Wäre $ g(T) $ invertierbar, dann hätten wir
	\[
		0 = g(T)^{-1} p(T) = g(T)^{-1} g(T) h(T)
	\]
	Also $ h(T) = 0 $, aber $ \deg h < \deg p $\qed
\end{subnote}

\begin{subproof}[Spektralsatz Satz \ref{4.28.9}]
	\begin{itemize}
		\item Wir bemerkten vorab, dass $ W_i $ $ T $-invariant (siehe \ref{3.21.2} (2))
		\item Sei $ p = p_1 \dotsb p_k $ 
			wir zeigen $ p_i \neq p_j $ für $ i \neq j $ und
			\[
				V = W_1 \oplus \dotsb \oplus W_k
			\]
			die orthogonale direkte Summe.
	\end{itemize}
	Beweis per Induktion mache
	\begin{description}
		\item[Ind. Anf.:]  $ k = 1 $ passt
		\item[Ind. Ann.:] gilt $ k - 1 $ 
		\item[Ind. Sch.] Lemma \ref{4.29.2} impliziert dass $ W_1^\perp $ $ T $-invariant ist.
			Bemerke $ p_1 = \MinPol (T_{W_{1} } ) $ 
			Betrachte $ T_{W_1^\perp}  $ und 
			\[
				\Kern p_1 \left( T_{W_1^\perp}  \right) = \left\{ 0 \right\} 
			\]
			($ x \in W_1^\perp $ und $ x \in \Kern p_1 (T) = W_1 \implies x = 0 $)
			Also ist $ p_1\left( T_{W_1^\perp}  \right)  $ ist invertierbar, und \textbf{somit} (wegen Bem. \ref{4.29.3}) ist $ p_1 $ \textbf{kein} Faktor von $ \MinPol (T_{W_1^\perp} ) $.
			Setze
			\[
				P_2 \coloneqq \MinPol \left( T_{W_1^\perp}  \right) 
			\]
			Also sind $ p_1, P_2 $ teilerfremd.
			Also ist $ p = \operatorname{kgV} (p_1, P_2) = p_1 P_2 $.
			Also ist $ P_2 = p_2 \dotsb p_k $, und $ p_1 \neq p_j $ für $ j = 2, \dotsc, k $.
			\begin{itemize}
				\item Nun wollen wir Ind. Ann. auf $ T_{W_1^\perp}  $ anwenden.
					Da $ T $ normal ist, ist auch $ T_{W_1^\perp}  $ normal.
					Wir bekommen (wie im Beweis Satz \ref{3.23.3}) folgt nun
					\[
						W_1^\perp = W_2 \oplus \dotsb \oplus W_k.
					\]
			\end{itemize}
			Alles zusammen
			\[
				V = W_1 \oplus W_1^\perp = W_1 \oplus \dotsb W_k
			\]
			orthogonale direkte Summe\qed
	\end{description}
\end{subproof}

\begin{subcorollary}[zum Spektralsatz]
	für $ K = \C  $\\
	Sei $ T \in \mathcal{L} (V, V) $ normal.
	Es gibt eine orthonormale Basis bestehend aus Eigenvektoren von $ T $.
\end{subcorollary}
\begin{subproof*}[Spektralsatz]
	$ \forall i = 1, \dotsc, k $, sind die $ p_i $ linear, $ p_i = (x - z_i) $ und $ W_i $ der Eigenraum zum Eigenwert $ z_i $.
	Wähle eine orthonormal-Basis G-S. $ \mathcal{X} _i $ für $ W_i $ und setze $ \forall i = 1, \dotsc, k $ 
	\[
		\mathcal{X} = \mathcal{X} _1 U \dotsb U \mathcal{X} _k\qed
	\]
\end{subproof*}

\begin{subcorollary}[Eigenvektoren von Hermite'schen Operatoren]
	$ K = \C  $, $ T $ normal.
	Es gilt: $ T $ ist Hermite'sch $ \iff  $ Eigenwerte von $ T \in \R  $ sind.
\end{subcorollary}
\begin{subproof*}[Korollar Eigenvektoren von Hermite'schen Operatoren]
	\begin{description}
		\item[``$ \implies  $'':] 
			Satz \ref{4.27.2}
		\item[``$ \implies  $'':] 
			Folgt aus orthogonal diagonalisierbar:
			\[
				D = [T]_{\mathcal{X} } = \begin{pmatrix} d_1 & & 0 \\ & \ddots & \\ 0 & & d_n \end{pmatrix} 
			\]
			$ d_i \in \R  $, $ D^* = D $\qed
	\end{description}
\end{subproof*}

\begin{subcorollary}[Eigenwerte von unitär]
	Sei $ K = \C  $, $ T $ normal.
	Es gilt $ T $ ist unitär $ \iff  $ alle Eigenwerte von $ T $ absolut Betrag $ = 1 $ haben.
\end{subcorollary}
\begin{subproof*}[Eigenwerte von unitären Operatoren]
	\begin{description}
		\item[``$ \implies  $'':] 
			Satz \ref{4.28.4}
		\item[``$ \impliedby  $'':]
			Aus Kor. orthogonal diagonalisierbar haben wir bekommen,
			$ \mathcal{X}  $ eine orthogonale Basis bestehend aus Eigenvektoren.
			\[
				D \coloneqq [T]_{\mathcal{X} } = \begin{pmatrix} z_1 && 0 \\ &\ddots& \\ 0 && z_n \end{pmatrix} 
			\]
			$ z_i $ sind Eigenwerte von $ \C  $.\\
			Beh.: $ D $ ist unitär!\\
			\textbf{Berechne} $ D^* = \overline{D^t} = \begin{pmatrix} \overline{z_1} &&0 \\ &\ddots&\\ 0 && \overline{z_n}  \end{pmatrix}  $\\
			Berechne
			\[
				\mathbb{D}^* = \begin{pmatrix} z_1 \overline{z_1}  && 0 \\ &\ddots& \\ 0 && z_n \overline{z_n}  \end{pmatrix} = \begin{pmatrix} 1 &&0 \\ &\ddots& \\ 0 && 1 \end{pmatrix} \qed
			\]
	\end{description}
	
\end{subproof*}


